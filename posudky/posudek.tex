\documentclass[czech,11pt,a4paper]{article}
\usepackage[utf8]{inputenc}
\usepackage{a4wide}
\usepackage[pdftex,breaklinks=true,colorlinks=true,urlcolor=blue,
  pagecolor=black,linkcolor=black]{hyperref}
\usepackage[czech]{babel}

\pagestyle{empty}

\renewcommand{\arraystretch}{1.3}

\begin{document}

\begin{center}
  {\Large Posudek vedoucího bakalářské práce}
\end{center}

\vspace{.5cm}

\noindent \begin{tabular}{lp{.75\textwidth}}
  {\bf Bakalářská práce:} &
  Zásuvný modul QGIS pro slučování vektorových dat \\
  {\bf Student:} & Tereza Fiedlerová \\
  {\bf Vedoucí:} & Ing. Martin Landa \\
  {\bf Oponent:} & Ing. Jan Růžička Ph.D. \\
\end{tabular}

\vspace{1cm}

Bakalářská práce Terezy Fiedlerové je zaměřena na specifickou oblast
analýzy vektorových dat v GIS, označované v odborné literatuře jako
\uv{vector conflation}.
\\

Text práce je formálně rozdělen do tří částí. První část poskytuje
úvod do problematiky, jsou v ní definovány základní pojmy a shrnuty
základní poznatky v této oblasti. Dále jsou zevrubně popsány
existující nástroje pro slučování vektorových dat, které jsou v
současnosti k dispozici.

V druhé části jsou popsány technologie, které byly využity při vývoji
vlastních nástrojů. Analytické výstupy práce shrnuje část poslední,
kde je popsán jak návrh a implementace knihovny GEOC, tak i navazující
zásuvný modul pro systém QGIS.
\\

Vezmeme-li v potaz, že studentka neměla doposud žádnou větší zkušenost
s~vývojem aplikací postavených na středně rozsáhlých knihovnách lze
konstatovat, že je výsledek práce nadmíru dobrý. Studentka se musela
seznámit s několika knihovnami jako např. JCS či GEOS. Tyto knihovny
jsou navíc napsány v různých programovacích jazycích. První zmíněná
knihovna je implementována v jazyku Java, druhá v C++. Dále se
věnovala objektovému návrhu vlastní knihovny s názvem GEOC. V rámci
knihovny implementovala vybrané algo\-ritmy z oblasti slučování
vektorových dat. V další fázi musela nastudovat rozhraní pro
programování aplikaci QGIS a proces tvorby zásuvných modulů. Jako
programovací jazyk byl zvolen C++. Se všemi těmito úkoly se dokázala
vypořádat nad rámec možností, které bakalářská práce umožňuje. V
závěru práce jsou všechny problémy, které byly v rámci zpracování
práce řešeny, podrobně rozebrány.
\\

Vzhledem k rozsahu a kvalitě zpracování hodnotím předloženou
bakalářskou práce jako nadprůměrnou. Po revizi textu a doplnění po
obsahové stránce by podle mého názoru obstála i jako práce
magisterská. V této souvislosti doufám, že se studentka tématu bude
věnovat i~nadále. Zásuvný modul pro slučování vektorových dat je
komunitou QGIS žádán, a jistě by našel své uživatele. Z tohoto pohledu
by byla škoda, kdyby projekt zůstal nedořešen a jeho výstupy nebyly
veřejně publikovány.
\\

Analytické výstupy práce budou autorkou prezentovány na mezinárodní
konferenci Geoinformatics FCE CTU, která se bude konat ve dnech 11. a
12. července na Stavební fakultě, ČVUT v Praze.
\\

\newpage

Jako vedoucí práce mohu ocenit nejen autorčin nadprůměrnou schopnost
samostatné práce a~tvůrčí potenciál, ale i odpovědnost, se kterou se
daného úkolu zhostila. Práce nejen splňuje požadavky kladené na
bakalářskou práci na studijním programu Geodézie a kartografie včetně
všech formálních náležitostí, ale je i výrazně překračuje. \\

Vzhledem k úrovni zpracování práce hodnotím bakalářskou práci Terezy
Fiedlerové stupněm

\begin{center}
{\bf --- A (výborně)  ---}
\end{center}

a doporučuji státnicové komisi, aby zvážila možnost podání návrhu
děkance fakulty na udělení pochvaly za vynikající zpracování, obhajobu
a přínos bakalářské práce.

\vspace{2cm}

\begin{tabular}{lp{.3\textwidth}c}
V~Solanech dne \today & & \ldots\ldots\ldots\ldots\ldots\ldots\ldots \\
& & Ing. Martin Landa \\
& & Fakulta stavební \\
& & ČVUT v Praze \\
\end{tabular}

\end{document}
