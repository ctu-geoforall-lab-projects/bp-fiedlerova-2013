\documentclass[czech,11pt,a4paper]{article}
\usepackage[utf8]{inputenc}
\usepackage{a4wide}
\usepackage[pdftex,breaklinks=true,colorlinks=true,urlcolor=blue,
  pagecolor=black,linkcolor=black]{hyperref}
\usepackage[czech]{babel}

\pagestyle{empty}

\renewcommand{\arraystretch}{1.3}

\begin{document}

\begin{center}
  {\Large Posudek vedoucího bakalářské práce}
\end{center}

\vspace{.5cm}

\noindent \begin{tabular}{lp{.75\textwidth}}
  {\bf Bakalářská práce:} &
  Zásuvný modul QGIS pro slučování vektorových dat \\
  {\bf Student:} & Tereza Fiedlerová \\
  {\bf Vedoucí:} & Ing. Martin Landa \\
  {\bf Oponent:} & Ing. Jan Růžička Ph.D. \\
\end{tabular}

\vspace{1cm}

Bakalářská práce Terezy Fiedlerové je zaměřena na specifickou oblasti
analýzy vektorových dat v GIS označované v odborné literatuře jako
\uv{vector conflation}.
\\

Text práce je formálně rozdělena do tří částí. První část poskytuje
úvod do problematiky, jsou definovány základní pojmy a shrnuty
poznatky v této oblasti. Dále je zevrubně popsány existující nástoje
pro slučování vektorových dat, kterému se práce dále věnuje.

V druhé části jsou popsány technologie, které byly dále využity při
vývoji vlastních nástrojů. Analytické výstupy práce shrnuje část
poslední, kde je popsán návrh a implementace knihovny GEOC a
navazujícího zásuvného modulu pro QGIS. 
\\

Jako vedoucí práce oceňuji pracovitost a nadšení, které studentka
projevovala při zpracování předkládané bakalářské prace. Vezme-li v
potaz, že neměla doposud žádnou zkušenost s vývoje aplikaci
postavených na středně rozsáhlých softwarov7ých projektech, tak lze
konstatovat, že je výsledek práce nadmíru dobrý. Studentka se musela
seznamit s několika knihovnami jako např. JCS či GEOS. Tyto knihovny
navíc nejsou napsány ve stejném programovacím jazyce, první je
navržena v jazyku Java, druhá v C++. Dále se věnovala objektovému
návrhu vlastní knihovny s názvem GEOC, která implementuje vybrané
algoritmy z oblasti slučování vektorových dat. V další fázi si musela
nastudovat rozhraní pro programování aplikaci QGIS a proces tvorby
zásuvných modulů pro tento projekt. Jako programovací jazyk byl zvolen
C++. Se všemi těmito oblasti se studentka dokázala vypořádat v rámci
možností, které bakalářská práce umožňuje. V zavěru práce jsou všechny
problémy, které byly v rámci zpracování práce řešeny, podrobně
rozebrány.
\\

Především vzhledem k rozsahu a kvalitě zpracování analytické části
práce hodnotím předloženou bakalářskou práce jako nadprůměrnou. Po
revizi textu a doplnění po obsahové stránce by podle mého názoru
obstála i jako diplomová práce. V této souvislosti doufám, že se
studentka tématu bude věnovat nadále. Koneckonců plán budoucích prací
co se týče knihivny GEOC a především zásuvného modulu QGIS je zmíněn v
závěru předkládané bakalářské práce. Zásuvný modul pro slučování
vektorových dat je komunitou QGIS žádán a jistě by našel své
uživatele. Z tohoto pohledu by bylo škoda, kdyby projekt zůstal
nedořešen a nebyl publikován. 

Analytické výstupy práce budou autorkou prezentovány na mezinárodní
konferenci Geoinformatics FCE CTU, která se bude ve dnech 11. a
12. července konat na Stavební fakultě, ČVUT v Praze.

Jako vedoucí práce mohu ocenit nejen její nadprůměrnou schopnost
samostatné práce a tvůrčí potenciál, ale i odpovědnost se kterým se
daného úkolu zhostila. 

Práce splňuje požadavky kladené na bakalářskou práci na studijním
programu Geodézie a kartografie včetně všech formálních náležitostí a
podle mého názoru je výrazně překračuje. \\

Vzhledem k úrovni zpracování práce hodnotím bakalářskou práci Terezy
Fiedlerové stupněm

\begin{center}
{\bf --- A (výborně)  ---}
\end{center}

a doporučuji státnicové komisi, aby zvážila možnost podání návrhu
děkance fakulty na udělení pochvaly za vynikající zpracování, obhajobu
a přínos bakalářské práce.

\vspace{2cm}

\begin{tabular}{lp{.3\textwidth}c}
V~Praze dne \today & & \ldots\ldots\ldots\ldots\ldots\ldots\ldots \\
& & Ing. Martin Landa \\
& & Fakulta stavební \\
& & ČVUT v Praze \\
\end{tabular}

\end{document}
