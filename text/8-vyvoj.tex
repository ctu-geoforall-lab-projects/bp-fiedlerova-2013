\chapter{Další vývoj}
\label{8-vyvoj}

V~rámci dalšího vývoje zásuvného modulu i~knihovny \zk{GEOC}
by bylo vhodné vylepšit některé algoritmy a~také funkcionalitu.
Některé možné úpravy algoritmů jsou už zmíněny v~části 
\ref{7-problemy}, proto je zde již neopakuji. 
Samozřejmě vždy lze optimalizovat kód a~zvýšit efektivitu aplikace,
tato kapitola však obsahuje spíše konkrétní návrhy, co by se dalo 
do~programu doplnit, aby měl širší uplatnění.

Nejbližším krokem by mělo být vytvoření \textit{Makefile}\footnote{Soubor 
určující postup nástroje \textit{make} při překladu zdrojových
souborů do~binárních souborů.} a~konfiguračního souboru, který by 
uživatelům umožnil snadné zprovoznění knihovny a~zásuvného modulu
s~využitím standardních nástrojů.

\section{Knihovna GEOC}
\label{geoc-vyvoj}

\subsubsection{Zpracování po~dlaždicích}
\label{dlazdice-vyvoj}

Větší efektivita při zpracování rozsáhlých dat je v~\zk{GIS} 
aplikacích zajištěna většinou jejich zpracováním po~částech. Je možné
si data rozdělit na~pravidelné dlaždice a~každou tuto část upravovat
samostatně. Ještě vhodnější způsob u~této aplikace je dle \cite{freitas}
rozčlenění dat pomocí shlukových algoritmů a~následné rozdělení
na~jednotlivá pole Voronoi teselací\footnote{Rozdělení roviny 
na~nepravidelné Voronoi polygony, které jsou duální k~trojúhelníkům
\zk{DT}.}.

Bez tohoto postupu je využití programu omezeno pouze na~menší objem
dat. Uživatel by si sice mohl data na~části rozdělit sám a~zpracovávat
je postupně, to je však nepohodlné a~časově náročné, nehledě na~to,
že by musel následně řešit problém úpravy prvků na~styku jednotlivých 
částí.


\subsubsection{Další algoritmy}
\label{dalsi-vyvoj}

Zatím jsou v~knihovně \zk{GEOC} implementovány pouze některé
základní algoritmy pro slučování map. V~rámci dalšího vývoje by
bylo možné postupně doplňovat další pro její širší využití.

V~současné době je v~oblasti slučování vektorových map 
(\textit{conflation}) snaha o~hledání společných elementů 
dvou datasetů nejen na~základě geometrie, ale i~s~přihlédnutím
k~atributům prvků. Zakomponování tohoto přístupu do~algoritmu 
pro určení odpovídajících si prvků je proto logicky vhodným 
dalším krokem jeho vývoje.


\section{Zásuvný modul Conflate}
\label{plugin-vyvoj}

\subsubsection{Protokol a~výsledek zpracování}
\label{protokol-vyvoj}

U~zásuvného modulu by bylo vhodné usnadnit uživateli vyhledání
a~opravu výsledných nevalidních geometrií. Nyní si musí sám
vyhledat prvky dle identifikátorů v~protokolu. Ideálně by
mohl být po kliknutí na~daný identifikátor v~okně projektu zobrazen
a~označen příslušný prvek.

Co se týče protokolu, lze ještě doplnit možnost jeho uložení
do~textového souboru.

\subsubsection{Volba toleranční vzdálenosti}
\label{tolerance-vyvoj}

K~lepšímu a~pohodlnějšímu spojení map by zajisté vedlo doporučení 
toleranční vzdálenosti, jelikož právě její volba může vést k~nereálným 
výsledkům. Nezkušený uživatel by tak nemusel metodou \uv{pokus-omyl} 
zjišťovat, jak tento parametr na\-stavit, pouze by ho mohl upravit dle svých 
představ. Jak už bylo několikrát řečeno, vyhovujícím návrhem by byl průměr 
z~několika nejkratších segmentů prvků dané vrstvy.