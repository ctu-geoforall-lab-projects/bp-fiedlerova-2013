\chapter{Další vývoj}
\label{8-vyvoj}

V~rámci dalšího vývoje zásuvného modulu i~knihovny \zk{GEOC}
by bylo vhodné vylepšit některé algoritmy a~také funkcionalitu.
Některé možné úpravy algoritmů jsou už zmíněny v~části 
\ref{7-problemy}, proto je zde již neopakuji. 
Samozřejmě vždy lze optimalizovat kód a~zvýšit efektivitu aplikace,
tato kapitola však obsahuje spíše konkrétní návrhy, co by se dalo 
do~programu doplnit, aby měl širší uplatnění.

Nejbližším krokem by mělo být vytvoření \textit{Makefile}\footnote{
Soubor určující postup nástroje \textit{make} při překladu zdrojových
souborů do~binárních souborů}. a~konfiguračního souboru, který by 
uživatelům umožnil snadné zprovoznění knihovny a~zásuvného modulu.

\section{Knihovna GEOC}
\label{geoc-vyvoj}

\subsubsection{Vertex Snapper}
\label{vs-vyvoj}

Jelikož se jedná o~nejjednodušší algoritmus pro spojování vektorových map,
nemusí být výsledek funkce \texttt{Vertex\-Snapper} vždy zcela podle
představ uživatele. Pro~širší využitelnost by bylo dobré usnadnit
uživateli editaci nevalidních prvků vzniklých zpracováním, především
takových, kdy se celý prvek přichytil do jednoho bodu. Po~proběhnutí
zarovnání vrstev by se jednoduše objevila nabídka, kde by uživatel mohl
zvolit jednu z~těchto možností (popř. i~nějakých dalších). 

% FORMÁTOVÁNÍ !!!!!!!!
\begin{itemize}
% \itemsep5pt  \parskip0pt \parsep0pt % to by tu vůbec nemělo být
 \item export nevalidních či bodových prvků do nové vrstvy 
 \item smazání všech nyní bodových prvků
 \item zrušení provedeného přichycení u~nevalidních prvků
\end{itemize}

K~lepším výsledkům by zajisté vedlo i~doporučení toleranční vzdálenosti,
jelikož právě její volba může vést k~nereálným výsledkům. Uživatel by
tak nemusel metodou \uv{pokus-omyl} zjišťovat, jak tento parametr 
nastavit, pouze by ho mohl upravit dle svých představ. Jak už bylo
několikrát řečeno, vyhovujícím návrhem by byl průměr z~několika
nejkratších segmentů prvků dané vrstvy. Tato funkce by se navíc dala
použít i~pro zpracování pomocí \texttt{Coverage\-Alignment}.


%\section{MatchingGeometry}
%\label{mg-vyvoj}

% různá kritéria pro polygony, linie, body


%section{CoverageAlignment}
%label{ca-vyvoj}

% přidání bodů do geometrií, pokud je mezi prvkem ref. a sub. vrstvy příliš velký rozdíl


% modifikace algoritmu tak, aby bylo možné spojit i vrstvy o stejné přesnosti - průměr prvků


% vylepšení algoritmu pro nalezení odpovídajících si bodů pro vrcholy tinu tak, aby měl
% správný tvar a velikost

\subsubsection{Zpracování po~dlaždicích}
\label{dlazdice-vyvoj}

 % zpracování po dlaždicích
Větší efektivita při zpracování rozsáhlých dat je v~\zk{GIS} 
aplikacích zajištěna většinou jejich zpracováním po~částech. Je možné
si data rozdělit na~pravidelné dlaždice a~každou tuto část upravovat
samostatně. Ještě vhodnější způsob u~této aplikace je %(dle Freitas, Alfonso)
rozdělení dat pomocí shlukových algoritmů a~následné rozdělení
na~jednotlivá pole Voronoi teselací\footnote{rozdělení roviny 
na~nepravidelné Voronoi polygony, které jsou duální k~trojúhelníkům
\zk{DT}}.

Bez tohoto postupu je využití programu omezeno pouze na~menší data.


\subsubsection{Další algoritmy}
\label{dalsi-vyvoj}

Zatím jsou v~knihovně \zk{GEOC} implementovány pouze některé
základní algoritmy pro slučování map. V~rámci dalšího vývoje by
bylo možné postupně doplňovat další pro její širší využití.

% zahrnutí atributů
V~současné době je v~oblasti slučování vektorových map 
(\textit{conflation}) snaha o~hledání společných elementů 
dvou datasetů nejen na~základě geometrie, ale i~s~přihlédnutím
k~atributům prvků. Zakomponování tohoto přístupu do~algoritmu 
pro určení odpovídajících si prvků je proto logicky vhodným 
dalším krokem jeho vývoje.

% úprava topologie před samotným spojením

\section{Zásuvný modul \textit{Conflate}}

U~zásuvného modulu by bylo vhodné usnadnit uživateli vyhledání
a~opravu výsledných nevalidních geometrií. Nyní si musí sám
vyhledat prvky dle identifikátorů v~protokolu. Ideálně by
mohl být po kliknutí na~daný identifikátor v~okně projektu zobrazen
a~označen příslušný prvek.

Co se týče protokolu, lze ještě doplnit možnost jeho uložení
do~textového souboru.
