% Soubory musí být v kódování, které je nastaveno v příkazu \usepackage[...]{inputenc}

 \documentclass[%
% draft,        % Testovací překlad
  12pt,         % Velikost základního písma je 12 bodů
  a4paper,      % Formát papíru je A4
  twoside,      % Jednostranný tisk
% Z následujicich voleb lze použít jen jednu:
% dvipdfm       % výstup bude zpracován programem 'dvipdfm' do PDF
% dvips	        % výstup bude zpracován programem 'dvips' do PS
 pdftex	        % překlad bude proveden programem 'pdftex' do PDF
]{report}       % Dokument třídy 'zpráva'
%

\usepackage[czech, english]{babel}

\usepackage[utf8]	% Kódování zdrojových souborů je UTF8
	{inputenc}      % Balíček pro nastavení kódování zdrojových souborů
\usepackage[dvips]{graphicx}    % Balíček 'graphicx' pro vkládání obrázků
                                % Nutné pro vložení log školy a fakulty při
                                % volbě 'vutstyle' balíčku 'thesis'

\usepackage{wrapfig}        % pro obtékání textu kolem obrázků a tabulek 
%\usepackage{indentfirst}   % pro odsazení prvního odstavce 
\usepackage{cmap}           % pro správné mapování znaků v pdf na Unicode
\usepackage[T1]{fontenc}    % totéž  

\usepackage[
    nohyperlinks    % Nebudou tvořeny hypertextové odkazy do seznamu zkratek
]{acronym}          % pro sazby zkratek a symbolů
                    % Nutné pro použití prostředí 'seznamzkratek'
                    % při volbě 'cvutstyle' balíčku 'thesiscvut'

\usepackage[
    unicode,                % záložky a informace budou v kódování unicode
    breaklinks=true,        % hypertextové odkazy mohou obsahovat zalomení řádku
    hypertexnames=false,    % názvy hypertextových odkazů nezávisle na názvech TeXu
    colorlinks=true,        % kolem odkazů nebude rámeček, odkazy budou barvou:
    citecolor=black,
    filecolor=black,
    linkcolor=black,
    urlcolor=black
]{hyperref}         % pro sazbu hypertextových odkazů
                    % Nutné pro použití příkazu 'nastavenipdf'
                    % při volbě 'cvutstyle' balíčku 'thesiscvut'

\usepackage{url}        %zalamovani radku v internetovem odkazu
\usepackage{fancyhdr}   %zahlavi, zapati
\usepackage[%
cvutstyle,          % Úvodní listy budou podle zvyklostí na ČVUT
% Z následujících voleb lze použít pouze jednu
%   diploma         % sazba diplomové práce
    bachelor        % sazba bakalářské práce
%   semestral       % sazba zprávy semestrálního projektu
]{thesiscvut}       % Balíček pro sazbu studentských prací

\newif\ifweb
\ifx\ifHtml\undefined % Mimo HTML.
    \webfalse
\else % V HTML.
    \webtrue
\fi 

%%%%%%%%%%%%%%%%%%%%%%%%%%%%%%%%%%%%%%%%%%%%%%%%%%%%%%%%%%%%%%%%%
%%%%%%%%%%% Definice informací o dokumentu  %%%%%%%%%%%%%%%%%%%%%
%%%%%%%%%%%%%%%%%%%%%%%%%%%%%%%%%%%%%%%%%%%%%%%%%%%%%%%%%%%%%%%%%

%% Název práce
\nazev{Zásuvný modul QGIS pro slučování vektorových dat}{QGIS plugin for vector conflation}

%% Jméno a příjmení autora
%% Příjmení bude na českých stranách samo vysázeno VELKÝMI písmeny
\autor{Tereza}{Fiedlerová}

%% Jméno a příjmení vedoucího práce včetně titulů
\garant{Ing. Martin Landa}

%% Označení oboru studia
\oborstudia{Geoinformatika}{}

%% Označení ústavu
\ustav{Katedra mapování a kartografie}{}

%% Rok obhajoby
\rok{2013}

%Mesic obhajoby
\mesic{červen}

%% Místo obhajoby
%% Na titulních stránkách bude automaticky vysázeno VELKÝMI písmeny
\misto{Praha}

%% Abstrakt
\abstrakt	{Abstrakt v~češtině}%
			{Abstract in English}

%% Klíčová slova
\klicovaslova	{Klíčová slova v~češtině}%
				{Keywords in English}
%%%%%%%%%%%%%%%%%%%%%%%%%%%%%%%%%%%%%%%%%%%%%%%%%%%%%%%%%%%%%%%%%%%%%%%%

%%%%%%%%%%%%%%%%%%%%%%%%%%%%%%%%%%%%%%%%%%%%%%%%%%%%%%%%%%%%%%%%%%%%%%%%
%% Nastavení polí ve Vlastnostech dokumentu PDF
%%%%%%%%%%%%%%%%%%%%%%%%%%%%%%%%%%%%%%%%%%%%%%%%%%%%%%%%%%%%%%%%%%%%%%%%
%% Při vloženém balíčku 'hyperref' 
%% lze použít příkaz '\nastavenipdf'
\nastavenipdf
%%%%%%%%%%%%%%%%%%%%%%%%%%%%%%%%%%%%%%%%%%%%%%%%%%%%%%%%%%%%%%%%%%%%%%%

%% Definice standardní přípony pro vkládané obrázky
%% Při převodu programu 'dvipdfm' nebo při překladu programem
%% 'pdftex' lze vkládat bitmapové obrázky.
%%\DeclareGraphicsExtensions{.png}
%% Při převodu programem 'dvips' lze vkládat pouze obrázky 'eps'
%\DeclareGraphicsExtensions{.eps}

%%% Začátek dokumentu
\begin{document}
\catcode`\-=12  % pro vypnuti aktivniho znaku '-' pouzivaneho napr. v \cline 

% aktivace záhlaví
\zahlavi

% předefinování vzhledu záhlaví
\renewcommand{\chaptermark}[1]{%
	\markboth{\MakeUppercase
	{%
	\thechapter.%
	\ #1}}{}}

% Vysázení přebalu práce
\vytvorobalku

% Vysázení titulní stránky práce
\vytvortitulku

% Vysázení listu zadani
\stranka{}%
	{\sffamily\Huge\centering\ ZDE VLOŽIT LIST ZADÁNÍ}%
	{\sffamily\centering Z~důvodu správného číslování stránek}

% Vysázení stránky s abstraktem
\vytvorabstrakt

% Vysázení prohlaseni o samostatnosti
\vytvorprohlaseni

% Vysázení poděkování
\stranka{%nahore
       }{%uprostred
       }{%dole
       \sffamily
	\begin{flushleft}
		\large
		\MakeUppercase{Poděkování}
	\end{flushleft}
	\vspace{1em}
		%\noindent
	\par\hspace{2ex}
	{Chtěla bych poděkovat vedoucímu ............... práce za připomínky a pomoc při zpracování této práce. Dále bych chtěla poděkovat \dots}
}

% Vysázení obsahu
\obsah

\chapter{Úvod}
\label{1-uvod}

Tato bakalářská práce se zabývá tématem slučování vektorových map
(\textit{vector - to - vector conflation}).

Spojování geografických dat z~různých zdrojů je jedním z~aktuálních
problémů v~oblasti geografických informačních systémů \zk{GIS}. 
S~rostoucí dostupností digitálních dat začala vzrůstat i~potřeba 
tato data kombinovat a~porovnávat. Na~internetu je volně k~dispozici 
spousta různých prostorových datasetů. Jejich přesnost, úplnost, 
atributy a~další vlastnosti jsou však velmi odlišné. Některé jsou 
geometricky přesné, ale neobsahují potřebné atributy nebo
některé žádoucí detaily. Méně přesné mapy s~potřebnými informacemi
lze upravit pomocí těch přesnějších, abychom získali požadovaný
výsledek. To a~veškeré podobné úlohy, které nás napadnou, jsou 
předmětem slučování map.

Pod anglickým pojmem \textit{conflation} se skrývá mnoho různých
procesů, úloh a~činností. Jejich cílem může být obohacení jedné 
mapy o~atributy či prvky z~mapy druhé, aktualizace datasetu, kombinace 
dvou map o~stejné přesnosti, detekce změn a jiné. Záměrem je vždy 
vytvoření nové mapy, z~níž lze vyčíst informace, které by nebylo možné
získat pouze z~jednoho vstupního zdroje.

Vzhledem k~velkému množství datasetů a~jejich rozsahu je automatizace
výše uvedených úloh logickým a~nevyhnutelným krokem. Ne vždy je však
automatické zpracování zárukou úspěchu. Každý algoritmus má své limity
a~nikdy nelze identifikovat $100\%$ odpovídajících si prvků. To může
být ještě sníženo nejednoznačností a~kvalitou vstupních dat. Proto
je stále třeba dodatečná ruční úprava a~kontrola dat. Ta je však
oproti úspoře času díky automatickému zpracování často zanedbatelná.

Výstupem práce má být zásuvný modul, který umožní spojení
vektorových map. Hlavním cílem je však vyhledání způsobů
řešení tohoto problému a~implementace některých z~nich.
Předpokládá se, že vznikne knihovna s~algoritmy pro slučování
vektorových dat, kterou bude možné dále rozšiřovat a~doplňovat
další funkcionalitu. Zásuvný modul bude využívat tuto knihovnu
a~s~jejím rozšiřováním bude možné i~sem přidávat další funkce.   

V~následující kapitole bude podrobněji popsána problematika 
slučování map (\textit{conflation}). V~kapitole \ref{3-nastroje}
pak budou představeny některé existující nástroje v~této oblasti.
Po~krátkém výčtu použitých technologií (kapitola \ref{4-technologie})
už se bude práce zabývat samotnou tvorbou knihovny a~zásuvného
modulu. Kapitola \ref{5-geoc} bude popisovat jednotlivé algoritmy
nově vznikající knihovny a~jejich implementaci. Následovat
bude popis zásuvného modulu a~jeho funkcionality včetně
ukázek jeho použití (kapitola \ref{6-plugin}). Poslední dvě kapitoly
budou věnovány řešení problémů a~nápady na~další vývoj 
a~vylepšení vzniklé aplikace.  
\chapter{Conflation}
\label{2-conflation}

\section{Definice 'conflation'}
\label{definice}

\textit{Conflation} neboli \textit{map matching, map merging, rubber sheeting} lze do~češtiny přeložit jako slučování či spojování map. Překlad tohoto pojmu však není vždy 
jasný, jelikož se jím označuje více různých úloh, procesů či činností. Proto zde uvádím několik definic z~různýc zdrojů. 

\begin{itemize}[leftmargin=*]
 \item Proces, jehož cílem je geometricky upravit (posunem, transformací) digitální data zobrazující stejné území (obvykle pořízená v~jiném čase) tak, aby byla 
    vzájemně geograficky korektní a~vzájemně se překrývaly. %% slovem korektní si nejsem jistá 
    
    (huic.com)

  \item Proces sjednocení dvou rozdílných datasetů.
    
    (Blasby, Davis, Kim, Ramsey: GIS Conflation Using Open Source Tools)

  \item Zabývá se kombinováním dat z~různých zdrojů. Jedná se o~jeden z~aktuálních problémů v~GIS.
   
    (Freitas, Afonso: Distributed Vector Based Spatial Data Conflation Services)

  \item Soubor funkcí a~procedur, které zarovnávají prvky jedné GIS vrstvy k~prvkům jiné a~následně převádí atributy mezi~těmito vrstvami.
   
    (http://www.princegeorgescountymd.gov/Government/AgencyIndex/\newline OITC/GIS/glossary.asp)

  \item Proces sladění poloh odpovídajících si prvků v~různých datových vrstvách. Funkce pro spojení map (\textit{conflation}) provádějí toto sladění tak, aby se 
    odpovídající si prvky přesně překrývaly. 
    
    (http://maps.unomaha.edu/Peterson/GIS/notes/GISAnal1.html,\newline http://www.gov.ns.ca/snsmr/land/standards/post/manual/appedxa1.asp)

  \item Sada činností, které vzájemně přizpůsobují prvky dvou geografických datových vrstev a~následně převádí atributy jedné vrstvy do~druhé. %% možná spíše zarovnávají než přizpůsobují
    
    (http://wiki.gis.com/wiki/index.php/GIS\_Glossary/C)
  
  \item \textit{Feature conflation} je proces kombinování geografických informací z~překrývajících se zdrojových dat, který zachovává přesnost dat, minimalizuje nadbytečná 
    data a~předchází konfliktům v~datech. %% nebo rozdíly?
    \textit{Conflation of geospatial data} (geoprostorových dat) je spojení či sladění dvou různých prostorových datasetů zahrnujících stejné území.
    
    \citation{gisencyclopedia} (Shekhar, Xiong: Encyclopedia of GIS)

\end{itemize}

\section{Historie} % asi ještě rozšířit, viz enc. of gis - geospatial conflation
\label{historie}

Až do 80.~let 20.~století bylo pořízení digitálních dat velmi drahé a~proto se často nestávalo, že by nějaká firma vlastnila více digitálních map jediného území. S~vývojem
počítačů a~digitálních technologií se však tato data stávala stále dostupnějšími a~náhle vyvstala potřeba kombinovat data o~jednom území z~více zdrojů a~provádění aktualizace
těchto dat. 

Jako první se k~takovému množství dat dostaly pochopitelně různé vládní instituce. Ačkoli první článek o~problematice spojování geografických dat z~více zdrojů vyšel už v~roce
 1981 (\textit{M. White: The~Theory of~Geographical Data Conflation}), k opravdovému rozvoji došlo až po~roce 1985. V~té době totiž vznikl projekt, jehož cílem bylo propojení 
mapových souborů organizací US~Census Bureau a US Geological Society. Vzhledem k~množství dat bylo nutné celý proces co nejvíce automatizovat. Použitý algoritmus, jehož 
hlavním autorem je Alan Saalfeld, byl založen na~nalezení odpovídajících si prvků a~následné transformaci dat. 

S~narůstající dostupností dat a~zveřejněním této myšlenky přibývalo i~menších firem zabývajících
se problémem kombinace mapových souborů z různých zdrojů. Původní algoritmy brali v~potaz pouze geometrickou podobnost prvků, později byly brány v~úvahu
i~jejich topologické vztahy a~s~rozvojem GIS technologií pak i~podobnost atributů jednotlivých prvků. Dnes jde o~jeden z~aktuálních problémů řešených v~oblasti GIS.


\section{Související pojmy} % slovníček dále používaných pojmů atd.
\label{pojmy}

adjustment, alignment, attribute transfer (přenos, převod atributů), geometry (geometrie - jako geometrický prvek), feature (prvek), feature collection (kolekce prvků),
internal conflation (v rámci jednoho datasetu), matching (odpovídající si), reference dataset (referenční dataset), subject dataset (upravovaný dataset), topologie, ... 


\section{Klasifikace 'conflation'}
\label{klasifikace}

\subsection{Dle typu vstupních vrstev}
\label{dle-vstupu}

\begin{enumerate}[leftmargin=*]
  \item \textbf{Imagery-to-Imagery, Raster-to-Raster}
    \subitem Jedná se o~případ, kdy je úkolem na~sebe napasovat dva rastrové mapové soubory. Nejčastěji jde o~ortofoto a~naskenovanou analogovou mapu daného území. Tato úloha
	      se využije, například pokud chceme porovnat starou mapu se~současným stavem reprezentovaným právě leteckým snímkem. Řešení tohoto problému vyžaduje poměrně 
	      složité techniky pro~nalezení odpovídajících si objektů. Velmi důležitým prvkem je kvalita a rozlišení vstupních dat.
  \item \textbf{Vector-to-Imagery, Vector-to-Raster}
    \subitem Kombinace vektorových a~rastrových dat stejného území se využívá často ke~zpřesnění vektorových dat jejich napasováním na~ortofoto. Hlavní náplní této oblasti
	      je vývoj algoritmů umožňující následující: % dopsat referenci na lit. - gis enc. 
	      \begin{enumerate}[leftmargin=*]
	       \item Detekce charakteristických hran rastrového obrazu a~jejich porovnání s~vektorovými daty.
	       \item Využití vektorových dat k~identifikaci hran v~rastru - tzv. \textit{Snakes algorithm}.
	       \item Užití stereo obrazu, výškových dat a~znalostí silničních dat pro~porovnání vektorových a~rastrových dat. % ??????????
	       \item Využití prostorových informací stejně jako dalších vlastností mapy (jako např. barev) k~rozpoznání odpovídajících si prvků.
	      \end{enumerate}
  \item \textbf{Vector-to-Vector}
    \subitem Případem sloučení dvou vektorových datasetů se zabývá tato práce. Jednou ze~specifických a~velmi častých aplikací je navázání silničních sítí, dále aktualizace
	      digitálních dat aj. Existuje mnoho různých algoritmů pro~slučování vektorových dat, přičemž základní přístupy k~řešení problému jsou následující: % dopsat referenci na lit. - gis enc. 
	      \begin{enumerate}[leftmargin=*]
	       \item Sloučení dat za~pomoci algoritmů, které pracují na~základě porovnávání geometrických vlastností prvků.
	       \item Algoritmy, které berou v~potaz podobnost tvarů prvků a~zároveň podobnost jejich atributů.
	       \item Spojení vektorových dat s~neznámým souřadnicovým systémem na~základě rozložení významných bodových prvků (např. křižovatky cest).
	      \end{enumerate}

\end{enumerate}


\subsection{Dle území zobrazovaného vstupními vrstvami}
\label{dle-uzemi}

\begin{enumerate}[leftmargin=*]
  \item \textbf{Horizontální} % - sousedící datasety
    \subitem Za~\textit{horizontal conflation} se označují procesy, které zpracovávají data ze~sousedících území. Cílem je získat mapové soubory, jejichž hranice na~sebe
	      dokonale navazují, a~to pokud možno bez~ztráty přesnosti.
  \item \textbf{Vertikální} % - překrývající se datasety
    \subitem Při~\textit{vertical conflation} obsahují vstupní data překrývající se území. Jde tedy o~dva či více souborů zobrazujících jediné území, třeba i~jen částečně.
	      Může se jednat o~dvě verze té samé mapy nebo o~dva datasety s~nějakými společnými prvky a~vlastnostmi.
	      Výsledkem celého procesu je jediný dataset, jehož přesnost není horší než přesnost původních dat a~obsahuje informace z~obou zdrojů, jde tedy o~vylepšenou, 
	      obsahově bohatší mapu s~odpovídající přesností. 
\end{enumerate}


\section{Obecný postup}
\label{postup}

Obecný postup při~slučování vektorových map z~více zdrojů se skládá z~několika kroků, které jsou uvedeny níže. Jako u~většiny podobných operací je třeba nejdříve provést 
přípravu dat a~následně data zpracovat a~upravit.

\begin{enumerate}[leftmargin=*]
  \item \textbf{Předzpracování dat}
    \subitem Tento krok slouží k~zajištění kompatibility vstupních dat tak, aby bylo možné je porovnávat. Obvykle spočívá v~převedení vstupních datasetů do~stejného 
	      souřadnicového systému, zajištění stejných základních jednotek a~dalších měnších úpravách. 
  \item \textbf{Kontrola kvality dat a~topologické správnosti vrstev}
    \subitem Zde se kontroluje vnitřní konzistence dat každé vstupní vrstvy. V~tomto případě záleží především na~požadavcích zvoleného algoritmu pro~sloučení map. Může jít
	      například o~odstranění topologických chyb v~dané vrstvě jako jsou nežádoucí drobné překryty či mezery mezi~polygony.
  \item \textbf{Vyhledání odpovídajících si prvků}
      \subitem Následuje vyhledání prvků, které si v~upravovaných datasetech odpovídají, tedy zobrazují stejný předmět ve~skutečnosti. Tento krok je nezbytný proto, aby bylo
		možné rozhodnout, jak na~sebe vstupní vrstvy navazují.
  \item \textbf{Sloučení geometrických prvků a/nebo atributů}
      \subitem Po~rozpoznání odpovídajících si prvků je už možné upravit geometrii či atributy prvku z~jedné vrstvy s~přihlédnutím k~vlastnostem prvku z~jiné vrstvy. 
		Při~procesu slučování různých datasetů lze převádět pouze atributové hodnoty mezi~odpovídajícími si prvky nebo pouze změnit geometrii prvku tak, aby odpovídala
		geometrii prvku z~jiné vrstvy, která bývá označena za~referenční. Ve~složitějších úlohách už je možné převádět zároveň atributy i~geometrii, a~to nejen 
		jednosměrně (tedy z~vrstvy referenční do~vrstvy upravované), ale výsledkem může být vrstva, jejíž geometrie je kombinací geometrických vlastností obou 
		vstupních vrstev.
  \item \textbf{Závěrečné úpravy}
      \subitem Po~provedení automatického a~někdy i~manuálního sloučení datových vrstev je vhodné provést kontrolu výsledku. Často jsou potřeba ještě drobné úpravy, aby výsledná
		data odpovídala počátečním požadavkům. Ne vždy jsou totiž při~automatickém procesu určeny správně všechny odpovídající si prvky a~některé algoritmy mohou
		při~úpravě geometrických vlastností narušit topologickou správnost dat.
\end{enumerate}


\section{Využití - hlavní aplikace}
\label{využití}

\subsection{oblasti, obory}
\label{obory}
kartografie, GIS, výpočetní geometrie, letecká fotogrammetrie, vojenský výcvik a výzkum, krizový management, dopravní mapy - aktualizace dat, trh s nemovitostmi, ...

\subsection{aplikace}
\label{aplikace}
zpřesnění map, aktualizace prostorových dat, georeferencování - záznam dat, detekce chyb, rozdílů, ...


\chapter{Existující nástroje}
\label{3-nastroje}

Tato kapitola se zabývá již existujícími nástroji pro~spojování vektorových map
 z~různých zdrojů (\textit{conflation}).

\section{Proprietární nástroje}
\label{proprietární}

Proprietárních nástrojů umožňujících řešení spojování vektorových map 
existuje celá řada. Některé programy jsou orientovány přímo na~tento problém,
jiné nabízejí funkce pro~slučování map pouze jako vedlejší funkcionalitu 
a~ne vždy je proto možné pomocí nich řešit složitější problémy. Následující
výčet neobsahuje všechen komerční software zabývající se slučováním
vektorových map, ale nejvýznamnější nástroje, které lze pro~zpracování 
použít. Vzhledem k~tomu, že se jedná o~komerční software, nebylo možné 
všechny níže popsané nástroje otestovat, proto jejich popis vychází 
především z~informací dostupných na~oficiálních internetových stránkách 
produktů.


\subsection{ESRI ArcGIS}
\label{arcgis}

V~softwaru ArcGIS 10 existuje několik nástrojů, které lze využít pro~spojování
geo\-grafických dat, přenos atributů a~odstraňování geometrických rozdílů 
mezi~datasety. 

\subsubsection{Spatial Adjustment}

Soubor nástrojů \textit{Spatial Adjustment} systému ArcGIS poskytuje 
základní funkce týkající se této problematiky. Má sloužit především 
k~úpravě dat z~různých zdrojů a~zajistit tak jejich celistvost. Může 
být použito několik metod pro~zarovnání jedné vrstvy či její části 
ke~druhé. Lze provést transformaci, navázání hran (\textit{edge matching})
nebo srovnání překrývajících se dat (\textit{rubber sheeting}).

Pro použití metody transformace je nejdříve třeba označit data, která budou
do~procesu vstupovat. Poté se vyberou dvojice uzlových bodů, které by si měly
odpovídat. Na~výběr je několik typů transformací. Všechny jsou prováděny 
na~základě vybraných dvojic identických bodů. 
%%Tento postup bychom tedy mohly označit za~ruční 'conflation'.

Do~procesu horizontálního zarovnání (\textit{horizontal conflation}) lze 
zařadit funkce nástroje \textit{Edge Match Tool}, který umožňuje navázat 
na~sebe dvě sousedící vrstvy. Zarovnání je poměrně snadné, stačí pouze 
zvolit toleranci (maximální vzdálenost pro navázání prvků) a~označit tímto
nástrojem hranici mezi vrstvami, kde by měly na~sebe navazovat. Ve~vybrané
oblasti se zobrazí indikátory naznačující způsob zarovnání, které lze ještě
ručně upravit. Po~potvrzení se provede zarovnání tak, aby byla zachována
topologie.

Dalším způsobem geometrického spojení vrstev, který \textit{Spatial Adjustment}
nabízí, je \textit{Rubber Sheeting}. V~tomto případě je třeba označit opět 
odpovídající si body a~navíc body, jejichž poloha by se neměla změnit. 
Při~spuštění zarovnání se dočasně vytvoří triangulační síť mezi označenými body
(vzájemně si odpovídající body). Poloha ostatních neoznačených bodů je pak 
vypočtena interpolací v~této síti.

Poslední z~nástrojů pro~kombinaci map je \textit{Attribute Transfer}. Pomocí 
něho je možné převést atributy mezi odpovídajícími si prvky, ale také upravit
jejich geometrii. Po~volbě jednoho či více atributů, které se mají převádět 
mezi~zdrojovou a~cílovou vrstvou lze interaktivně vybírat odpovídající si 
dvojice prvků v~obou vrstvách. Mezi~těmito prvky se převedou atributy a~pokud
je zaškrtnuta volba \textit{Transfer Geometry} neboli \uv{převést geometrii},
změní se geometrie cílového prvku dle~zdrojového. Převod lze provést také 
najednou pro~více vybraných prvků.
%% http://help.arcgis.com/en/arcgisdesktop/10.0/help/index.html#/About_spatial_adjustment_attribute_transfer/001t000000v9000000/
%% http://help.arcgis.com/en/arcgisdesktop/10.0/help/index.html#/About_spatial_adjustment_rubbersheeting/001t000000v3000000/

\subsubsection{Integrate}

Nástroj \textit{Integrate} umožňuje sladění dvou datasetů. Na~vstupu vyžaduje 
dvě či více vrstev, které chceme sladit, a~dále maximální vzdálenost, při~které
lze považovat prvky za~odpovídající si. U~každého datasetu navíc lze zadat 
prioritu (\textit{rank}). Prvky s~nižší prioritou se pak budou zarovnávat 
k~těm s~prioritou vyšší. Tato funkce je velmi užitečná, pokud máme dvě 
překrývající se nebo sousedící vrstvy s~malými rozdíly např. v~důsledku různé
přesnosti.
%% http://help.arcgis.com/en/arcgisdesktop/10.0/help/index.html#//00170000002s000000

\subsection{ConfleX}
\label{conflex}

ConfleX je software pro~automatické spojování vektorových GIS dat,
který pro~automatizaci využívá umělou inteligenci. ConfleX umožňuje zpracování 
i~takových případů, kdy se zdrojová mapa s~cílovou nepřekrývají nebo nejsou 
topologicky identické. Systém porovnává každé dva segmenty z~obou map a~jejich
vztah k~ostatním segmentům, na~základě tohoto postupu pak rozhodne, zda se 
jedná o~stejné prvky či nikoli. Kromě automatického procesu umožňuje program 
i~následnou ruční editaci.

ConfleX je k~dispozici jako samostatná aplikace ale také jako extenze 
programu ArcGIS 9/10.

\subsection{Intergraph GeoMedia Fusion}
\label{geomedia}

Nástroj GeoMedia Fusion firmy Intergraph je navržen pro~úpravu dat, aby byly
topologicky korektní, validaci atributů a~integraci dat. Cílem je umožnit 
snadnou údržbu dat v~rozsáhlých geografických databázích, kde jsou data 
získávána z~různých zdrojů. Nástroj porovnává dva datasety obsahující 
rozdílné reprezentace stejné skutečnosti. Nejdříve automaticky vytvoří 
\textit{conflation links} indikující způsob spojení vrstev, které lze ještě
ručně editovat. Následně umožní geometrie i~atributy těchto dvou reprezentací
sjednotit. GeoMedia Fusion slouží k~úpravě bodových, liniových i~plošných dat
včetně jejich atributů.


\subsection{MapMerger}
\label{mapmerger}

MapMerger je GIS nástroj firmy ESEA zaměřený na~slučování geometrie a~atributů
vektorových map a~kontrolu kvality dat. Umožňuje převod atributů mezi dvěma 
překrývajícími se mapami, navázání hranic dvou sousedících map, přidání prvků 
z~jedné mapy do~druhé, synchronizaci mapy s~její aktualizovanou verzí, 
identifikaci geometrických a~atributových rozdílů mezi~dvěma verzemi té samé 
mapy. 

\section{Open Source nástroje}
\label{open-source}

V~této sféře zatím neexistuje mnoho nástrojů, které by komplexně řešili problém
spojování vektorových map (\textit{conflation}). Většinou jde pouze o~malé 
programy či zásuvné moduly k~větším projektům, pomocí nichž se dá provést manuální
nebo poloautomatické sloučení vektorových datasetů, případně jejich atributů. 
Ovšem většinou je cesta k~dosažení cíle pomocí těchto nástrojů poměrně složitá 
a~ne vždy jsou výsledky takové, jak bychom si představovali. Asi jediným 
ucelenějším nástrojem je knihovna \textit{JCS} implementovaná jako kolekce 
zásuvných modulů v~programu OpenJUMP.

\subsection{JCS - Java Conflation Suite}
\label{JCS}

\textit{JCS} je \textit{open source} knihovna napsaná v~jazyce Java, která 
zahrnuje API a~soubor interaktivních nástrojů, které slouží ke~slučování 
prostorových datasetů. Byla vyvinuta společností Vivid Solutions, Inc. 
Obsahuje funkce umožňující provádění různých procesů spojených se~spojování
vektorových map, které jsou zaměřeny především na~polygonové případně liniové
datasety, co se týče bodových vrstev, je její funkcionalita poměrně omezená. 
Knihovna \textit{JCS} je závislá na~knihovně \textit{JTS - Java Topology Suite},
která poskytuje základní geometrické funkce pro~práci s~prostorovými
daty. Obě knihovny jsou navrženy v~souladu s~OGC specifikací 
\textit{Simple Features} \footnote{Specifikace týkající se 2D prostorových 
predikátů a~funkcí}. \textit{JCS} vznikla v~rámci projektu JUMP Unified 
Mapping Platform. Je poskytována pod~licencí LGPL.

\subsubsection{Architektura}
\label{jcspic}
  \begin{figure}[hbt]
    \centering
      \includegraphics[width=250pt]{./pictures/JCS_Architecture.png}
      \caption{Architektura JCS 
	  (\todo{zdroj})}
      \label{fig:architektura}
  \end{figure}


Knihovna \textit{JCS} používá pro~vizualizaci dat a~interakci JUMP WorkBench
a~API. Pro~poskytování základní geometrické funkcionality je pak využíváno 
knihovny \textit{JTS}. JUMP API umožňuje vstup a~výstup prostorových dat 
a~další funkcionalitu s~nimi spojenou. Jádro \textit{JCS} tvoří Conflation API
obsahující algoritmy pro~kontrolu a~slučování prostorových dat 
(\textit{conflation}). Funkce \textit{JCS} jsou v~projektu OpenJUMP 
implementovány v~podobě kolekce zásuvných modulů - \textit{QA, Conflate, 
RoadMatcher}.

\subsubsection{OpenJUMP projekt}

OpenJUMP je \textit{open source} projekt, který vyvinula firma Vivid Solutions,
Inc. Jedná se o~GIS software, který umožňuje základní práci 
s~prostorovými daty v~rastrové či vektorové podobě a~jejich atributy.

\subsubsection{JTS - Java Topology Suite}
\label{kap:jts}

Knihovna \textit{JTS} je jedním ze~základních prvků projektu OpenJUMP.
Je napsána v~jazyce Java a~poskytována pod~licencí LGPL. Obsahuje třídy 
pro~reprezentaci geometrických objektů a~základní funkce pro~práci 
s~prostorovými daty dle~specifikace \textit{Simple Features} pro~SQL 
od~OGC. Kromě tříd reprezentujících geometrické prvky dle~zmíněné
specifikace zahrnuje další podpůrné třídy pro~reprezentaci seznamu souřadnic,
aplikaci geometrického filtru (např. při~transformaci), uchovávání informace
o~maximální a~minimální souřadnici objektu a~jiné. Dále jsou součástí této 
knihovny třídy umožňující geometrické výpočty jako je vzájemná polohu bodu 
a~linie, výpočet průsečíku, prostorové analýzy, test polohy bodu a~uzavřené
oblasti atd. 

\subsubsection{Funkcionalita JCS} %% JINÝ NÁZEV?

Projekt \textit{JCS} neumožňuje změnu souřadnicového systému. Proto je
automaticky přepokládáno, že vrstvy vstupující do~zpracování mají stejný
prostorový referenční systém. Souřadnice bodů výstupních vrstev mají vždy
desetinnou přesnost.  %% JE TO SPRÁVNĚ ??

Při většině výpočtů v~zásuvných modulech \textit{JCS} je využívána tzv. 
Hausdorffova metrika. Na~rozdíl od~euklidovské metriky totiž nezkoumá jen
nejkratší vzdálenost mezi prvky, ale i~vzdálenost největší.
Zohledňuje tedy do~jisté míry i~topologické vztahy.  %% ML: Kostrbate... opraveno

Na~výpočet je Hausdorffova vzdálenost poměrně složitá. Proto se v~algoritmech
použitých v~\textit{JCS} používá spíše vrcholová Hausdorffova vzdálenost 
(\textit{Vertex Hausdorff Distance}), která není vztažena ke~geometrickému 
prvku, ale pouze k~jeho vrcholům. Tato varianta Hausdorffovy vzdálenosti 
ve~většině případů vrací stejně dobré výsledky. 

Postup spojování vektorových datasetů pomocí knihovny \textit{JCS} je založen
na~nalezení geometrických rozdílů mezi~oběma mapami a~následném odstranění 
těchto rozdílů.

K~detekci geometrických rozdílů je použit algoritmus pro~prostorové rozdíly.
Tento algoritmus funguje tak, že postupně porovnává geometrické prvky z~obou 
datasetů popřípadě pouze jejich jednotlivé části a~pokud se tyto prvky shodují,
označí je jako odpovídající si. Výsledkem jsou pak ty prvky z~obou datasetů, 
ke~kterým nebyly nalezeny žádné odpovídající prvky. Nalezení odpovídajících si
prvků probíhá následovně. Pokud je požadována přesná shoda, provede se 
testování, zda jsou prvky stejného geometrického typu a~zda se rovná jejich 
obsah. Porovnání obsahu se zakládá na~porovnávání jednotlivých komponent 
a~seznamu bodů daných geometrií. Ne vždy však je předpokládána přesná shoda,
proto je možné určit i~prvky, které splňují podmínku danou tolerancí. Zde se
provádí porovnání Hausdorffovy vzdálenosti mezi~prvky s~touto tolerancí, 
nepočítá se přitom přímo tato vzdálenost. Rozhodující je, jestliže obalová 
zóna o~velikosti vzdálenostní tolerance prvního prvku obsahuje prvek druhý 
a~naopak. Tato podmínka je ekvivalentní k~podmínce, že Hausdorffova vzdálenost
musí být menší nebo rovna toleranci. 

%% ML: doplnit ilustraci? .. stejná jako u popisu meho algoritmu CoverageAlignment

Pro~odstraňování překrytů či mezer neexistuje exaktní algoritmus,
ale jsou využívány různé heuristiky poskytující dobré topologické výsledky.

Následující výčet podává přehled nejčastějších úloh, které je možné 
s~tímto nástrojem řešit.

\begin{itemize}
 \item Jako \textit{Coverage Cleaning} je  označován proces hledání
    a~odstraňování topologických chyb - mezer a~překrytů v~rámci jedné
    vektorové mapy tvořené polygony či multipolygony. Detekce nežádoucích
    mezer mezi~polygony je založena na~rozpoznání blízkých liniových segmentů,
    kde blízkost je určena zvolenou vzdálenostní tolerancí. 

 \item Další často řešenou úlohou je \textit{Boundary Alignment}, což by
    bylo možné přeložit do češtiny jako \uv{zarovnání či navázání hranic}. 
    Cílem je napojit k~sobě dvě vektorové vrstvy, které zobrazují sousedící
    území, tak, aby se mezi~nimi nevyskytovaly nežádoucí mezery či překryty.
    Výsledkem jsou tedy plynule navazující datasety, které tvoří bezešvou mapu.
    Při této úloze je nutné zvolit přesnější referenční vrstvu, jejíž 
    geometrické vlastnosti se nezmění.

 \item U \textit{Coverage Alignment} máme na~vstupu naopak dvě vektorové
    mapy zobrazující to samé území nebo alespoň jeho část. Tyto mapy se 
    tedy výrazně překrývají. Úkolem je upravit méně přesnou vrstvu tak,
    aby odpovídala vrstvě referenční popřípadě její části, pokud se vrstvy
    úplně nepřekrývají. Proces spočívá v~posunutí vrcholů polygonů upravované
    vrstvy do~blízkých vrcholů vrstvy referenční.

 \item Poměrně specifickou, ale velmi častou úlohou je \textit{Road Network 
    Matching} neboli \uv{spojování silničních sítí}. Na~vstupu máme dvě 
    vektorové mapy té samé silniční sítě. Při~této úloze hledáme podobnost
    mezi~liniovými prvky obou datasetů, které označíme za~odpovídající si.
    Poté je vytvořena nová vrstva silniční sítě, která obsahuje odpovídající
    si prvky, přičemž při~mírných odlišnostech použije liniové prvky z~přesnější
    mapy. \textit{JCS} bohužel zatím neumožňuje automatické provedení této
    úlohy. Pouze nalezne odpovídající si prvky a~další kroky už je nutné provést
    manuálně.

\item Úloha \textit{Geometry Difference Detection}, v~češtině \uv{detekce
    geometrických rozdílů}, na~rozdíl od~předchozích nijak neupravuje vstupní
    vrstvy ani z~nich netvoří jiné. Cílem je pouze nalézt rozdíly mezi vstupními
    datasety. Nejčastěji je používána pro~rozpoznání změn mezi~dvěma verzemi
    jedné vektorové mapy (např. po aktualizaci).
\end{itemize}

\subsubsection{Popis zásuvných modulů}

Zásuvné moduly ze skupiny \textit{QA - Quality Assurance} umožňují najít
geometrické rozdíly a~nesrovnalosti mezi~datasety, ale také v~rámci jediného
datasetu. Funkce zde obsažené neslouží k~opravě či propojení vrstev, ale 
pouze k~identifikaci geometrických rozdílů.

\begin{itemize}
 \item \textit{Find Misaligned Segments} - slouží k~nalezení segmentů 
    ze~dvou datasetů, které by si v~rámci dané tolerance měli odpovídat,
    ale je mezi~nimi mezera či překryt. 
 \item \textit{Find Overlaps} - najde překrývající se prvky ze~dvou datasetů.
 \item \textit{Find Coverage Gaps} - umožňuje nalézt mezery mezi~polygony
    jednoho datasetu, které jsou užší než zadaná vzdálenostní tolerance
    a~zároveň je mezi~hranami polygonů, které tvoří tuto mezeru, úhel menší
    než daná úhlová tolerance.
 \item \textit{Find Coverage Overlaps} - najde všechny překryty mezi~polygony
    v~rámci jednoho datasetu, respektive všechny překrývající se polygony.
 \item \textit{Find Close Vertices} - identifikuje body (samostatné body,
    vrcholy linií či polygonů) ze~dvou různých datasetů, jejichž vzdálenost
    je menší než daná tolerance.
 \item \textit{Find Offset Boundary Corners} - slouží k~nalezení hranic
    polygonů ze~dvou sousedících vektorových map, které by na~sebe měly
    navazovat, ale je mezi~nimi posun menší než zadaná tolerance.
 \item \textit{DiffSegmentsPlugin} - identifikuje liniové segmenty, které
    jsou obsaženy pouze v~jedné ze~zadaných vrstev, nikoli v~obou dvou.
 \item \textit{DiffGeometryPlugin} - funguje stejně jako předchozí funkce
    s~tím rozdílem, že hledá i~samostatné geometrie (celé linie, polygony),
    nikoli jen liniové segmenty.
\end{itemize}

Další zásuvné moduly zařazené do~skupiny s~názvem \textit{Conflate} slouží
k~samotnému spojování vektorových map a~navázání dvou sousedních map.

\begin{itemize}
 \item \textit{Vertex Snapper} - identifikuje a~napojí k~sobě blízké uzlové
    body, vrcholy ze~dvou překrývajících se datasetů. Při~použití této funkce
    je nutné označit, která vrstva je referenční (s~body z~této vrstvy se 
    nebude hýbat).
 \item \textit{Coverage Alignment} - zarovná geometrii jednoho datasetu 
    k~jinému referenčnímu datasetu v~místech, kde se překrývají nebo spolu
    sousedí. Na~rozdíl od~předchozí funkce nepracuje pouze s~odpovídajícími
    si body, ale s~celými geometriemi.
 \item \textit{PolygonToolboxMatcherPlugin} - tento nástroj slouží k~identifikaci
    podobných polygonů ve~dvou různých datasetech, přičemž umožňuje různá 
    nastavení tak, aby bylo možné najít opravdu jen odpovídající si polygony
    popřípadě více polygonů odpovídajících jednomu či naopak. %% NEFUNGUJE MI - CHYBA U ME NEBO V PLUGINU?
 \item \textit{AlignmentToolboxPlugin} - slouží k~zarovnání dvou vrstev k~sobě.
    Bohužel zatím není plně funkční. %% NEFUNGUJE ???
\end{itemize}

Skupina označena jako \textit{Clean} obsahuje funkce k~k~opravě nepřesností
nalezených pomocí funkcí skupiny \textit{QA} v~rámci jednoho datasetu.

\begin{itemize}
 \item \textit{Remove Coverage Gaps} - odstraní mezery mezi~polygony jedné
    vrstvy dle~zadané tolerance.
 \item \textit{Remove Short Segments} - tato funkce by měla dokázat odstranit
    liniové segmenty kratší než daná tolerance tak, aby co nejméně porušila 
    topologii vrstvy. Zatím však umožňuje pouze odstranění krátkých izolovaných
    segmentů.
 \item \textit{CoverageCleaningToolboxPlugin} - poskytuje stejnou funkcionalitu
    jako první nástroj z~této skupiny, navíc umožňuje odhalit překryty 
    mezi~polygony jedné vrstvy.
\end{itemize}

Poslední skupina zásuvných modulů je nazvána \textit{Roads}. Zabývá se 
spojováním vektorových map silniční sítě.

\begin{itemize}
 \item \textit{RoadMatcherToolboxPlugin} - umožňuje vytvořit vrstvu s~rozdíly
    mezi~silnicemi ze~dvou vrstev a~na~základě těchto rozdílů a~identifikovaných
    společných prvků jednu z~těchto vrstev navázat na~druhou referenční tak, 
    aby si odpovídaly.
\end{itemize}


%\subsection{RoadMatcher Plugin}
%Zásuvný modul pro OpenJUMP ..

\subsection{OpenStreetMap}
\label{OSM}

OpenStreetMap je projekt sloužící k~tvorbě a~vizualizaci geografických dat.
Jedná se o~\textit{open source} projekt, což znamená, že ho může kdokoli 
využívat a~přispívat do něj. V~OpenStreetMap existuje mnoho nástrojů 
pro~editaci prostorových dat a~některé z~nich umožňují i~manuální nebo
poloautomatické spojování datasetů z~různých zdrojů (\textit{conflation}). 
Bohužel ani zde však neexistuje žádný komplexnější nástroj jako je výše
popsaný \textit{JCS}. Navíc většina těchto nástrojů je vytvořena pro~nějaký
konkrétní účel jako např. pro~spojování silničních sítí v~USA a~ne vždy
je proto jejich použití zcela obecné. Uživatel si tedy mnohdy musí poradit
sám a~použít několik různých nástrojů, aby dosáhl požadovaného výsledku.
Dále uvádím nástroje, které lze pro~některé činnosti související 
se~spojováním map použít. 

\subsubsection{JOSM conflation plugin}

\textit{Java OpenStreetMap Editor} (JOSM) je desktopová aplikace umožnující
editaci dat projektu OpenStreetMap. Jedním ze~zásuvných modulů pro~tuto 
aplikaci je \textit{Conflation}, který umožňuje spojování vektorových dat. 
Tento nástroj však je stále označen jako experimentální, což znamená, že ne 
vždy funguje zcela správně. Nástroj umožňuje zarovnat prvky jedné vrstvy tak,
aby souhlasily s~prvky druhé vrstvy, která je označena za~referenční. 

\label{josmpic}
  \begin{figure}[hbt]
    \centering
      \includegraphics[width=350pt]{./pictures/josm.png}
      \caption{JOSM conflation (\todo{zdroj})}
      \label{fig:josm}
  \end{figure} 

V~prvním kroku je určena referenční a~upravovaná vrstva. Poté je třeba v~obou
těchto vrstvách označit prvky, které by si měly odpovídat. V~dalším kroku se 
provede automatická identifikace dvojic vybraných prvků z~obou datasetů 
a~nakonec se provede samotný proces zarovnání prvků (\textit{conflate}), kdy 
jsou prvky upravované vrstvy změněny tak, aby odpovídaly prvkům vrtsvy 
referenční. Jedná se tedy o~proces poloautomatický, jelikož nejdříve je třeba
ručně identifikovat odpovídající si prvky a~na~základě tohoto kroku je 
provedena automatická úprava prvků.


\subsubsection{Potlatch 2 merging tool}

\textit{Potlatch 2 merging tool} je nástroj původně navržený pro~spojování 
vektorových dat cyklistických tras v~rámci England Cycling Data 
Project\footnote{Projekt pod záštitou britského ministerstva dopravy, 
který si klade za~cíl umožnit dostupnost informací o~síti cyklostezek
ve~Velké Británii prostřednictvím OpenStreetMap.}. %% tou záštitou si nejsem jistá
V~případě tohoto nástroje se jedná především o~proces přenosu atributů. 
Pokud máme v~mapě dva odpovídající si prvky, pomocí tohoto nástroje je 
můžeme označit za~odpovídající a~jednoduše sloučit jejich atributy. Nutno 
však podotknout, že výběr odpovídajících si prvků musíme provést vždy ručně. 


\subsection{Univerzitní projekty}
\label{univerzitní}

Mimo výše uvedené nástroje existuje také několik projektů různých světových
univerzit zabývajících se buď kombinací geo\-grafických dat obecně nebo 
spojováním map silničních sítí. Možnosti využití těchto projektů pro~běžného
uživatele se velmi různí. A~ne vždy lze zcela volně tyto projekty vyzkoušet. 
Jedním z~těchto projektů je i~Conflation System MBP, který byl vyvinut 
na~katedře počítačů (\textit{Computer Science Department}) na~Central Washington 
University v~USA. V~rámci tohoto projektu vznikl \textit{MBPConflate} 
software, který má přispět k~výzkumu v~oblasti slučování geo\-grafických dat. 
Program umožňuje automatické spojení map, poskytuje nástroje pro~následnou 
kontrolu kvali\-ty výsledné mapy. Je navržen tak, aby bylo snadné implementovat
nové techniky a~algoritmy v~této oblasti.  

\label{mbppic}
  \begin{figure}[hbt]
    \centering
      \includegraphics[width=350pt]{./pictures/MBPconflate.png}
      \caption{MBPConflate (\todo{zdroj})}
      \label{fig:mbp}
  \end{figure}

\chapter{Zatím nevím}
\label{4-nevim}

\section{GEOS - Geometry Engine, Open Source}

\textit{GEOS (Geometry Engine - Open Source)} je knihovna implementující 2D prostorové predikáty a funkce dle~OGC specifikace \textit{Simple Features} pro SQL. \textit{GEOS}
je přepisem knihovny \textit{Java Topology Suite (JTS)}, o~níž jsem se zmiňovala výše, do~jazyka C++. Knihovna je projektem OSGeo (The Open Source Geospatial Foundation) a je
dostupná pod licencí LGPL.

\section{OGR Simple Feature Library}

\textit{OGR Simple Feature Library} je \textit{open source} knihovna umožňující čtení a poř. i zápis vektorových dat různých formátů jako je ESRI Shapefile, S-57, SDTS, 
PostGIS, Oracle Spatial atd. \textit{OGR} je součástí knihovny \textit{GDAL}\footnote{Geospatial Data Abstraction Library - knihovna umožňující čtení a zápis rastrových dat}.
\chapter{Knihovna GEOC}
\label{5-geoc}

\section{Tvorba knihovny GEOC}
\label{knihovna}

Algoritmy týkající se slučování vektorových map (\textit{conflation}) byly
implementovány v~externí knihovně \textit{GEOC} bez závislosti na~QGIS API. 
Zásuvný modul \textit{Conflate} umožňuje využití funkcionality knihovny 
v~Quantum GIS. Díky tomuto oddělení vznikla nezávislá knihovna, kterou bude
možné případně použít i~v~jiných programech a~projektech.

Tato kapitola popisuje jednotlivé algoritmy a~jejich implementaci v~knihovně
\textit{GEOC}, zároveň také stručně pojednává o~možnostech  jejich využití.
Podrobný popis jednotlivých tříd a~jejich metod je podrobněji popsán v~dokumentaci. % viz příloha

% úvodní poznámky, proč samost. knihovna (aby se to dalo využít i jinde než v mém pluginu)
% co popisuje kapitola, vyjmenování algoritmů, jakou funkcionalitu by měla zajišťovat (stručně)

\section{Vertex Snapper} 
\label{vertexsnapper}

Při zpracování vrstev z~více zdrojů někdy stačí pouze upřesnit polohu či tvar 
prvků z~cílové vrstvy tak, aby se přiblížil prvkům z~vrstvy referenční. 
\mbox{Není-li} podrobnost obou datasetů příliš rozdílná, lze využít jednoduchého 
postupu \textbf{přichycení blízkých vrcholů}.

\subsection{Popis algoritmu}
\label{vs-algoritmus}

Nejjednodušším způsobem kombinace dvou vektorových vrstev je pouhé přichycení 
blízkých vrcholů cílové vrstvy k~vrstvě referenční. Obecný postup je následující:

\begin{enumerate}
 \item Na~počátku je třeba určit vzdálenostní toleranci, tedy maximální 
    vzdálenost mezi~dvěma body, kdy ještě bude provedeno jejich přichycení.
 \item Ke~každému prvku ze~zpracovávané vrstvy nalezneme nejbližší prvky 
    z~vrstvy referenční. To jsou prvky, jejichž nejkratší vzdálenost 
    od~zpracovávaného prvku není větší	než vzdálenostní tolerance.
 \item Pro~každý bod ze~zpracovávaného prvku vypočteme vzdálenosti ke~všem
    bodům z~blízkých referenčních prvků.
 \item Pokud nejmenší z~těchto délek je menší než vzdálenostní tolerance, 
    pak posuneme zpracovávaný bod do~odpovídajícího referenčního bodu s~touto
    nejmenší vzdáleností.
 \item Takto projdeme postupně všechny vrcholy všech prvků cílové vrstvy 
    a~snažíme se k~nim nalézt blízké body z~prvků vrstvy referenční. 
\end{enumerate}

% obrázek ilustrujici postup zpracovani
\label{vspic}
  \begin{figure}[hbt]
    \centering
      \includegraphics[width=350pt]{./pictures/vs-princip.pdf}
      \caption{Postup přichycení vrcholů (\todo{zdroj})}
      \label{fig:vs-princip}
  \end{figure}

\subsection{Implementace} % text asi přepsat 
\label{vs-implementace}
% popis mé implementace algoritmu + zmínit třídy a funkce s odkazy na literaturu
Algoritmus pro~přichycení vrcholů upravované vrstvy k~referenční je 
implementován ve~třídách \texttt{Vertex\-Snapper} 
a~\texttt{Vertex\-Geometry\-Editor\-Operation}. Při~použití v~externí aplikaci 
stačí po~předání vstupních parametrů třídě \texttt{Vertex\-Snapper} zavolat 
funkci \texttt{snap}. Ta vyhledá blízké prvky s~využitím prostorových 
indexů sestavených metodou \texttt{build\-Index} téže třídy.

Výsledky vyhledávání poté předá funkci \texttt{snap\-Vertices}. Uvnitř této
metody se vytvoří instance  třídy \texttt{Vertex\-Geometry\-Editor\-Operation},
která edituje příslušnou geometrii přichycením blízkých vrcholů.
\texttt{Vertex\-Geometry\-Editor\-Operation} je potomkem třídy 
\texttt{geos::\-operation::\-Coordinate\-Operation}, která je \textit{inter\-face}
\footnote{rozhraní třídy, kde jsou deklarovány pouze abstraktní metody bez
implementace, není možné vytvořit instanci této třídy} třídou pro editaci 
geometrie.


\subsection{Využití}
\label{vs-vyuziti}

Přichycení bodů jedné vrstvy k~vrstvě druhé má své výhody i~nevýhody, které 
je před volbou tohoto způsobu zpracování třeba zvážit. Použití této metody 
je vhodné v~takových případech, kdy máme k~dispozici dvě vrstvy o~rozdílné 
přesnosti (tento rozdíl však nesmí být příliš veliký) a~prvky vzájemně se 
překrývající. Cílem je upřesnit polohu a~tvar prvků z~méně přesného datasetu. 
Dopředu je třeba si uvědomit, že kromě polohy prvků je měněn i~jejich tvar.

Využít by tento postup šel i~pro~přichycení dvou sousedních vrstev o~stejné 
přesnosti, avšak to znamená, že by se změnil pouze tvar krajních prvků 
(nedošlo by k~posunu celé vrstvy), a~to nemusí být vždy žádoucí. Jako jediný
z~algoritmů \textit{GEOC} má pak smysl pro~bodové vrstvy.

Pro~rozumné výsledky je důležité zvolit vhodnou toleranční vzdálenost. Tato 
hodnota by měla odpovídat maximální vzdálenosti, o~kterou se vrchol prvku 
může posunout. Při~volbě příliš krátké vzdálenosti se výsledná vrstva nemusí 
vůbec odlišovat od té vstupní. Naopak \mbox{je-li} zvolená vzdálenost delší 
než nejkratší úsek geometrie (linie, polygonu), může dojít k~přichycení dvou 
bodů k~jednomu bodu z~referenční vrstvy. Zda je toto přípustné či nikoli je 
už na~rozhodnutí uživatele.

\label{vsinvalid}
  \begin{figure}[hbt]
    \centering
      \includegraphics[width=350pt]{./pictures/vs-invalid.pdf}
      \caption{Vznik nevalidní geometrie při přichycení vrcholů (\todo{zdroj})}
      \label{fig:vs-nevalidni}
  \end{figure} 

V~některých případech může dojít ke~vzniku nevalidních geometrií, to je takových,
jejichž segmenty se vzájemně protínají apod. To se nejčastěji stává u~protáhlých 
úzkých prvků a~jiných speciálních tvarů. Příkladem může být situace uvedená 
na~obrázku \ref{fig:vs-nevalidni}, kde růžový polygon je prvkem přichycen 
k~referenční fialové vrstvě, ale z~důvodu nedostatečné hustoty vrcholů a~nevhodného
tvaru vzniká nežádoucí křížení. 


%Někdy může být tento postup výhodný i~v~případě, kdy chceme odstranit drobné mezery či 
%překryty v~rámci jedné vrstvy. Poté stačí nastavit danou vrstvu jako referenční i~jako
%cílovou.  -- teď to ale nejde, protože se z mezer stanou překryty apod. (vrstva se v 
%průběhu nemění), ale šlo by to asi dodělat. 


\section{Coverage Alignment} 
\label{coverage alignment}

\textit{Coverage alignment} lze vysvětlit jako \textbf{zarovnání jedné vrstvy 
k~vrstvě druhé}. Tento způsob je složitější než výše uvedené přichytávání vrcholů.
V~knihovně \textit{GEOC} je využit opět pro~úpravu jedné vrstvy na~základě vrstvy 
referenční. Do~upravované vrstvy nejsou žádné prvky přidávány ani z~ní vymazávány,
pouze jde o~jejich modifikaci. Velmi podobný algoritmus se však dá použít 
i~ke~kombinaci dvou vrstev. 

\subsection{Popis algoritmu}
\label{ca-algoritmus}

Nejčastější používaný postup při~spojování vektorových map je následující.

\begin{enumerate}
 \item Nejprve je třeba nalézt odpovídající si prvky v~obou překrývajících se 
    vrstvách. Kritéria pro~určení odpovídajících si prvků mohou být velmi 
    odlišná. Existuje mnoho algoritmů řešících tuto problematiku, přičemž 
    postupy se mohou různit podle toho, zda je úkolem vyhledání 
    odpovídajících si bodů, polygonů či linií. Kritéria a~postup použitý 
    v~knihovně \textit{GEOC} je popsán níže.
 \item Poté, co se určí odpovídající si prvky, musí se určit totožné vrcholy 
    těchto dvojic prvků. Ty z~vrcholů, které jsou určeny s~dostatečnou 
    přesností (ta může být určena například danou vzdálenostní tolerancí), 
    jsou označeny jako body budoucí triangulační sítě.
 \item Jak už bylo naznačeno, z~nalezených bodů se vytvoří pomocí Delaunayho 
    triangulace\footnote{Delaunayho triangulace z~množiny bodů v~rovině vytvoří takovou 
    trojúhelníkovou síť, pro kterou platí, že v~kružnici opsané každému
    trojúhelníku, neleží žádný jiný bod. DT maximalizuje
    minimální úhly trojúhelníků.} trojúhelníková síť. 
    %ML: doplnit poznamku pod carou - kratka charakteristka DT .. doplneno
 \item Následně se provede lokální, nejčastěji afinní transformace v~každém 
    trojúhelníku sítě. Tak se přetransformují body cílové vrstvy do~systému 
    vrstvy referenční.
 \item Celý postup je možné iterativně opakovat, dokud nedosáhneme 
    požadovaného výsledku (ten může být dán např. podmínkou minimálního 
    množství nalezených odpovídajících si vrcholů či prvků).
\end{enumerate}

% obrázek ilustrujici postup zpracovani, vč. tinu apod.
\label{capic}
  \begin{figure}[hbt]
    \centering
      \includegraphics[width=400pt]{./pictures/ca-princip.pdf}
      \caption{Postup zarovnání vrstev (\todo{zdroj})}
      \label{fig:ca-princip}
  \end{figure}

V~knihovně \textit{GEOC} je pro nalezení odpovídajících si prvků využito 
obdobného postupu jako ve~výše zmiňované knihovně \textit{JCS}. Využívá 
se vrcholová Hausdorffova vzdálenost, přičemž tato vzdálenost není počítána 
přímo. Splnění podmínky, že dané prvky nejsou od~sebe dále, než je daná 
Hausdorffova vzdálenost, se testuje pomocí obalových zón jednotlivých prvků 
následujícím způsobem.

\begin{enumerate}
 \item Máme dva prvky A a~B ze~dvou různých překrývajících se vrstev.
 \item Pokud prvek B leží v~obalové zóně prvku A o~velikosti vzdálenostní 
    tolerance a~A leží v~obalové zóně prvku B o~stejné velikosti, je možné, 
    že si prvky odpovídají a~pokračuje se dalším krokem. V~opačném případě 
    si prvky neodpovídají.
 \item Dále se testuje, zda hranice prvku B leží v~obalové zóně hranice prvku
    A a~naopak. Je-li splněna i~tato podmínka, pak jsou prvky označeny 
    za~odpovídající si.
\end{enumerate}

%% ML: doplnit ilustraci .. obrázek obal. zón

\subsection{Implementace} % text asi přepsat 
\label{ca-implementace}
% popis mé implementace algoritmu + zmínit třídy a funkce s odkazy na literaturu
Veškeré zpracování je opět schováno pod jedinou funkcí \texttt{align} třídy
\texttt{Coverage\-Alignment}. Ta postupně volá metody provádějící jednotlivé
kroky algoritmu.

Nejdříve je třeba nalézt odpovídající si prvky. K~tomu slouží třída 
\texttt{Matching\-Geometry}, která k~dané geometrii najde odpovídající
(\texttt{set\-Match}). Obdobně jako u~\texttt{Vertex\-Snapper} je i~zde 
využito prostorových indexů, které jsou vytvořeny metodou 
\texttt{build\-Index}. Určení blízkých bodů je pak prostřednictvím
funkcí \texttt{choose\-Matching\-Points, \-find\-Closest\-Points,
\-clean\-Matching\-Points} a~dalšími třídy \texttt{Co\-ve\-ra\-ge\-Align\-ment}.

Třetím krokem je vytvoření TINu metodou \texttt{create\-TIN}, která k~tomu
využívá třídu \texttt{Tri\-an\-gu\-la\-tion}.  

Konečně je provedena postupně transformace všech prvků. Funkce pro
transformaci poskytuje třída \texttt{Affine\-Trans\-for\-mation},
která transformuje prvky na~základě předané geometrie a~triangulační
sítě. Ta je volána prostřednictvím třídy pro editaci 
\texttt{Align\-Geo\-metry\-Edi\-tor\-Ope\-ra\-tion}.


\subsection{Využití}
\label{ca-vyuziti}

Na~rozdíl od~předchozího algoritmu je tento trochu šířeji využitelný. Je opět 
vhodný pro~zpřesnění vrstvy dle vrstvy referenční, avšak tentokrát nejsou 
pouze přichytávány blízké vrcholy, ale jsou upravovány téměř všechny vrcholy. 
To zajišťuje reálnější výsledky i~v situacích, kdy hustota vrcholů v~obou 
datasetech je velmi rozdílná.

% poznámka o výhodách a nevýhodách využití u liniových a polygon. datech


% Vysázení seznamu obrázků
%\seznamobrazku

% Vysázení seznamu tabulek
%\seznamtabulek

% Začátek příloh
\prilohy

% Vysázení seznamu příloh
\seznampriloh

% Vložení souboru 'text/prilohy' s přílohami
%%%%%%%%%%%%%%%%%%%%%%%%%%%%%%%%%%%%%%%%%%%%%%%%%%%%%%%%%%%%%%%%%%%%%%%%%%%%%%%%%%%%
%%                 PŘÍLOHA - TESTOVÁNÍ                                           %%
%%%%%%%%%%%%%%%%%%%%%%%%%%%%%%%%%%%%%%%%%%%%%%%%%%%%%%%%%%%%%%%%%%%%%%%%%%%%%%%%%%%
\chapter{Testování rychlosti algoritmů}
\label{priloha-testovani}

Testování bylo provedeno na~přenosném počítači s~parametry uvedenými v~tabulce 
\ref{tab:parametry}.

\begin{table}[!h]
 \centering
\begin{tabular}{l|l}
 typ počítače & \textit{HP ProBook 6450b} \\
 paměť & \textit{4 GiB} \\
 procesor &\textit{Intel® Core™ i5 CPU M 450 @ 2.40GHz $\times$ 4 }\\
 operační systém &\textit{Ubuntu 12.04 LTS, 64 bit}\\
\end{tabular}
  \caption{ Parametry počítače využitého pro testování}
  \label{tab:parametry}
\end{table}

V~následujících tabulkách jsou uvedeny časy zpracování pro danou upravovanou
a~referenční vrstvu a~zvolenou toleranční vzdálenost. Časy jsou uváděny 
v~sekundách.

\begin{table}[!h]
\centering
 \begin{tabular}{|c|c|c|c|c|}
  \hline
     & \multicolumn{2}{c|}{vyber\_obce, vyber\_CR} & 
	\multicolumn{2}{c|}{zel1, zel2} \\
  \hline
   id  &  ~~100 m~ & 10 000 m & ~~~100 m & 10 000 m\\
  \hline
  \hline
  1  & 19.89 & 23.52 & 4.41 & 5.16 \\ 
2  & 19.83 & 23.00 & 4.41 & 4.44 \\
3  & 21.98 & 23.63 & 4.43 & 4.66 \\
4  & 22.09 & 21.40 & 4.89 & 5.18 \\
5  & 21.99 & 21.50 & 4.90 & 5.20 \\
6  & 19.89 & 22.59 & 4.89 & 4.67 \\
7  & 19.76 & 22.59 & 4.91 & 5.17 \\
8  & 19.77 & 22.21 & 4.89 & 5.15 \\
9  & 21.95 & 23.61 & 4.45 & 4.92 \\
10 & 22.00 & 21.97 & 4.48 & 4.68 \\
  \hline
  \hline
  průměr & 20.92& 22.60 &4.67 & 4.92 \\
  \hline
 \end{tabular}
  \caption{ Čas zpracování [s] bez prostor. indexů pro 
	    \texttt{Vertex\-Snapper}}
  \label{tab:vs-bez}
\end{table}

\begin{table}
\centering
 \begin{tabular}{|c|c|c|c|c|}
  \hline
     & \multicolumn{2}{c|}{vyber\_obce, vyber\_CR} & 
	\multicolumn{2}{c|}{zel1, zel2} \\
  \hline
   id  &  ~~100 m~ & 10 000 m & ~~~100 m & 10 000 m\\
  \hline
  \hline
  1  &0.78 & 0.69 &  0.45 & 0.47 \\
  2  &0.75 & 0.72 &  0.51 & 0.49 \\
  3  &0.69 & 0.76 &  0.48 & 0.50 \\
  4  &0.72 & 0.71 &  0.50 & 0.45 \\
  5  &0.77 & 0.78 &  0.48 & 0.48 \\
  6  &0.82 & 0.76 &  0.49 & 0.49 \\
  7  &0.90 & 0.70 &  0.43 & 0.50 \\
  8  &0.69 & 0.78 &  0.47 & 0.44 \\
  9  &0.68 & 0.69 &  0.43 & 0.51 \\
  10 &0.76 & 0.84 &  0.44 & 0.49 \\
  \hline
  \hline
  průměr & 0.76 & 0.74 &0.47 &0.48\\
  \hline
 \end{tabular}
  \caption{ Čas zpracování [s] s prostor. indexy pro 
	    \texttt{Vertex\-Snapper}}
  \label{tab:vs-s}
\end{table}

\begin{table}
\centering
 \begin{tabular}{|c|c|c|c|c|}
  \hline
     & \multicolumn{2}{c|}{vyber\_obce, vyber\_CR} & 
	\multicolumn{2}{c|}{zel1, zel2} \\
  \hline
   id  &  ~~100 m~ & ~1 000 m & ~1 000 m & 10 000 m\\
  \hline
  \hline
1  & 38.75 & 37.68 & 10.02 & 11.45 \\ 
2  & 38.55 & 37.93 & 10.36 & 8.99  \\
3  & 38.51 & 37.44 & 10.24 & 10.06 \\
4  & 41.60 & 36.95 & 11.22 & 10.41 \\
5  & 34.02 & 41.81 & 10.04 & 11.22 \\
6  & 40.31 & 38.55 & 10.44 & 10.02 \\
7  & 38.77 & 41.39 & 11.21 & 9.88 \\
8  & 38.54 & 36.79 & 10.09 & 11.30 \\
9  & 37.69 & 39.69 & 9.67  & 11.32 \\
10 & 36.72 & 41.60 & 11.67 & 10.29 \\
  \hline
  \hline
  průměr & 38.35 & 38.98 & 10.51 & 10.49\\
  \hline
 \end{tabular}
  \caption{ Čas zpracování [s] bez prostor. indexů pro 
	    \texttt{Coverage\-Alignment}}
  \label{tab:ca-bez}
\end{table}

\vspace{-200pt} % najít elegantnější řešení
\begin{table}
\centering
 \begin{tabular}{|c|c|c|c|c|}
  \hline
     & \multicolumn{2}{c|}{vyber\_obce, vyber\_CR} & 
	\multicolumn{2}{c|}{zel1, zel2} \\
  \hline
   id  &  ~~100 m~ & ~1 000 m & ~1 000 m & 10 000 m\\
  \hline
  \hline
1  &1.05 & 1.82 &  0.73 & 3.99 \\
2  &1.13 & 1.78 &  0.69 & 3.98 \\
3  &1.14 & 1.66 &  0.68 & 4.01 \\
4  &1.04 & 1.65 &  0.77 & 4.37 \\
5  &1.04 & 1.79 &  0.78 & 4.39 \\
6  &1.03 & 1.79 &  0.70 & 3.96 \\
7  &1.06 & 1.87 &  0.69 & 3.99 \\
8  &1.12 & 1.84 &  0.77 & 3.99 \\
9  &1.02 & 1.80 &  0.68 & 4.39 \\
10 &1.14 & 1.76 &  0.70 & 4.38 \\
  \hline
  \hline
  průměr & 1.08 &1.78 & 0.72 & 4.15 \\
  \hline
 \end{tabular}
  \caption{ Čas zpracování [s] s~prostor. indexy pro 
	    \texttt{Coverage\-Alignment}}
  \label{tab:ca-s}
\end{table} 

%%%%%%%%%%%%%%%%%%%%%%%%%%%%%%%%%%%%%%%%%%%%%%%%%%%%%%%%%%%%%%%%%%%%%%%%%%%%%%%%%%%
%%                 PŘÍLOHA - UŽIVATELSKÁ PŘÍRUČKA                                %%
%%%%%%%%%%%%%%%%%%%%%%%%%%%%%%%%%%%%%%%%%%%%%%%%%%%%%%%%%%%%%%%%%%%%%%%%%%%%%%%%%%%

\newpage
\chapter{Uživatelská příručka}
\label{priloha-prirucka}

Tato příručka je psána pro použití modulu \textit{Conflate} s~Quantum GIS 1.9.0 , 
v~jiných verzích se může způsob načtení modulu, popř. i~jiné činnosti mírně lišit.

\section{Instalace}
\label{prirucka-instalace}
Instalace ... %Je potřeba knihovna \textit{GEOC} ..

\section{Načtení zásuvného modulu}
\label{prirucka-nacteni}

Načtení zásuvného modulu lze provést ve~Správci zásuvných modulů.
\begin{center}
\textit{Zásuvné moduly~$\rightarrow$~Spravovat~zásuvné~moduly... 
(Plugins~$\rightarrow$~Plugin Manager...)}
\end{center}
Zde je třeba zaškrtnout \textit{Conflate Plugin}. Po~provedení tohoto kroku by se 
měla objevit ikonka modulu v~nástrojové liště a~také v~menu \textit{Zásuvné moduly}. 
Pokud se plugin nezobrazuje ve~Správci zásuvných modulů, je třeba nastavit cestu 
k~souboru~\textit{.so} v~\textit{Nastavení}.
\begin{center} 
\textit{Nastavení~$\rightarrow$~Volby~$\rightarrow$~Systém~$\rightarrow$~Cesty 
k~zásuvným modulům (Settings~$\rightarrow$~Options$\rightarrow$~System~$\rightarrow$
~Plugin paths)}
\end{center}


\section{Spuštění a nastavení dialogu}
\label{prirucka-spusteni}

\subsection{Výběr vstupních vrstev}
Před spuštěním dialogu \textit{Conflate} je třeba mít v~aktuálním projektu 
načteny vrstvy, které chceme zpracovávat. Pokud přidáme vrstvu až po~otevření
dialogu, lze ji načíst do~výběru vrstev tlačítkem \textit{Refresh}.
Po~otevření dialogu je třeba provést výběr \textbf{re\-ferenční vrstvy} 
(\textit{Reference Layer}) a~\textbf{upravované vrstvy} (\textit{Subject Layer}). 
Re\-ferenční vrstva je obvykla ta s~vyšší přesností, která se nebude měnit. 
Upravovanou vrstvu naopak chceme zarovnat k~vrstvě referenční. 
Obě vrstvy by měly být stejného geometrického typu (polygon~-~polygon, 
linie~-~linie, bod~-~bod).

\subsection{Metoda zpracování}
Dalším krokem je volba způsobu zpracování (\textit{Select the way of conflation}). 
Na~výběr jsou tyto metody:

\begin{itemize}
 \item \textbf{Přichycení vrcholů} (\textit{Snap Vertices}) - tato metoda vyhledá 
	blízké vrcholy z~obou datasetů a~změní polohu bodů upravované vrstvy tak, 
	aby odpovídala poloze blízkých bodů vrstvy referenční.
% obrázek, jak funguje ?
 \item \textbf{Zarovnání vrstev} (\textit{Coverage Alignment}) - princip této 
	metody je složitější. Na~rozdíl od~předchozí metody nepracuje s~jednotlivými
	body, ale s celými prvky. Upravuje i~některé prvky, k nimž neexistují 
	žádné odpovídající ve~vrstvě refe\-renční, a~to na~základě změny okolních 
	prvků. Je proto vhodnější zejména pro datasety, které mají rozdílný počet
	prvků. Obecně je možné s~touto metodou dosáhnout přesnějších a~často
	reálnějších výsledků, avšak na~úkor času zpraco\-vání.
% obrázek, jak funguje ?
 \item \textbf{Napasování linií} (\textit{Match lines} - tato metoda vyhledá
	odpovídající si úseky linií ze dvou různých vrstev. Takto nalezené
	páry zprůměruje a~vytvoří z~nich nové linie. Při volbě tohoto způsobu
	zpracování nezáleží na~tom, která vrstva je referenční a~která upravovaná.
	Napasování linií lze použít pouze na~dvojice liniových vrstev.
\end{itemize}

\subsection{Další nastavení}
Poté je třeba nastavit \textbf{toleranční vzdálenost} (\textit{Distance 
Tolerance}) v~jednotkách projektu. Tato vzdálenost udává, v~jaké maximální 
vzdálenosti mohou být odpovídající si prvky z~obou vrstev a~zároveň, jak 
moc se tedy může cílová vrstva měnit. V~ideálním případě by tato vzdálenost 
neměla přesahovat nejkratší segment geometrie všech prvků upravované vrstvy 
(tzn. nejkratší úsek linie, nejkratší stranu polygonu). Nevhodná volba této 
vzdálenosti může vést ke~vzniku nevalidních geometrií či nereálným výsledkům.

U~poslední metody (\textit{Match Lines}) se zviditelní ještě možnost nastavení
\textbf{kritéria podobnosti} \textit{Matching criterium}. Jedná se o~minimální 
podobnost dvou segmentů, která musí být dodržena pro jejich spárování. 
Tato hodnota se udává v~procentech. Při zadání $100 \%$ jsou výsledkem pouze 
naprosto totožné segmenty. Kritérium podobnosti se počítá na základě úhlu mezi 
segmenty, vzdálenosti jejich koncových bodů a~rozdílu jejich délek. 

Volba toleranční vzdálenosti a~metody zarovnání může výrazně ovlivnit rychlost 
zpracování, je proto doporučeno volit raději vždy menší vzdálenost a~jednodušší
metodu a~teprve po zobrazení výsledků případně parametry změnit.

Poslední nastavovanou možností je \textbf{automatická oprava geometrie}
 zaškrtnutím \textit{Try to repair invalid geometries}. Při této volbě se 
program pokusí opravit nově vzniklé nevalidní geometrie. Opravováno je pouze 
křížení úseků v~polygonu a~tzv. \uv{slepé větve}. Automatická oprava však 
může ovlivnit přesnost výsledku a~někdy i~vytvořit topologické chyby, proto 
je třeba ji používat opatrně.

Po~nastavení všech parametrů (včetně výstupní vrstvy - viz dále) se stisknutím 
tlačítka \textit{Process} spustí zpracování. 

\subsection{Výstupní vrstva}

Před spuštěním zpracování lze ještě nastavit název a~cestu, kam se má uložit
výsledný soubor. Upravená vrstva se vždy uloží jako nová vrstva ve~formátu 
\textit{shapefile}. Do~pole \textit{Output shapefile} lze zadat \textbf{cestu 
k~výstupnímu souboru}, popřípadě ji vybrat pomocí tlačítka \textit{Browse} 
(\textit{Procházet}) vpravo. 

Pokud zadáte pouze název výstupní vrstvy nebo toto pole necháte prázdné, 
uloží se nová vrstva do~aktuálního adresáře pod daným názvem nebo pod názvem
upravované vrstvy s~příslušným pořadovým číslem. 


\section{Výsledek}
\label{prirucka-vysledek}

Upravená vrstva se automaticky přidá do~otevřeného projektu v~QGISu.

V~dialogu zásuvného modulu se do~textového pole vlevo vypíše protokol o~zpracování.
Formát protokolu je popsán na~obrázku \ref{fig:protokol}.  

\label{protokol}
  \begin{figure}[H]
    \centering
      %\input{./pictures/protokol.pdf_tex}
      \includegraphics{./pictures/protokol.pdf}
      \caption{Formát protokolu}
      \label{fig:protokol}
  \end{figure} 

Výsledkem zpracování mohou být i~nevalidní geometrie, jejichž identifikátory lze
nalézt v~protokolu. Proto jsou často nezbytné ještě závěrečné úpravy s~využitím
editačních nástrojů QGISu.

% Konec dokumentu
\end{document}

% 
% \documentclass[11pt,twoside,a4paper]{report}
% \usepackage[czech, english]{babel}
% \usepackage[T1]{fontenc} % pouzije EC fonty
% \usepackage[utf8]{inputenc}
% 
% \usepackage{graphicx}
% \usepackage{wrapfig}        % pro obtékání textu kolem obrázk? a tabulek 
% 
% \newcommand\TypeOfWork{Bakalářská práce}  \typeout{Bakalarska prace}
% 
% \newcommand\StudProgram{Geodézie a Kartografie, strukturovaný, Bakalářský}
% \newcommand\StudBranch{Geoinformatika}
% 
% \newcommand\WorkTitle{Vector-to-vector conflation}
% \newcommand\FirstandFamilyName{Tereza Fiedlerová}
% \newcommand\Supervisor{Ing. Martin Landa}
% 
% \usepackage[
% pdftitle={\WorkTitle},
% pdfauthor={\FirstandFamilyName},
% bookmarks=true,
% colorlinks=true,
% breaklinks=true,
% urlcolor=red,
% citecolor=blue,
% linkcolor=blue,
% unicode=true,
% ]{hyperref}
% 
% \begin{document}
% 
% \selectlanguage{czech}
% 
% \iflanguage{czech}{
% 	 \typeout{************************************************}
% 	 \typeout{Zvoleny jazyk: cestina}
% 	 \typeout{Typ prace: \TypeOfWork}
% 	 \typeout{Studijni program: \StudProgram}
% 	 \typeout{Obor: \StudBranch}
% 	 \typeout{Jmeno: \FirstandFamilyName}
% 	 \typeout{Nazev prace: \WorkTitle}
% 	 \typeout{Vedouci prace: \Supervisor}
% 	 \typeout{***************************************************}
% 	 \newcommand\Department{Katedra mapování a kartografie}
% 	 \newcommand\Faculty{Fakulta stavební}
% 	 \newcommand\University{České vysoké učení technické v Praze}
% 	 \newcommand\labelSupervisor{Vedoucí práce}
% 	 \newcommand\labelStudProgram{Studijní program}
% 	 \newcommand\labelStudBranch{Obor}
% }
% 
% %%%%%%%%%%%%%%%%%%%%%%%%%%    Titulni stranka / Title page 
% 
% %\coverpagestarts
% 
% %\begin{titlepage}
% % F
% %\end{titlepage}
% 
% 
% %%%%%%%%%%%%%%%%%%%%%%%%%%%   Prohlaseni / Declaration 
% 
% %\declaration{V Doksích dne \today}
% 
% %%%%%%%%%%%%%%%%%%%%%%%%%%%%    Abstract 
% 
% %\abstractpage
% %\begin{abstract}
% %This bachelor's project deals with the problem of ...
% 
% %\vglue60mm
% 
% %\noindent{\Huge \textbf{Abstrakt}}
% %\vskip 2.75\baselineskip
% 
% %\noindent
% %Tato bakalářská práce se zabývá problémem ...
% %\end{abstract}
% 
% %%%%%%%%%%%%%%%%%%%%%%%%%%%%%%%%  Obsah / Table of Contents 
% 
% %\tableofcontents
% 
% 
% %%%%%%%%%%%%%%%%%%%%%%%%%%%%%%%  Seznam obrazku / List of Figures 
% 
% %\listoffigures
% 
% 
% %%%%%%%%%%%%%%%%%%%%%%%%%%%%%%%  Seznam tabulek / List of Tables
% 
% %\listoftables
% 
% 
% %**************************************************************
% 
% %\mainbodystarts
% 
% \normalfont
% %\parskip=0.2\baselineskip plus 0.2\baselineskip minus 0.1\baselineskip
% 
% %**************************************************************
% 
% 
% \cleardoublepage\chapter{Úvod}
\label{1-uvod}

Tato bakalářská práce se zabývá tématem slučování vektorových map
(\textit{vector - to - vector conflation}).

Spojování geografických dat z~různých zdrojů je jedním z~aktuálních
problémů v~oblasti geografických informačních systémů \zk{GIS}. 
S~rostoucí dostupností digitálních dat začala vzrůstat i~potřeba 
tato data kombinovat a~porovnávat. Na~internetu je volně k~dispozici 
spousta různých prostorových datasetů. Jejich přesnost, úplnost, 
atributy a~další vlastnosti jsou však velmi odlišné. Některé jsou 
geometricky přesné, ale neobsahují potřebné atributy nebo
některé žádoucí detaily. Méně přesné mapy s~potřebnými informacemi
lze upravit pomocí těch přesnějších, abychom získali požadovaný
výsledek. To a~veškeré podobné úlohy, které nás napadnou, jsou 
předmětem slučování map.

Pod anglickým pojmem \textit{conflation} se skrývá mnoho různých
procesů, úloh a~činností. Jejich cílem může být obohacení jedné 
mapy o~atributy či prvky z~mapy druhé, aktualizace datasetu, kombinace 
dvou map o~stejné přesnosti, detekce změn a jiné. Záměrem je vždy 
vytvoření nové mapy, z~níž lze vyčíst informace, které by nebylo možné
získat pouze z~jednoho vstupního zdroje.

Vzhledem k~velkému množství datasetů a~jejich rozsahu je automatizace
výše uvedených úloh logickým a~nevyhnutelným krokem. Ne vždy je však
automatické zpracování zárukou úspěchu. Každý algoritmus má své limity
a~nikdy nelze identifikovat $100\%$ odpovídajících si prvků. To může
být ještě sníženo nejednoznačností a~kvalitou vstupních dat. Proto
je stále třeba dodatečná ruční úprava a~kontrola dat. Ta je však
oproti úspoře času díky automatickému zpracování často zanedbatelná.

Výstupem práce má být zásuvný modul, který umožní spojení
vektorových map. Hlavním cílem je však vyhledání způsobů
řešení tohoto problému a~implementace některých z~nich.
Předpokládá se, že vznikne knihovna s~algoritmy pro slučování
vektorových dat, kterou bude možné dále rozšiřovat a~doplňovat
další funkcionalitu. Zásuvný modul bude využívat tuto knihovnu
a~s~jejím rozšiřováním bude možné i~sem přidávat další funkce.   

V~následující kapitole bude podrobněji popsána problematika 
slučování map (\textit{conflation}). V~kapitole \ref{3-nastroje}
pak budou představeny některé existující nástroje v~této oblasti.
Po~krátkém výčtu použitých technologií (kapitola \ref{4-technologie})
už se bude práce zabývat samotnou tvorbou knihovny a~zásuvného
modulu. Kapitola \ref{5-geoc} bude popisovat jednotlivé algoritmy
nově vznikající knihovny a~jejich implementaci. Následovat
bude popis zásuvného modulu a~jeho funkcionality včetně
ukázek jeho použití (kapitola \ref{6-plugin}). Poslední dvě kapitoly
budou věnovány řešení problémů a~nápady na~další vývoj 
a~vylepšení vzniklé aplikace.  
% \cleardoublepage\chapter{Conflation}
\label{2-conflation}

\section{Definice 'conflation'}
\label{definice}

\textit{Conflation} neboli \textit{map matching, map merging, rubber sheeting} lze do~češtiny přeložit jako slučování či spojování map. Překlad tohoto pojmu však není vždy 
jasný, jelikož se jím označuje více různých úloh, procesů či činností. Proto zde uvádím několik definic z~různýc zdrojů. 

\begin{itemize}[leftmargin=*]
 \item Proces, jehož cílem je geometricky upravit (posunem, transformací) digitální data zobrazující stejné území (obvykle pořízená v~jiném čase) tak, aby byla 
    vzájemně geograficky korektní a~vzájemně se překrývaly. %% slovem korektní si nejsem jistá 
    
    (huic.com)

  \item Proces sjednocení dvou rozdílných datasetů.
    
    (Blasby, Davis, Kim, Ramsey: GIS Conflation Using Open Source Tools)

  \item Zabývá se kombinováním dat z~různých zdrojů. Jedná se o~jeden z~aktuálních problémů v~GIS.
   
    (Freitas, Afonso: Distributed Vector Based Spatial Data Conflation Services)

  \item Soubor funkcí a~procedur, které zarovnávají prvky jedné GIS vrstvy k~prvkům jiné a~následně převádí atributy mezi~těmito vrstvami.
   
    (http://www.princegeorgescountymd.gov/Government/AgencyIndex/\newline OITC/GIS/glossary.asp)

  \item Proces sladění poloh odpovídajících si prvků v~různých datových vrstvách. Funkce pro spojení map (\textit{conflation}) provádějí toto sladění tak, aby se 
    odpovídající si prvky přesně překrývaly. 
    
    (http://maps.unomaha.edu/Peterson/GIS/notes/GISAnal1.html,\newline http://www.gov.ns.ca/snsmr/land/standards/post/manual/appedxa1.asp)

  \item Sada činností, které vzájemně přizpůsobují prvky dvou geografických datových vrstev a~následně převádí atributy jedné vrstvy do~druhé. %% možná spíše zarovnávají než přizpůsobují
    
    (http://wiki.gis.com/wiki/index.php/GIS\_Glossary/C)
  
  \item \textit{Feature conflation} je proces kombinování geografických informací z~překrývajících se zdrojových dat, který zachovává přesnost dat, minimalizuje nadbytečná 
    data a~předchází konfliktům v~datech. %% nebo rozdíly?
    \textit{Conflation of geospatial data} (geoprostorových dat) je spojení či sladění dvou různých prostorových datasetů zahrnujících stejné území.
    
    \citation{gisencyclopedia} (Shekhar, Xiong: Encyclopedia of GIS)

\end{itemize}

\section{Historie} % asi ještě rozšířit, viz enc. of gis - geospatial conflation
\label{historie}

Až do 80.~let 20.~století bylo pořízení digitálních dat velmi drahé a~proto se často nestávalo, že by nějaká firma vlastnila více digitálních map jediného území. S~vývojem
počítačů a~digitálních technologií se však tato data stávala stále dostupnějšími a~náhle vyvstala potřeba kombinovat data o~jednom území z~více zdrojů a~provádění aktualizace
těchto dat. 

Jako první se k~takovému množství dat dostaly pochopitelně různé vládní instituce. Ačkoli první článek o~problematice spojování geografických dat z~více zdrojů vyšel už v~roce
 1981 (\textit{M. White: The~Theory of~Geographical Data Conflation}), k opravdovému rozvoji došlo až po~roce 1985. V~té době totiž vznikl projekt, jehož cílem bylo propojení 
mapových souborů organizací US~Census Bureau a US Geological Society. Vzhledem k~množství dat bylo nutné celý proces co nejvíce automatizovat. Použitý algoritmus, jehož 
hlavním autorem je Alan Saalfeld, byl založen na~nalezení odpovídajících si prvků a~následné transformaci dat. 

S~narůstající dostupností dat a~zveřejněním této myšlenky přibývalo i~menších firem zabývajících
se problémem kombinace mapových souborů z různých zdrojů. Původní algoritmy brali v~potaz pouze geometrickou podobnost prvků, později byly brány v~úvahu
i~jejich topologické vztahy a~s~rozvojem GIS technologií pak i~podobnost atributů jednotlivých prvků. Dnes jde o~jeden z~aktuálních problémů řešených v~oblasti GIS.


\section{Související pojmy} % slovníček dále používaných pojmů atd.
\label{pojmy}

adjustment, alignment, attribute transfer (přenos, převod atributů), geometry (geometrie - jako geometrický prvek), feature (prvek), feature collection (kolekce prvků),
internal conflation (v rámci jednoho datasetu), matching (odpovídající si), reference dataset (referenční dataset), subject dataset (upravovaný dataset), topologie, ... 


\section{Klasifikace 'conflation'}
\label{klasifikace}

\subsection{Dle typu vstupních vrstev}
\label{dle-vstupu}

\begin{enumerate}[leftmargin=*]
  \item \textbf{Imagery-to-Imagery, Raster-to-Raster}
    \subitem Jedná se o~případ, kdy je úkolem na~sebe napasovat dva rastrové mapové soubory. Nejčastěji jde o~ortofoto a~naskenovanou analogovou mapu daného území. Tato úloha
	      se využije, například pokud chceme porovnat starou mapu se~současným stavem reprezentovaným právě leteckým snímkem. Řešení tohoto problému vyžaduje poměrně 
	      složité techniky pro~nalezení odpovídajících si objektů. Velmi důležitým prvkem je kvalita a rozlišení vstupních dat.
  \item \textbf{Vector-to-Imagery, Vector-to-Raster}
    \subitem Kombinace vektorových a~rastrových dat stejného území se využívá často ke~zpřesnění vektorových dat jejich napasováním na~ortofoto. Hlavní náplní této oblasti
	      je vývoj algoritmů umožňující následující: % dopsat referenci na lit. - gis enc. 
	      \begin{enumerate}[leftmargin=*]
	       \item Detekce charakteristických hran rastrového obrazu a~jejich porovnání s~vektorovými daty.
	       \item Využití vektorových dat k~identifikaci hran v~rastru - tzv. \textit{Snakes algorithm}.
	       \item Užití stereo obrazu, výškových dat a~znalostí silničních dat pro~porovnání vektorových a~rastrových dat. % ??????????
	       \item Využití prostorových informací stejně jako dalších vlastností mapy (jako např. barev) k~rozpoznání odpovídajících si prvků.
	      \end{enumerate}
  \item \textbf{Vector-to-Vector}
    \subitem Případem sloučení dvou vektorových datasetů se zabývá tato práce. Jednou ze~specifických a~velmi častých aplikací je navázání silničních sítí, dále aktualizace
	      digitálních dat aj. Existuje mnoho různých algoritmů pro~slučování vektorových dat, přičemž základní přístupy k~řešení problému jsou následující: % dopsat referenci na lit. - gis enc. 
	      \begin{enumerate}[leftmargin=*]
	       \item Sloučení dat za~pomoci algoritmů, které pracují na~základě porovnávání geometrických vlastností prvků.
	       \item Algoritmy, které berou v~potaz podobnost tvarů prvků a~zároveň podobnost jejich atributů.
	       \item Spojení vektorových dat s~neznámým souřadnicovým systémem na~základě rozložení významných bodových prvků (např. křižovatky cest).
	      \end{enumerate}

\end{enumerate}


\subsection{Dle území zobrazovaného vstupními vrstvami}
\label{dle-uzemi}

\begin{enumerate}[leftmargin=*]
  \item \textbf{Horizontální} % - sousedící datasety
    \subitem Za~\textit{horizontal conflation} se označují procesy, které zpracovávají data ze~sousedících území. Cílem je získat mapové soubory, jejichž hranice na~sebe
	      dokonale navazují, a~to pokud možno bez~ztráty přesnosti.
  \item \textbf{Vertikální} % - překrývající se datasety
    \subitem Při~\textit{vertical conflation} obsahují vstupní data překrývající se území. Jde tedy o~dva či více souborů zobrazujících jediné území, třeba i~jen částečně.
	      Může se jednat o~dvě verze té samé mapy nebo o~dva datasety s~nějakými společnými prvky a~vlastnostmi.
	      Výsledkem celého procesu je jediný dataset, jehož přesnost není horší než přesnost původních dat a~obsahuje informace z~obou zdrojů, jde tedy o~vylepšenou, 
	      obsahově bohatší mapu s~odpovídající přesností. 
\end{enumerate}


\section{Obecný postup}
\label{postup}

Obecný postup při~slučování vektorových map z~více zdrojů se skládá z~několika kroků, které jsou uvedeny níže. Jako u~většiny podobných operací je třeba nejdříve provést 
přípravu dat a~následně data zpracovat a~upravit.

\begin{enumerate}[leftmargin=*]
  \item \textbf{Předzpracování dat}
    \subitem Tento krok slouží k~zajištění kompatibility vstupních dat tak, aby bylo možné je porovnávat. Obvykle spočívá v~převedení vstupních datasetů do~stejného 
	      souřadnicového systému, zajištění stejných základních jednotek a~dalších měnších úpravách. 
  \item \textbf{Kontrola kvality dat a~topologické správnosti vrstev}
    \subitem Zde se kontroluje vnitřní konzistence dat každé vstupní vrstvy. V~tomto případě záleží především na~požadavcích zvoleného algoritmu pro~sloučení map. Může jít
	      například o~odstranění topologických chyb v~dané vrstvě jako jsou nežádoucí drobné překryty či mezery mezi~polygony.
  \item \textbf{Vyhledání odpovídajících si prvků}
      \subitem Následuje vyhledání prvků, které si v~upravovaných datasetech odpovídají, tedy zobrazují stejný předmět ve~skutečnosti. Tento krok je nezbytný proto, aby bylo
		možné rozhodnout, jak na~sebe vstupní vrstvy navazují.
  \item \textbf{Sloučení geometrických prvků a/nebo atributů}
      \subitem Po~rozpoznání odpovídajících si prvků je už možné upravit geometrii či atributy prvku z~jedné vrstvy s~přihlédnutím k~vlastnostem prvku z~jiné vrstvy. 
		Při~procesu slučování různých datasetů lze převádět pouze atributové hodnoty mezi~odpovídajícími si prvky nebo pouze změnit geometrii prvku tak, aby odpovídala
		geometrii prvku z~jiné vrstvy, která bývá označena za~referenční. Ve~složitějších úlohách už je možné převádět zároveň atributy i~geometrii, a~to nejen 
		jednosměrně (tedy z~vrstvy referenční do~vrstvy upravované), ale výsledkem může být vrstva, jejíž geometrie je kombinací geometrických vlastností obou 
		vstupních vrstev.
  \item \textbf{Závěrečné úpravy}
      \subitem Po~provedení automatického a~někdy i~manuálního sloučení datových vrstev je vhodné provést kontrolu výsledku. Často jsou potřeba ještě drobné úpravy, aby výsledná
		data odpovídala počátečním požadavkům. Ne vždy jsou totiž při~automatickém procesu určeny správně všechny odpovídající si prvky a~některé algoritmy mohou
		při~úpravě geometrických vlastností narušit topologickou správnost dat.
\end{enumerate}


\section{Využití - hlavní aplikace}
\label{využití}

\subsection{oblasti, obory}
\label{obory}
kartografie, GIS, výpočetní geometrie, letecká fotogrammetrie, vojenský výcvik a výzkum, krizový management, dopravní mapy - aktualizace dat, trh s nemovitostmi, ...

\subsection{aplikace}
\label{aplikace}
zpřesnění map, aktualizace prostorových dat, georeferencování - záznam dat, detekce chyb, rozdílů, ...


% \cleardoublepage\chapter{Existující nástroje}
\label{3-nastroje}

Tato kapitola se zabývá již existujícími nástroji pro~spojování vektorových map
 z~různých zdrojů (\textit{conflation}).

\section{Proprietární nástroje}
\label{proprietární}

Proprietárních nástrojů umožňujících řešení spojování vektorových map 
existuje celá řada. Některé programy jsou orientovány přímo na~tento problém,
jiné nabízejí funkce pro~slučování map pouze jako vedlejší funkcionalitu 
a~ne vždy je proto možné pomocí nich řešit složitější problémy. Následující
výčet neobsahuje všechen komerční software zabývající se slučováním
vektorových map, ale nejvýznamnější nástroje, které lze pro~zpracování 
použít. Vzhledem k~tomu, že se jedná o~komerční software, nebylo možné 
všechny níže popsané nástroje otestovat, proto jejich popis vychází 
především z~informací dostupných na~oficiálních internetových stránkách 
produktů.


\subsection{ESRI ArcGIS}
\label{arcgis}

V~softwaru ArcGIS 10 existuje několik nástrojů, které lze využít pro~spojování
geo\-grafických dat, přenos atributů a~odstraňování geometrických rozdílů 
mezi~datasety. 

\subsubsection{Spatial Adjustment}

Soubor nástrojů \textit{Spatial Adjustment} systému ArcGIS poskytuje 
základní funkce týkající se této problematiky. Má sloužit především 
k~úpravě dat z~různých zdrojů a~zajistit tak jejich celistvost. Může 
být použito několik metod pro~zarovnání jedné vrstvy či její části 
ke~druhé. Lze provést transformaci, navázání hran (\textit{edge matching})
nebo srovnání překrývajících se dat (\textit{rubber sheeting}).

Pro použití metody transformace je nejdříve třeba označit data, která budou
do~procesu vstupovat. Poté se vyberou dvojice uzlových bodů, které by si měly
odpovídat. Na~výběr je několik typů transformací. Všechny jsou prováděny 
na~základě vybraných dvojic identických bodů. 
%%Tento postup bychom tedy mohly označit za~ruční 'conflation'.

Do~procesu horizontálního zarovnání (\textit{horizontal conflation}) lze 
zařadit funkce nástroje \textit{Edge Match Tool}, který umožňuje navázat 
na~sebe dvě sousedící vrstvy. Zarovnání je poměrně snadné, stačí pouze 
zvolit toleranci (maximální vzdálenost pro navázání prvků) a~označit tímto
nástrojem hranici mezi vrstvami, kde by měly na~sebe navazovat. Ve~vybrané
oblasti se zobrazí indikátory naznačující způsob zarovnání, které lze ještě
ručně upravit. Po~potvrzení se provede zarovnání tak, aby byla zachována
topologie.

Dalším způsobem geometrického spojení vrstev, který \textit{Spatial Adjustment}
nabízí, je \textit{Rubber Sheeting}. V~tomto případě je třeba označit opět 
odpovídající si body a~navíc body, jejichž poloha by se neměla změnit. 
Při~spuštění zarovnání se dočasně vytvoří triangulační síť mezi označenými body
(vzájemně si odpovídající body). Poloha ostatních neoznačených bodů je pak 
vypočtena interpolací v~této síti.

Poslední z~nástrojů pro~kombinaci map je \textit{Attribute Transfer}. Pomocí 
něho je možné převést atributy mezi odpovídajícími si prvky, ale také upravit
jejich geometrii. Po~volbě jednoho či více atributů, které se mají převádět 
mezi~zdrojovou a~cílovou vrstvou lze interaktivně vybírat odpovídající si 
dvojice prvků v~obou vrstvách. Mezi~těmito prvky se převedou atributy a~pokud
je zaškrtnuta volba \textit{Transfer Geometry} neboli \uv{převést geometrii},
změní se geometrie cílového prvku dle~zdrojového. Převod lze provést také 
najednou pro~více vybraných prvků.
%% http://help.arcgis.com/en/arcgisdesktop/10.0/help/index.html#/About_spatial_adjustment_attribute_transfer/001t000000v9000000/
%% http://help.arcgis.com/en/arcgisdesktop/10.0/help/index.html#/About_spatial_adjustment_rubbersheeting/001t000000v3000000/

\subsubsection{Integrate}

Nástroj \textit{Integrate} umožňuje sladění dvou datasetů. Na~vstupu vyžaduje 
dvě či více vrstev, které chceme sladit, a~dále maximální vzdálenost, při~které
lze považovat prvky za~odpovídající si. U~každého datasetu navíc lze zadat 
prioritu (\textit{rank}). Prvky s~nižší prioritou se pak budou zarovnávat 
k~těm s~prioritou vyšší. Tato funkce je velmi užitečná, pokud máme dvě 
překrývající se nebo sousedící vrstvy s~malými rozdíly např. v~důsledku různé
přesnosti.
%% http://help.arcgis.com/en/arcgisdesktop/10.0/help/index.html#//00170000002s000000

\subsection{ConfleX}
\label{conflex}

ConfleX je software pro~automatické spojování vektorových GIS dat,
který pro~automatizaci využívá umělou inteligenci. ConfleX umožňuje zpracování 
i~takových případů, kdy se zdrojová mapa s~cílovou nepřekrývají nebo nejsou 
topologicky identické. Systém porovnává každé dva segmenty z~obou map a~jejich
vztah k~ostatním segmentům, na~základě tohoto postupu pak rozhodne, zda se 
jedná o~stejné prvky či nikoli. Kromě automatického procesu umožňuje program 
i~následnou ruční editaci.

ConfleX je k~dispozici jako samostatná aplikace ale také jako extenze 
programu ArcGIS 9/10.

\subsection{Intergraph GeoMedia Fusion}
\label{geomedia}

Nástroj GeoMedia Fusion firmy Intergraph je navržen pro~úpravu dat, aby byly
topologicky korektní, validaci atributů a~integraci dat. Cílem je umožnit 
snadnou údržbu dat v~rozsáhlých geografických databázích, kde jsou data 
získávána z~různých zdrojů. Nástroj porovnává dva datasety obsahující 
rozdílné reprezentace stejné skutečnosti. Nejdříve automaticky vytvoří 
\textit{conflation links} indikující způsob spojení vrstev, které lze ještě
ručně editovat. Následně umožní geometrie i~atributy těchto dvou reprezentací
sjednotit. GeoMedia Fusion slouží k~úpravě bodových, liniových i~plošných dat
včetně jejich atributů.


\subsection{MapMerger}
\label{mapmerger}

MapMerger je GIS nástroj firmy ESEA zaměřený na~slučování geometrie a~atributů
vektorových map a~kontrolu kvality dat. Umožňuje převod atributů mezi dvěma 
překrývajícími se mapami, navázání hranic dvou sousedících map, přidání prvků 
z~jedné mapy do~druhé, synchronizaci mapy s~její aktualizovanou verzí, 
identifikaci geometrických a~atributových rozdílů mezi~dvěma verzemi té samé 
mapy. 

\section{Open Source nástroje}
\label{open-source}

V~této sféře zatím neexistuje mnoho nástrojů, které by komplexně řešili problém
spojování vektorových map (\textit{conflation}). Většinou jde pouze o~malé 
programy či zásuvné moduly k~větším projektům, pomocí nichž se dá provést manuální
nebo poloautomatické sloučení vektorových datasetů, případně jejich atributů. 
Ovšem většinou je cesta k~dosažení cíle pomocí těchto nástrojů poměrně složitá 
a~ne vždy jsou výsledky takové, jak bychom si představovali. Asi jediným 
ucelenějším nástrojem je knihovna \textit{JCS} implementovaná jako kolekce 
zásuvných modulů v~programu OpenJUMP.

\subsection{JCS - Java Conflation Suite}
\label{JCS}

\textit{JCS} je \textit{open source} knihovna napsaná v~jazyce Java, která 
zahrnuje API a~soubor interaktivních nástrojů, které slouží ke~slučování 
prostorových datasetů. Byla vyvinuta společností Vivid Solutions, Inc. 
Obsahuje funkce umožňující provádění různých procesů spojených se~spojování
vektorových map, které jsou zaměřeny především na~polygonové případně liniové
datasety, co se týče bodových vrstev, je její funkcionalita poměrně omezená. 
Knihovna \textit{JCS} je závislá na~knihovně \textit{JTS - Java Topology Suite},
která poskytuje základní geometrické funkce pro~práci s~prostorovými
daty. Obě knihovny jsou navrženy v~souladu s~OGC specifikací 
\textit{Simple Features} \footnote{Specifikace týkající se 2D prostorových 
predikátů a~funkcí}. \textit{JCS} vznikla v~rámci projektu JUMP Unified 
Mapping Platform. Je poskytována pod~licencí LGPL.

\subsubsection{Architektura}
\label{jcspic}
  \begin{figure}[hbt]
    \centering
      \includegraphics[width=250pt]{./pictures/JCS_Architecture.png}
      \caption{Architektura JCS 
	  (\todo{zdroj})}
      \label{fig:architektura}
  \end{figure}


Knihovna \textit{JCS} používá pro~vizualizaci dat a~interakci JUMP WorkBench
a~API. Pro~poskytování základní geometrické funkcionality je pak využíváno 
knihovny \textit{JTS}. JUMP API umožňuje vstup a~výstup prostorových dat 
a~další funkcionalitu s~nimi spojenou. Jádro \textit{JCS} tvoří Conflation API
obsahující algoritmy pro~kontrolu a~slučování prostorových dat 
(\textit{conflation}). Funkce \textit{JCS} jsou v~projektu OpenJUMP 
implementovány v~podobě kolekce zásuvných modulů - \textit{QA, Conflate, 
RoadMatcher}.

\subsubsection{OpenJUMP projekt}

OpenJUMP je \textit{open source} projekt, který vyvinula firma Vivid Solutions,
Inc. Jedná se o~GIS software, který umožňuje základní práci 
s~prostorovými daty v~rastrové či vektorové podobě a~jejich atributy.

\subsubsection{JTS - Java Topology Suite}
\label{kap:jts}

Knihovna \textit{JTS} je jedním ze~základních prvků projektu OpenJUMP.
Je napsána v~jazyce Java a~poskytována pod~licencí LGPL. Obsahuje třídy 
pro~reprezentaci geometrických objektů a~základní funkce pro~práci 
s~prostorovými daty dle~specifikace \textit{Simple Features} pro~SQL 
od~OGC. Kromě tříd reprezentujících geometrické prvky dle~zmíněné
specifikace zahrnuje další podpůrné třídy pro~reprezentaci seznamu souřadnic,
aplikaci geometrického filtru (např. při~transformaci), uchovávání informace
o~maximální a~minimální souřadnici objektu a~jiné. Dále jsou součástí této 
knihovny třídy umožňující geometrické výpočty jako je vzájemná polohu bodu 
a~linie, výpočet průsečíku, prostorové analýzy, test polohy bodu a~uzavřené
oblasti atd. 

\subsubsection{Funkcionalita JCS} %% JINÝ NÁZEV?

Projekt \textit{JCS} neumožňuje změnu souřadnicového systému. Proto je
automaticky přepokládáno, že vrstvy vstupující do~zpracování mají stejný
prostorový referenční systém. Souřadnice bodů výstupních vrstev mají vždy
desetinnou přesnost.  %% JE TO SPRÁVNĚ ??

Při většině výpočtů v~zásuvných modulech \textit{JCS} je využívána tzv. 
Hausdorffova metrika. Na~rozdíl od~euklidovské metriky totiž nezkoumá jen
nejkratší vzdálenost mezi prvky, ale i~vzdálenost největší.
Zohledňuje tedy do~jisté míry i~topologické vztahy.  %% ML: Kostrbate... opraveno

Na~výpočet je Hausdorffova vzdálenost poměrně složitá. Proto se v~algoritmech
použitých v~\textit{JCS} používá spíše vrcholová Hausdorffova vzdálenost 
(\textit{Vertex Hausdorff Distance}), která není vztažena ke~geometrickému 
prvku, ale pouze k~jeho vrcholům. Tato varianta Hausdorffovy vzdálenosti 
ve~většině případů vrací stejně dobré výsledky. 

Postup spojování vektorových datasetů pomocí knihovny \textit{JCS} je založen
na~nalezení geometrických rozdílů mezi~oběma mapami a~následném odstranění 
těchto rozdílů.

K~detekci geometrických rozdílů je použit algoritmus pro~prostorové rozdíly.
Tento algoritmus funguje tak, že postupně porovnává geometrické prvky z~obou 
datasetů popřípadě pouze jejich jednotlivé části a~pokud se tyto prvky shodují,
označí je jako odpovídající si. Výsledkem jsou pak ty prvky z~obou datasetů, 
ke~kterým nebyly nalezeny žádné odpovídající prvky. Nalezení odpovídajících si
prvků probíhá následovně. Pokud je požadována přesná shoda, provede se 
testování, zda jsou prvky stejného geometrického typu a~zda se rovná jejich 
obsah. Porovnání obsahu se zakládá na~porovnávání jednotlivých komponent 
a~seznamu bodů daných geometrií. Ne vždy však je předpokládána přesná shoda,
proto je možné určit i~prvky, které splňují podmínku danou tolerancí. Zde se
provádí porovnání Hausdorffovy vzdálenosti mezi~prvky s~touto tolerancí, 
nepočítá se přitom přímo tato vzdálenost. Rozhodující je, jestliže obalová 
zóna o~velikosti vzdálenostní tolerance prvního prvku obsahuje prvek druhý 
a~naopak. Tato podmínka je ekvivalentní k~podmínce, že Hausdorffova vzdálenost
musí být menší nebo rovna toleranci. 

%% ML: doplnit ilustraci? .. stejná jako u popisu meho algoritmu CoverageAlignment

Pro~odstraňování překrytů či mezer neexistuje exaktní algoritmus,
ale jsou využívány různé heuristiky poskytující dobré topologické výsledky.

Následující výčet podává přehled nejčastějších úloh, které je možné 
s~tímto nástrojem řešit.

\begin{itemize}
 \item Jako \textit{Coverage Cleaning} je  označován proces hledání
    a~odstraňování topologických chyb - mezer a~překrytů v~rámci jedné
    vektorové mapy tvořené polygony či multipolygony. Detekce nežádoucích
    mezer mezi~polygony je založena na~rozpoznání blízkých liniových segmentů,
    kde blízkost je určena zvolenou vzdálenostní tolerancí. 

 \item Další často řešenou úlohou je \textit{Boundary Alignment}, což by
    bylo možné přeložit do češtiny jako \uv{zarovnání či navázání hranic}. 
    Cílem je napojit k~sobě dvě vektorové vrstvy, které zobrazují sousedící
    území, tak, aby se mezi~nimi nevyskytovaly nežádoucí mezery či překryty.
    Výsledkem jsou tedy plynule navazující datasety, které tvoří bezešvou mapu.
    Při této úloze je nutné zvolit přesnější referenční vrstvu, jejíž 
    geometrické vlastnosti se nezmění.

 \item U \textit{Coverage Alignment} máme na~vstupu naopak dvě vektorové
    mapy zobrazující to samé území nebo alespoň jeho část. Tyto mapy se 
    tedy výrazně překrývají. Úkolem je upravit méně přesnou vrstvu tak,
    aby odpovídala vrstvě referenční popřípadě její části, pokud se vrstvy
    úplně nepřekrývají. Proces spočívá v~posunutí vrcholů polygonů upravované
    vrstvy do~blízkých vrcholů vrstvy referenční.

 \item Poměrně specifickou, ale velmi častou úlohou je \textit{Road Network 
    Matching} neboli \uv{spojování silničních sítí}. Na~vstupu máme dvě 
    vektorové mapy té samé silniční sítě. Při~této úloze hledáme podobnost
    mezi~liniovými prvky obou datasetů, které označíme za~odpovídající si.
    Poté je vytvořena nová vrstva silniční sítě, která obsahuje odpovídající
    si prvky, přičemž při~mírných odlišnostech použije liniové prvky z~přesnější
    mapy. \textit{JCS} bohužel zatím neumožňuje automatické provedení této
    úlohy. Pouze nalezne odpovídající si prvky a~další kroky už je nutné provést
    manuálně.

\item Úloha \textit{Geometry Difference Detection}, v~češtině \uv{detekce
    geometrických rozdílů}, na~rozdíl od~předchozích nijak neupravuje vstupní
    vrstvy ani z~nich netvoří jiné. Cílem je pouze nalézt rozdíly mezi vstupními
    datasety. Nejčastěji je používána pro~rozpoznání změn mezi~dvěma verzemi
    jedné vektorové mapy (např. po aktualizaci).
\end{itemize}

\subsubsection{Popis zásuvných modulů}

Zásuvné moduly ze skupiny \textit{QA - Quality Assurance} umožňují najít
geometrické rozdíly a~nesrovnalosti mezi~datasety, ale také v~rámci jediného
datasetu. Funkce zde obsažené neslouží k~opravě či propojení vrstev, ale 
pouze k~identifikaci geometrických rozdílů.

\begin{itemize}
 \item \textit{Find Misaligned Segments} - slouží k~nalezení segmentů 
    ze~dvou datasetů, které by si v~rámci dané tolerance měli odpovídat,
    ale je mezi~nimi mezera či překryt. 
 \item \textit{Find Overlaps} - najde překrývající se prvky ze~dvou datasetů.
 \item \textit{Find Coverage Gaps} - umožňuje nalézt mezery mezi~polygony
    jednoho datasetu, které jsou užší než zadaná vzdálenostní tolerance
    a~zároveň je mezi~hranami polygonů, které tvoří tuto mezeru, úhel menší
    než daná úhlová tolerance.
 \item \textit{Find Coverage Overlaps} - najde všechny překryty mezi~polygony
    v~rámci jednoho datasetu, respektive všechny překrývající se polygony.
 \item \textit{Find Close Vertices} - identifikuje body (samostatné body,
    vrcholy linií či polygonů) ze~dvou různých datasetů, jejichž vzdálenost
    je menší než daná tolerance.
 \item \textit{Find Offset Boundary Corners} - slouží k~nalezení hranic
    polygonů ze~dvou sousedících vektorových map, které by na~sebe měly
    navazovat, ale je mezi~nimi posun menší než zadaná tolerance.
 \item \textit{DiffSegmentsPlugin} - identifikuje liniové segmenty, které
    jsou obsaženy pouze v~jedné ze~zadaných vrstev, nikoli v~obou dvou.
 \item \textit{DiffGeometryPlugin} - funguje stejně jako předchozí funkce
    s~tím rozdílem, že hledá i~samostatné geometrie (celé linie, polygony),
    nikoli jen liniové segmenty.
\end{itemize}

Další zásuvné moduly zařazené do~skupiny s~názvem \textit{Conflate} slouží
k~samotnému spojování vektorových map a~navázání dvou sousedních map.

\begin{itemize}
 \item \textit{Vertex Snapper} - identifikuje a~napojí k~sobě blízké uzlové
    body, vrcholy ze~dvou překrývajících se datasetů. Při~použití této funkce
    je nutné označit, která vrstva je referenční (s~body z~této vrstvy se 
    nebude hýbat).
 \item \textit{Coverage Alignment} - zarovná geometrii jednoho datasetu 
    k~jinému referenčnímu datasetu v~místech, kde se překrývají nebo spolu
    sousedí. Na~rozdíl od~předchozí funkce nepracuje pouze s~odpovídajícími
    si body, ale s~celými geometriemi.
 \item \textit{PolygonToolboxMatcherPlugin} - tento nástroj slouží k~identifikaci
    podobných polygonů ve~dvou různých datasetech, přičemž umožňuje různá 
    nastavení tak, aby bylo možné najít opravdu jen odpovídající si polygony
    popřípadě více polygonů odpovídajících jednomu či naopak. %% NEFUNGUJE MI - CHYBA U ME NEBO V PLUGINU?
 \item \textit{AlignmentToolboxPlugin} - slouží k~zarovnání dvou vrstev k~sobě.
    Bohužel zatím není plně funkční. %% NEFUNGUJE ???
\end{itemize}

Skupina označena jako \textit{Clean} obsahuje funkce k~k~opravě nepřesností
nalezených pomocí funkcí skupiny \textit{QA} v~rámci jednoho datasetu.

\begin{itemize}
 \item \textit{Remove Coverage Gaps} - odstraní mezery mezi~polygony jedné
    vrstvy dle~zadané tolerance.
 \item \textit{Remove Short Segments} - tato funkce by měla dokázat odstranit
    liniové segmenty kratší než daná tolerance tak, aby co nejméně porušila 
    topologii vrstvy. Zatím však umožňuje pouze odstranění krátkých izolovaných
    segmentů.
 \item \textit{CoverageCleaningToolboxPlugin} - poskytuje stejnou funkcionalitu
    jako první nástroj z~této skupiny, navíc umožňuje odhalit překryty 
    mezi~polygony jedné vrstvy.
\end{itemize}

Poslední skupina zásuvných modulů je nazvána \textit{Roads}. Zabývá se 
spojováním vektorových map silniční sítě.

\begin{itemize}
 \item \textit{RoadMatcherToolboxPlugin} - umožňuje vytvořit vrstvu s~rozdíly
    mezi~silnicemi ze~dvou vrstev a~na~základě těchto rozdílů a~identifikovaných
    společných prvků jednu z~těchto vrstev navázat na~druhou referenční tak, 
    aby si odpovídaly.
\end{itemize}


%\subsection{RoadMatcher Plugin}
%Zásuvný modul pro OpenJUMP ..

\subsection{OpenStreetMap}
\label{OSM}

OpenStreetMap je projekt sloužící k~tvorbě a~vizualizaci geografických dat.
Jedná se o~\textit{open source} projekt, což znamená, že ho může kdokoli 
využívat a~přispívat do něj. V~OpenStreetMap existuje mnoho nástrojů 
pro~editaci prostorových dat a~některé z~nich umožňují i~manuální nebo
poloautomatické spojování datasetů z~různých zdrojů (\textit{conflation}). 
Bohužel ani zde však neexistuje žádný komplexnější nástroj jako je výše
popsaný \textit{JCS}. Navíc většina těchto nástrojů je vytvořena pro~nějaký
konkrétní účel jako např. pro~spojování silničních sítí v~USA a~ne vždy
je proto jejich použití zcela obecné. Uživatel si tedy mnohdy musí poradit
sám a~použít několik různých nástrojů, aby dosáhl požadovaného výsledku.
Dále uvádím nástroje, které lze pro~některé činnosti související 
se~spojováním map použít. 

\subsubsection{JOSM conflation plugin}

\textit{Java OpenStreetMap Editor} (JOSM) je desktopová aplikace umožnující
editaci dat projektu OpenStreetMap. Jedním ze~zásuvných modulů pro~tuto 
aplikaci je \textit{Conflation}, který umožňuje spojování vektorových dat. 
Tento nástroj však je stále označen jako experimentální, což znamená, že ne 
vždy funguje zcela správně. Nástroj umožňuje zarovnat prvky jedné vrstvy tak,
aby souhlasily s~prvky druhé vrstvy, která je označena za~referenční. 

\label{josmpic}
  \begin{figure}[hbt]
    \centering
      \includegraphics[width=350pt]{./pictures/josm.png}
      \caption{JOSM conflation (\todo{zdroj})}
      \label{fig:josm}
  \end{figure} 

V~prvním kroku je určena referenční a~upravovaná vrstva. Poté je třeba v~obou
těchto vrstvách označit prvky, které by si měly odpovídat. V~dalším kroku se 
provede automatická identifikace dvojic vybraných prvků z~obou datasetů 
a~nakonec se provede samotný proces zarovnání prvků (\textit{conflate}), kdy 
jsou prvky upravované vrstvy změněny tak, aby odpovídaly prvkům vrtsvy 
referenční. Jedná se tedy o~proces poloautomatický, jelikož nejdříve je třeba
ručně identifikovat odpovídající si prvky a~na~základě tohoto kroku je 
provedena automatická úprava prvků.


\subsubsection{Potlatch 2 merging tool}

\textit{Potlatch 2 merging tool} je nástroj původně navržený pro~spojování 
vektorových dat cyklistických tras v~rámci England Cycling Data 
Project\footnote{Projekt pod záštitou britského ministerstva dopravy, 
který si klade za~cíl umožnit dostupnost informací o~síti cyklostezek
ve~Velké Británii prostřednictvím OpenStreetMap.}. %% tou záštitou si nejsem jistá
V~případě tohoto nástroje se jedná především o~proces přenosu atributů. 
Pokud máme v~mapě dva odpovídající si prvky, pomocí tohoto nástroje je 
můžeme označit za~odpovídající a~jednoduše sloučit jejich atributy. Nutno 
však podotknout, že výběr odpovídajících si prvků musíme provést vždy ručně. 


\subsection{Univerzitní projekty}
\label{univerzitní}

Mimo výše uvedené nástroje existuje také několik projektů různých světových
univerzit zabývajících se buď kombinací geo\-grafických dat obecně nebo 
spojováním map silničních sítí. Možnosti využití těchto projektů pro~běžného
uživatele se velmi různí. A~ne vždy lze zcela volně tyto projekty vyzkoušet. 
Jedním z~těchto projektů je i~Conflation System MBP, který byl vyvinut 
na~katedře počítačů (\textit{Computer Science Department}) na~Central Washington 
University v~USA. V~rámci tohoto projektu vznikl \textit{MBPConflate} 
software, který má přispět k~výzkumu v~oblasti slučování geo\-grafických dat. 
Program umožňuje automatické spojení map, poskytuje nástroje pro~následnou 
kontrolu kvali\-ty výsledné mapy. Je navržen tak, aby bylo snadné implementovat
nové techniky a~algoritmy v~této oblasti.  

\label{mbppic}
  \begin{figure}[hbt]
    \centering
      \includegraphics[width=350pt]{./pictures/MBPconflate.png}
      \caption{MBPConflate (\todo{zdroj})}
      \label{fig:mbp}
  \end{figure}

% \cleardoublepage\chapter{Zatím nevím}
\label{4-nevim}

\section{GEOS - Geometry Engine, Open Source}

\textit{GEOS (Geometry Engine - Open Source)} je knihovna implementující 2D prostorové predikáty a funkce dle~OGC specifikace \textit{Simple Features} pro SQL. \textit{GEOS}
je přepisem knihovny \textit{Java Topology Suite (JTS)}, o~níž jsem se zmiňovala výše, do~jazyka C++. Knihovna je projektem OSGeo (The Open Source Geospatial Foundation) a je
dostupná pod licencí LGPL.

\section{OGR Simple Feature Library}

\textit{OGR Simple Feature Library} je \textit{open source} knihovna umožňující čtení a poř. i zápis vektorových dat různých formátů jako je ESRI Shapefile, S-57, SDTS, 
PostGIS, Oracle Spatial atd. \textit{OGR} je součástí knihovny \textit{GDAL}\footnote{Geospatial Data Abstraction Library - knihovna umožňující čtení a zápis rastrových dat}.
% \cleardoublepage\chapter{Knihovna GEOC}
\label{5-geoc}

\section{Tvorba knihovny GEOC}
\label{knihovna}

Algoritmy týkající se slučování vektorových map (\textit{conflation}) byly
implementovány v~externí knihovně \textit{GEOC} bez závislosti na~QGIS API. 
Zásuvný modul \textit{Conflate} umožňuje využití funkcionality knihovny 
v~Quantum GIS. Díky tomuto oddělení vznikla nezávislá knihovna, kterou bude
možné případně použít i~v~jiných programech a~projektech.

Tato kapitola popisuje jednotlivé algoritmy a~jejich implementaci v~knihovně
\textit{GEOC}, zároveň také stručně pojednává o~možnostech  jejich využití.
Podrobný popis jednotlivých tříd a~jejich metod je podrobněji popsán v~dokumentaci. % viz příloha

% úvodní poznámky, proč samost. knihovna (aby se to dalo využít i jinde než v mém pluginu)
% co popisuje kapitola, vyjmenování algoritmů, jakou funkcionalitu by měla zajišťovat (stručně)

\section{Vertex Snapper} 
\label{vertexsnapper}

Při zpracování vrstev z~více zdrojů někdy stačí pouze upřesnit polohu či tvar 
prvků z~cílové vrstvy tak, aby se přiblížil prvkům z~vrstvy referenční. 
\mbox{Není-li} podrobnost obou datasetů příliš rozdílná, lze využít jednoduchého 
postupu \textbf{přichycení blízkých vrcholů}.

\subsection{Popis algoritmu}
\label{vs-algoritmus}

Nejjednodušším způsobem kombinace dvou vektorových vrstev je pouhé přichycení 
blízkých vrcholů cílové vrstvy k~vrstvě referenční. Obecný postup je následující:

\begin{enumerate}
 \item Na~počátku je třeba určit vzdálenostní toleranci, tedy maximální 
    vzdálenost mezi~dvěma body, kdy ještě bude provedeno jejich přichycení.
 \item Ke~každému prvku ze~zpracovávané vrstvy nalezneme nejbližší prvky 
    z~vrstvy referenční. To jsou prvky, jejichž nejkratší vzdálenost 
    od~zpracovávaného prvku není větší	než vzdálenostní tolerance.
 \item Pro~každý bod ze~zpracovávaného prvku vypočteme vzdálenosti ke~všem
    bodům z~blízkých referenčních prvků.
 \item Pokud nejmenší z~těchto délek je menší než vzdálenostní tolerance, 
    pak posuneme zpracovávaný bod do~odpovídajícího referenčního bodu s~touto
    nejmenší vzdáleností.
 \item Takto projdeme postupně všechny vrcholy všech prvků cílové vrstvy 
    a~snažíme se k~nim nalézt blízké body z~prvků vrstvy referenční. 
\end{enumerate}

% obrázek ilustrujici postup zpracovani
\label{vspic}
  \begin{figure}[hbt]
    \centering
      \includegraphics[width=350pt]{./pictures/vs-princip.pdf}
      \caption{Postup přichycení vrcholů (\todo{zdroj})}
      \label{fig:vs-princip}
  \end{figure}

\subsection{Implementace} % text asi přepsat 
\label{vs-implementace}
% popis mé implementace algoritmu + zmínit třídy a funkce s odkazy na literaturu
Algoritmus pro~přichycení vrcholů upravované vrstvy k~referenční je 
implementován ve~třídách \texttt{Vertex\-Snapper} 
a~\texttt{Vertex\-Geometry\-Editor\-Operation}. Při~použití v~externí aplikaci 
stačí po~předání vstupních parametrů třídě \texttt{Vertex\-Snapper} zavolat 
funkci \texttt{snap}. Ta vyhledá blízké prvky s~využitím prostorových 
indexů sestavených metodou \texttt{build\-Index} téže třídy.

Výsledky vyhledávání poté předá funkci \texttt{snap\-Vertices}. Uvnitř této
metody se vytvoří instance  třídy \texttt{Vertex\-Geometry\-Editor\-Operation},
která edituje příslušnou geometrii přichycením blízkých vrcholů.
\texttt{Vertex\-Geometry\-Editor\-Operation} je potomkem třídy 
\texttt{geos::\-operation::\-Coordinate\-Operation}, která je \textit{inter\-face}
\footnote{rozhraní třídy, kde jsou deklarovány pouze abstraktní metody bez
implementace, není možné vytvořit instanci této třídy} třídou pro editaci 
geometrie.


\subsection{Využití}
\label{vs-vyuziti}

Přichycení bodů jedné vrstvy k~vrstvě druhé má své výhody i~nevýhody, které 
je před volbou tohoto způsobu zpracování třeba zvážit. Použití této metody 
je vhodné v~takových případech, kdy máme k~dispozici dvě vrstvy o~rozdílné 
přesnosti (tento rozdíl však nesmí být příliš veliký) a~prvky vzájemně se 
překrývající. Cílem je upřesnit polohu a~tvar prvků z~méně přesného datasetu. 
Dopředu je třeba si uvědomit, že kromě polohy prvků je měněn i~jejich tvar.

Využít by tento postup šel i~pro~přichycení dvou sousedních vrstev o~stejné 
přesnosti, avšak to znamená, že by se změnil pouze tvar krajních prvků 
(nedošlo by k~posunu celé vrstvy), a~to nemusí být vždy žádoucí. Jako jediný
z~algoritmů \textit{GEOC} má pak smysl pro~bodové vrstvy.

Pro~rozumné výsledky je důležité zvolit vhodnou toleranční vzdálenost. Tato 
hodnota by měla odpovídat maximální vzdálenosti, o~kterou se vrchol prvku 
může posunout. Při~volbě příliš krátké vzdálenosti se výsledná vrstva nemusí 
vůbec odlišovat od té vstupní. Naopak \mbox{je-li} zvolená vzdálenost delší 
než nejkratší úsek geometrie (linie, polygonu), může dojít k~přichycení dvou 
bodů k~jednomu bodu z~referenční vrstvy. Zda je toto přípustné či nikoli je 
už na~rozhodnutí uživatele.

\label{vsinvalid}
  \begin{figure}[hbt]
    \centering
      \includegraphics[width=350pt]{./pictures/vs-invalid.pdf}
      \caption{Vznik nevalidní geometrie při přichycení vrcholů (\todo{zdroj})}
      \label{fig:vs-nevalidni}
  \end{figure} 

V~některých případech může dojít ke~vzniku nevalidních geometrií, to je takových,
jejichž segmenty se vzájemně protínají apod. To se nejčastěji stává u~protáhlých 
úzkých prvků a~jiných speciálních tvarů. Příkladem může být situace uvedená 
na~obrázku \ref{fig:vs-nevalidni}, kde růžový polygon je prvkem přichycen 
k~referenční fialové vrstvě, ale z~důvodu nedostatečné hustoty vrcholů a~nevhodného
tvaru vzniká nežádoucí křížení. 


%Někdy může být tento postup výhodný i~v~případě, kdy chceme odstranit drobné mezery či 
%překryty v~rámci jedné vrstvy. Poté stačí nastavit danou vrstvu jako referenční i~jako
%cílovou.  -- teď to ale nejde, protože se z mezer stanou překryty apod. (vrstva se v 
%průběhu nemění), ale šlo by to asi dodělat. 


\section{Coverage Alignment} 
\label{coverage alignment}

\textit{Coverage alignment} lze vysvětlit jako \textbf{zarovnání jedné vrstvy 
k~vrstvě druhé}. Tento způsob je složitější než výše uvedené přichytávání vrcholů.
V~knihovně \textit{GEOC} je využit opět pro~úpravu jedné vrstvy na~základě vrstvy 
referenční. Do~upravované vrstvy nejsou žádné prvky přidávány ani z~ní vymazávány,
pouze jde o~jejich modifikaci. Velmi podobný algoritmus se však dá použít 
i~ke~kombinaci dvou vrstev. 

\subsection{Popis algoritmu}
\label{ca-algoritmus}

Nejčastější používaný postup při~spojování vektorových map je následující.

\begin{enumerate}
 \item Nejprve je třeba nalézt odpovídající si prvky v~obou překrývajících se 
    vrstvách. Kritéria pro~určení odpovídajících si prvků mohou být velmi 
    odlišná. Existuje mnoho algoritmů řešících tuto problematiku, přičemž 
    postupy se mohou různit podle toho, zda je úkolem vyhledání 
    odpovídajících si bodů, polygonů či linií. Kritéria a~postup použitý 
    v~knihovně \textit{GEOC} je popsán níže.
 \item Poté, co se určí odpovídající si prvky, musí se určit totožné vrcholy 
    těchto dvojic prvků. Ty z~vrcholů, které jsou určeny s~dostatečnou 
    přesností (ta může být určena například danou vzdálenostní tolerancí), 
    jsou označeny jako body budoucí triangulační sítě.
 \item Jak už bylo naznačeno, z~nalezených bodů se vytvoří pomocí Delaunayho 
    triangulace\footnote{Delaunayho triangulace z~množiny bodů v~rovině vytvoří takovou 
    trojúhelníkovou síť, pro kterou platí, že v~kružnici opsané každému
    trojúhelníku, neleží žádný jiný bod. DT maximalizuje
    minimální úhly trojúhelníků.} trojúhelníková síť. 
    %ML: doplnit poznamku pod carou - kratka charakteristka DT .. doplneno
 \item Následně se provede lokální, nejčastěji afinní transformace v~každém 
    trojúhelníku sítě. Tak se přetransformují body cílové vrstvy do~systému 
    vrstvy referenční.
 \item Celý postup je možné iterativně opakovat, dokud nedosáhneme 
    požadovaného výsledku (ten může být dán např. podmínkou minimálního 
    množství nalezených odpovídajících si vrcholů či prvků).
\end{enumerate}

% obrázek ilustrujici postup zpracovani, vč. tinu apod.
\label{capic}
  \begin{figure}[hbt]
    \centering
      \includegraphics[width=400pt]{./pictures/ca-princip.pdf}
      \caption{Postup zarovnání vrstev (\todo{zdroj})}
      \label{fig:ca-princip}
  \end{figure}

V~knihovně \textit{GEOC} je pro nalezení odpovídajících si prvků využito 
obdobného postupu jako ve~výše zmiňované knihovně \textit{JCS}. Využívá 
se vrcholová Hausdorffova vzdálenost, přičemž tato vzdálenost není počítána 
přímo. Splnění podmínky, že dané prvky nejsou od~sebe dále, než je daná 
Hausdorffova vzdálenost, se testuje pomocí obalových zón jednotlivých prvků 
následujícím způsobem.

\begin{enumerate}
 \item Máme dva prvky A a~B ze~dvou různých překrývajících se vrstev.
 \item Pokud prvek B leží v~obalové zóně prvku A o~velikosti vzdálenostní 
    tolerance a~A leží v~obalové zóně prvku B o~stejné velikosti, je možné, 
    že si prvky odpovídají a~pokračuje se dalším krokem. V~opačném případě 
    si prvky neodpovídají.
 \item Dále se testuje, zda hranice prvku B leží v~obalové zóně hranice prvku
    A a~naopak. Je-li splněna i~tato podmínka, pak jsou prvky označeny 
    za~odpovídající si.
\end{enumerate}

%% ML: doplnit ilustraci .. obrázek obal. zón

\subsection{Implementace} % text asi přepsat 
\label{ca-implementace}
% popis mé implementace algoritmu + zmínit třídy a funkce s odkazy na literaturu
Veškeré zpracování je opět schováno pod jedinou funkcí \texttt{align} třídy
\texttt{Coverage\-Alignment}. Ta postupně volá metody provádějící jednotlivé
kroky algoritmu.

Nejdříve je třeba nalézt odpovídající si prvky. K~tomu slouží třída 
\texttt{Matching\-Geometry}, která k~dané geometrii najde odpovídající
(\texttt{set\-Match}). Obdobně jako u~\texttt{Vertex\-Snapper} je i~zde 
využito prostorových indexů, které jsou vytvořeny metodou 
\texttt{build\-Index}. Určení blízkých bodů je pak prostřednictvím
funkcí \texttt{choose\-Matching\-Points, \-find\-Closest\-Points,
\-clean\-Matching\-Points} a~dalšími třídy \texttt{Co\-ve\-ra\-ge\-Align\-ment}.

Třetím krokem je vytvoření TINu metodou \texttt{create\-TIN}, která k~tomu
využívá třídu \texttt{Tri\-an\-gu\-la\-tion}.  

Konečně je provedena postupně transformace všech prvků. Funkce pro
transformaci poskytuje třída \texttt{Affine\-Trans\-for\-mation},
která transformuje prvky na~základě předané geometrie a~triangulační
sítě. Ta je volána prostřednictvím třídy pro editaci 
\texttt{Align\-Geo\-metry\-Edi\-tor\-Ope\-ra\-tion}.


\subsection{Využití}
\label{ca-vyuziti}

Na~rozdíl od~předchozího algoritmu je tento trochu šířeji využitelný. Je opět 
vhodný pro~zpřesnění vrstvy dle vrstvy referenční, avšak tentokrát nejsou 
pouze přichytávány blízké vrcholy, ale jsou upravovány téměř všechny vrcholy. 
To zajišťuje reálnější výsledky i~v situacích, kdy hustota vrcholů v~obou 
datasetech je velmi rozdílná.

% poznámka o výhodách a nevýhodách využití u liniových a polygon. datech

% 
% 
% %**************************************************************
% 
% %\bibliographystyle{abbrv}
% %\bibliography{literatura}
% 
% \appendix
% 
% %\cleardoublepage\include{priloha-prirucka}
% 
% \end{document}
