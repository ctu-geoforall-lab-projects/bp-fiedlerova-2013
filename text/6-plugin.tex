\chapter{Zásuvný modul pro QGIS}
\label{6-plugin}

Tato kapitola popisuje tvorbu zásuvného modulu \textbf{Conflate}.
Uvedeny jsou nejdříve obecné informace o~zásuvných modulech v~programu
\zkratkatext{QGIS} a~poté už kon\-krétní popis tvorby modulu pro 
slučování map. 

\section{Tvorba QGIS zásuvného modulu}
\label{plugin-tvorba}

Do projektu \zkratkatext{QGIS} je možné psát zásuvné moduly v~jazycích C++ 
nebo Python. Dnes už jsou častější zásuvné moduly psané v~Pythonu, přesto 
se však objevují i~ty v~C++, a~to z~těchto důvodů. \zk{QGIS} jako takový 
je napsán v~tomto jazyce. Jedná se o~objektově orientovaný jazyk, hodí se 
proto pro větší projekty. Nevýhodou C++ pluginů je nutnost kompilace, kterou 
však vyváží výsledná vyšší rychlost běhu aplikace. V~případě zásuvného modulu 
\textbf{Conflate} byl jazyk C++ zvolen i~z~důvodu použití knihovny \zk{GEOS}, 
která je psána rovněž v~tomto jazyce.

Vzhledem k tomu, že C++ zásuvné moduly většinou využívají QGIS knihovny
\textit{libqgis*.so} publikováné pod licencí \zk{GPL}, musí být 
tyto moduly pod stejnou licencí.

C++ zásuvné moduly \zk{QGIS}u jsou dynamické knihovny (\textit{.so} nebo 
\textit{.dll} \footnote{ Přípona \textit{.so} se používá v~Linuxu, přípona 
\textit{.dll} pod systémem Windows.}). Načtení modulu v~\zk{QGIS}u probíhá 
za~běhu programu. Při startu aplikace \textit{Správce zásuvných modulů 
(Plugin Manager)} vyhledá všechny soubory s~danou příponou ve~složce 
se~zásuvnými moduly, popřípadě v~dalších složkách, jejichž umístění
je definováno v~uživatelském nastavení, a~načte je. Pro správné načtení musí 
zásuvný modul obsahovat následující.

\begin{enumerate}
 \item Funkce \texttt{classFactory()}, která vytvoří instanci daného pluginu.
 \item Metoda \texttt{initGui()}, jejímž prostřednictvím jsou zobrazeny prvky 
	uživatelského rozhraní (ikona apod.) v~nabídce \textit{Zásuvné moduly 
	(Plugins)} a~v~liště nástrojů.
 \item Metoda \texttt{unload()}, která odstraní alokované elementy a~samotnou
	instanci třídy zásuvného modulu (pomocí destruktoru) při ukončení 
	programu.
 \item Další externí C ~funkce (\texttt{na\-me(), descrip\-tion(), version(),
	category()}) pro správné zobrazení ve~\textit{Správci zásuvný modulů}.
\end{enumerate}

V~modulu \textbf{Conflate} jsou všechny tyto funkce obsaženy ve~třídě
\texttt{Qgs\-Con\-flate\-Plugin}. 
Podrobnější popis tvorby C++ pluginu lze najít v~\textit{QGIS Coding and 
Compilation Guide} \cite{qgisCCguide}, odkud byly čerpány výše uvedené 
informace.



\section{Zásuvný modul Conflate}
\label{plugin-navrh}

Zásuvný modul \textbf{Conflate} pro \zk{QGIS}, který je jedním z~výstupů této
práce, umožňuje spojení vektorových datasetů. \textbf{Conflate} provádí 
sjednocení dvou vrstev zvolených uživatelem vybranou metodou a~parametry. Pro 
tento proces úpravy vrstev využívá tříd a~metod výše popsané knihovny \zk{GEOC}. 
Výstupem je nová vrstva a~také textový protokol o~zpracování.

Modul se skládá celkem ze~tří tříd \texttt{QgsConflateProvider}, 
\texttt{QgsConflatePlugin} a~ \texttt{QgsDialog}. Třída \texttt{QgsConflatePlugin} 
obsahuje funkce pro vytvoření a~načtení zásuvného modulu do~\zk{QGIS} tak, jak 
bylo uvedeno v~kapitole \ref{plugin-tvorba}. Zobrazování a~interakci s~uživatelem 
zajišťuje třída \texttt{QgsDialog}. Vzhled grafického rozhraní byl vytvořen 
pomocí nástroje Qt Designer. Konečně třída \texttt{QgsConflateProvider} 
se stará o~funkcionalitu aplikace a~obsahuje tak funkce využívající knihovnu 
\zk{GEOC} a~další metody pro kopírování vrstvy, změnu geometrie, výpis 
protokolu apod.

\vspace{0.5cm}
\label{schema}
  \begin{figure}[h]
    \centering
      \input{./pictures/schema.pdf_tex}
      \caption{Architektura zásuvného modulu \textit{Conflate}}
      \label{fig:schema}
  \end{figure} 

Následuje výčet činností probíhajících při spuštění a~použití \textbf{Conflate}.
Tento výčet má za~úkol pouze zprostředkovat čtenáři představu, jak modul funguje.
Podrobný návod pro jeho použití z~hlediska uživatele je pak rozepsán v~příloze
\ref{priloha-prirucka} v~uživatelské příručce. Popis jednotlivých metod
lze nalézt v~anglické dokumentaci k~zásuvnému modulu (příloha~\ref{priloha-obsahCD}).

\begin{enumerate}
 \item Při otevření dialogu se do~rozbalovací nabídky načtou názvy vrstev 
	otevřených v~programu.
 \item Dalším krokem je výběr vrstev a~nastavení možností uživatelem.
 \item Po~spuštění zpracování se nejdříve zkopíruje upravovaná vrstva 
	(\textit{Subject layer}) do~nové vektorové vrstvy pomocí funkce 
	\texttt{copyLayer()}. Formát nové vrstvy je vždy 
	\textit{shapefile}\footnote{ESRI Shapefile je datový formát pro ukládání 
	vektorových prostorových dat  vyvinutý firmou Esri. Jedná se o~jeden 
	  z~nejrozšířenějších formátů dat pro geografické informační systémy,
	  přípona souborů v tomto formátu je \textit{.shp}.}.
	Tato vrstva je vytvořena v~aktuální složce a~její název, pokud není
	zadán uživatelem, odpovídá vzoru
	\begin{center}
	 \texttt{nazev\_puvodni\_vrstvy(nejnizsi\_nepouzite\_cislo).shp},
	\end{center}
	např. \texttt{subject(3).shp}.
 \item Poté je převedena geometrie prvků zvolených vrstev (upravovaná
	a~referenční) na~reprezentaci geometrie knihovny \zk{GEOC},
	tedy z~\texttt{Qgs\-Geo\-metry} na~\texttt{GEOC\-Geo\-metry}. 
	Tyto geometrie 	jsou uloženy do~vektorů, které se pak předají 
	třídám \zk{GEOC}.
 \item Jak už bylo naznačeno, je vytvořena instance příslušné třídy 
	\zk{GEOC} a~té jsou předány parametry - vektory s~prvky vybraných
	vrstev, toleranční vzdálenost a~případně i~další. 
 \item Dále je volána funkce vytvořené třídy pro zarovnání datasetů. Při volbě 
	při\-chycení vrcholů \textit{Snap vertices} v~dialogu je použita třída 
	\texttt{Vertex\-Snapper} a~metoda \texttt{snap()}, při volbě zarovnání 
	vrstev \textit{Coverage Alignment} třída \texttt{Cove\-rage\-Align\-ment} 
	a~metoda \texttt{align()}, poslední možností je \textit{Match Lines}, která
	využívá třídu \texttt{Line\-Matcher} a~její metodu \texttt{match()}. 
	Výsledek je uložen do~nového vektoru.
 \item Na závěr je upravena geometrie nové vrstvy (kopie upravované vrstvy) dle
	vý\-sledků úprav provedených algoritmy knihovny \zk{GEOC}.
 \item Nová vrstva se automaticky přidá do~aktuálního projektu v~\zk{QGIS} 
	a~zobrazí se v~panelu \textit{Vrstvy (Layers)}.
 \item Po~zpracování vrstvy se do textového okna dialogu vypíše protokol 
	o~zpracování, který obsahuje název vstupních a~výstupních dat, počet
	zpracovávaných prvků a~počet nevalidních prvků včetně výpisu jejich
	identifikátorů (\textit{id}).
\end{enumerate}





 

