%%%%%%%%%%%%%%%%%%%%%%%%%%%%%%%%%%%%%%%%%%%%%%%%%%%%%%%%%%%%%%%%%%%%%%%%%%%%%%%%%%%
%%                 PŘÍLOHA - TESTOVÁNÍ                                           %%
%%%%%%%%%%%%%%%%%%%%%%%%%%%%%%%%%%%%%%%%%%%%%%%%%%%%%%%%%%%%%%%%%%%%%%%%%%%%%%%%%%%
\chapter{Testování rychlosti algoritmů}
\label{priloha-testovani}

Testování bylo provedeno na~přenosném počítači s~parametry uvedenými v~tabulce 
\ref{tab:parametry}.

\begin{table}[!h]
 \centering
\begin{tabular}{l|l}
 typ počítače & \textit{HP ProBook 6450b} \\
 paměť & \textit{4 GiB} \\
 procesor &\textit{Intel® Core™ i5 CPU M 450 @ 2.40GHz $\times$ 4 }\\
 operační systém &\textit{Ubuntu 12.04 LTS, 64 bit}\\
\end{tabular}
  \caption{ Parametry počítače využitého pro testování}
  \label{tab:parametry}
\end{table}

V~následujících tabulkách jsou uvedeny časy zpracování pro danou upravovanou
a~referenční vrstvu a~zvolenou toleranční vzdálenost. Časy jsou uváděny 
v~sekundách.

\begin{table}[!h]
\centering
 \begin{tabular}{|c|c|c|c|c|}
  \hline
     & \multicolumn{2}{c|}{vyber\_obce, vyber\_CR} & 
	\multicolumn{2}{c|}{zel1, zel2} \\
  \hline
   id  &  ~~100 m~ & 10 000 m & ~~~100 m & 10 000 m\\
  \hline
  \hline
  1  & 19.89 & 23.52 & 4.41 & 5.16 \\ 
2  & 19.83 & 23.00 & 4.41 & 4.44 \\
3  & 21.98 & 23.63 & 4.43 & 4.66 \\
4  & 22.09 & 21.40 & 4.89 & 5.18 \\
5  & 21.99 & 21.50 & 4.90 & 5.20 \\
6  & 19.89 & 22.59 & 4.89 & 4.67 \\
7  & 19.76 & 22.59 & 4.91 & 5.17 \\
8  & 19.77 & 22.21 & 4.89 & 5.15 \\
9  & 21.95 & 23.61 & 4.45 & 4.92 \\
10 & 22.00 & 21.97 & 4.48 & 4.68 \\
  \hline
  \hline
  průměr & 20.92& 22.60 &4.67 & 4.92 \\
  \hline
 \end{tabular}
  \caption{ Čas zpracování [s] bez prostor. indexů pro 
	    \texttt{Vertex\-Snapper}}
  \label{tab:vs-bez}
\end{table}

\begin{table}
\centering
 \begin{tabular}{|c|c|c|c|c|}
  \hline
     & \multicolumn{2}{c|}{vyber\_obce, vyber\_CR} & 
	\multicolumn{2}{c|}{zel1, zel2} \\
  \hline
   id  &  ~~100 m~ & 10 000 m & ~~~100 m & 10 000 m\\
  \hline
  \hline
  1  &0.78 & 0.69 &  0.45 & 0.47 \\
  2  &0.75 & 0.72 &  0.51 & 0.49 \\
  3  &0.69 & 0.76 &  0.48 & 0.50 \\
  4  &0.72 & 0.71 &  0.50 & 0.45 \\
  5  &0.77 & 0.78 &  0.48 & 0.48 \\
  6  &0.82 & 0.76 &  0.49 & 0.49 \\
  7  &0.90 & 0.70 &  0.43 & 0.50 \\
  8  &0.69 & 0.78 &  0.47 & 0.44 \\
  9  &0.68 & 0.69 &  0.43 & 0.51 \\
  10 &0.76 & 0.84 &  0.44 & 0.49 \\
  \hline
  \hline
  průměr & 0.76 & 0.74 &0.47 &0.48\\
  \hline
 \end{tabular}
  \caption{ Čas zpracování [s] s prostor. indexy pro 
	    \texttt{Vertex\-Snapper}}
  \label{tab:vs-s}
\end{table}

\begin{table}
\centering
 \begin{tabular}{|c|c|c|c|c|}
  \hline
     & \multicolumn{2}{c|}{vyber\_obce, vyber\_CR} & 
	\multicolumn{2}{c|}{zel1, zel2} \\
  \hline
   id  &  ~~100 m~ & ~1 000 m & ~1 000 m & 10 000 m\\
  \hline
  \hline
1  & 38.75 & 37.68 & 10.02 & 11.45 \\ 
2  & 38.55 & 37.93 & 10.36 & 8.99  \\
3  & 38.51 & 37.44 & 10.24 & 10.06 \\
4  & 41.60 & 36.95 & 11.22 & 10.41 \\
5  & 34.02 & 41.81 & 10.04 & 11.22 \\
6  & 40.31 & 38.55 & 10.44 & 10.02 \\
7  & 38.77 & 41.39 & 11.21 & 9.88 \\
8  & 38.54 & 36.79 & 10.09 & 11.30 \\
9  & 37.69 & 39.69 & 9.67  & 11.32 \\
10 & 36.72 & 41.60 & 11.67 & 10.29 \\
  \hline
  \hline
  průměr & 38.35 & 38.98 & 10.51 & 10.49\\
  \hline
 \end{tabular}
  \caption{ Čas zpracování [s] bez prostor. indexů pro 
	    \texttt{Coverage\-Alignment}}
  \label{tab:ca-bez}
\end{table}

\vspace{-200pt} % najít elegantnější řešení
\begin{table}
\centering
 \begin{tabular}{|c|c|c|c|c|}
  \hline
     & \multicolumn{2}{c|}{vyber\_obce, vyber\_CR} & 
	\multicolumn{2}{c|}{zel1, zel2} \\
  \hline
   id  &  ~~100 m~ & ~1 000 m & ~1 000 m & 10 000 m\\
  \hline
  \hline
1  &1.05 & 1.82 &  0.73 & 3.99 \\
2  &1.13 & 1.78 &  0.69 & 3.98 \\
3  &1.14 & 1.66 &  0.68 & 4.01 \\
4  &1.04 & 1.65 &  0.77 & 4.37 \\
5  &1.04 & 1.79 &  0.78 & 4.39 \\
6  &1.03 & 1.79 &  0.70 & 3.96 \\
7  &1.06 & 1.87 &  0.69 & 3.99 \\
8  &1.12 & 1.84 &  0.77 & 3.99 \\
9  &1.02 & 1.80 &  0.68 & 4.39 \\
10 &1.14 & 1.76 &  0.70 & 4.38 \\
  \hline
  \hline
  průměr & 1.08 &1.78 & 0.72 & 4.15 \\
  \hline
 \end{tabular}
  \caption{ Čas zpracování [s] s~prostor. indexy pro 
	    \texttt{Coverage\-Alignment}}
  \label{tab:ca-s}
\end{table} 

%%%%%%%%%%%%%%%%%%%%%%%%%%%%%%%%%%%%%%%%%%%%%%%%%%%%%%%%%%%%%%%%%%%%%%%%%%%%%%%%%%%
%%                 PŘÍLOHA - UŽIVATELSKÁ PŘÍRUČKA                                %%
%%%%%%%%%%%%%%%%%%%%%%%%%%%%%%%%%%%%%%%%%%%%%%%%%%%%%%%%%%%%%%%%%%%%%%%%%%%%%%%%%%%

\newpage
\chapter{Uživatelská příručka}
\label{priloha-prirucka}

Tato příručka je psána pro použití modulu \textit{Conflate} s~Quantum GIS 1.9.0 , 
v~jiných verzích se může způsob načtení modulu, popř. i~jiné činnosti mírně lišit.

\section{Instalace}
\label{prirucka-instalace}
Instalace ... %Je potřeba knihovna \textit{GEOC} ..

\section{Načtení zásuvného modulu}
\label{prirucka-nacteni}

Načtení zásuvného modulu lze provést ve~Správci zásuvných modulů.
\begin{center}
\textit{Zásuvné moduly~$\rightarrow$~Spravovat~zásuvné~moduly... 
(Plugins~$\rightarrow$~Plugin Manager...)}
\end{center}
Zde je třeba zaškrtnout \textit{Conflate Plugin}. Po~provedení tohoto kroku by se 
měla objevit ikonka modulu v~nástrojové liště a~také v~menu \textit{Zásuvné moduly}. 
Pokud se plugin nezobrazuje ve~Správci zásuvných modulů, je třeba nastavit cestu 
k~souboru~\textit{.so} v~\textit{Nastavení}.
\begin{center} 
\textit{Nastavení~$\rightarrow$~Volby~$\rightarrow$~Systém~$\rightarrow$~Cesty 
k~zásuvným modulům (Settings~$\rightarrow$~Options$\rightarrow$~System~$\rightarrow$
~Plugin paths)}
\end{center}


\section{Spuštění a nastavení dialogu}
\label{prirucka-spusteni}

\subsection{Výběr vstupních vrstev}
Před spuštěním dialogu \textit{Conflate} je třeba mít v~aktuálním projektu 
načteny vrstvy, které chceme zpracovávat. Pokud přidáme vrstvu až po~otevření
dialogu, lze ji načíst do~výběru vrstev tlačítkem \textit{Refresh}.
Po~otevření dialogu je třeba provést výběr \textbf{re\-ferenční vrstvy} 
(\textit{Reference Layer}) a~\textbf{upravované vrstvy} (\textit{Subject Layer}). 
Re\-ferenční vrstva je obvykla ta s~vyšší přesností, která se nebude měnit. 
Upravovanou vrstvu naopak chceme zarovnat k~vrstvě referenční. 
Obě vrstvy by měly být stejného geometrického typu (polygon~-~polygon, 
linie~-~linie, bod~-~bod).

\subsection{Metoda zpracování}
Dalším krokem je volba způsobu zpracování (\textit{Select the way of conflation}). 
Na~výběr jsou tyto metody:

\begin{itemize}
 \item \textbf{Přichycení vrcholů} (\textit{Snap Vertices}) - tato metoda vyhledá 
	blízké vrcholy z~obou datasetů a~změní polohu bodů upravované vrstvy tak, 
	aby odpovídala poloze blízkých bodů vrstvy referenční.
% obrázek, jak funguje ?
 \item \textbf{Zarovnání vrstev} (\textit{Coverage Alignment}) - princip této 
	metody je složitější. Na~rozdíl od~předchozí metody nepracuje s~jednotlivými
	body, ale s celými prvky. Upravuje i~některé prvky, k nimž neexistují 
	žádné odpovídající ve~vrstvě refe\-renční, a~to na~základě změny okolních 
	prvků. Je proto vhodnější zejména pro datasety, které mají rozdílný počet
	prvků. Obecně je možné s~touto metodou dosáhnout přesnějších a~často
	reálnějších výsledků, avšak na~úkor času zpraco\-vání.
% obrázek, jak funguje ?
 \item \textbf{Napasování linií} (\textit{Match lines} - tato metoda vyhledá
	odpovídající si úseky linií ze dvou různých vrstev. Takto nalezené
	páry zprůměruje a~vytvoří z~nich nové linie. Při volbě tohoto způsobu
	zpracování nezáleží na~tom, která vrstva je referenční a~která upravovaná.
	Napasování linií lze použít pouze na~dvojice liniových vrstev.
\end{itemize}

\subsection{Další nastavení}
Poté je třeba nastavit \textbf{toleranční vzdálenost} (\textit{Distance 
Tolerance}) v~jednotkách projektu. Tato vzdálenost udává, v~jaké maximální 
vzdálenosti mohou být odpovídající si prvky z~obou vrstev a~zároveň, jak 
moc se tedy může cílová vrstva měnit. V~ideálním případě by tato vzdálenost 
neměla přesahovat nejkratší segment geometrie všech prvků upravované vrstvy 
(tzn. nejkratší úsek linie, nejkratší stranu polygonu). Nevhodná volba této 
vzdálenosti může vést ke~vzniku nevalidních geometrií či nereálným výsledkům.

U~poslední metody (\textit{Match Lines}) se zviditelní ještě možnost nastavení
\textbf{kritéria podobnosti} \textit{Matching criterium}. Jedná se o~minimální 
podobnost dvou segmentů, která musí být dodržena pro jejich spárování. 
Tato hodnota se udává v~procentech. Při zadání $100 \%$ jsou výsledkem pouze 
naprosto totožné segmenty. Kritérium podobnosti se počítá na základě úhlu mezi 
segmenty, vzdálenosti jejich koncových bodů a~rozdílu jejich délek. 

Volba toleranční vzdálenosti a~metody zarovnání může výrazně ovlivnit rychlost 
zpracování, je proto doporučeno volit raději vždy menší vzdálenost a~jednodušší
metodu a~teprve po zobrazení výsledků případně parametry změnit.

Poslední nastavovanou možností je \textbf{automatická oprava geometrie}
 zaškrtnutím \textit{Try to repair invalid geometries}. Při této volbě se 
program pokusí opravit nově vzniklé nevalidní geometrie. Opravováno je pouze 
křížení úseků v~polygonu a~tzv. \uv{slepé větve}. Automatická oprava však 
může ovlivnit přesnost výsledku a~někdy i~vytvořit topologické chyby, proto 
je třeba ji používat opatrně.

Po~nastavení všech parametrů (včetně výstupní vrstvy - viz dále) se stisknutím 
tlačítka \textit{Process} spustí zpracování. 

\subsection{Výstupní vrstva}

Před spuštěním zpracování lze ještě nastavit název a~cestu, kam se má uložit
výsledný soubor. Upravená vrstva se vždy uloží jako nová vrstva ve~formátu 
\textit{shapefile}. Do~pole \textit{Output shapefile} lze zadat \textbf{cestu 
k~výstupnímu souboru}, popřípadě ji vybrat pomocí tlačítka \textit{Browse} 
(\textit{Procházet}) vpravo. 

Pokud zadáte pouze název výstupní vrstvy nebo toto pole necháte prázdné, 
uloží se nová vrstva do~aktuálního adresáře pod daným názvem nebo pod názvem
upravované vrstvy s~příslušným pořadovým číslem. 


\section{Výsledek}
\label{prirucka-vysledek}

Upravená vrstva se automaticky přidá do~otevřeného projektu v~QGISu.

V~dialogu zásuvného modulu se do~textového pole vlevo vypíše protokol o~zpracování.
Formát protokolu je popsán na~obrázku \ref{fig:protokol}.  

\label{protokol}
  \begin{figure}[H]
    \centering
      %\input{./pictures/protokol.pdf_tex}
      \includegraphics{./pictures/protokol.pdf}
      \caption{Formát protokolu}
      \label{fig:protokol}
  \end{figure} 

Výsledkem zpracování mohou být i~nevalidní geometrie, jejichž identifikátory lze
nalézt v~protokolu. Proto jsou často nezbytné ještě závěrečné úpravy s~využitím
editačních nástrojů QGISu.