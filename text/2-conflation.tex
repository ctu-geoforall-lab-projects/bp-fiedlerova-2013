\chapter{Conflation}
\label{2-conflation}

\section{Definice 'conflation'}

\textit{Conflation} neboli \textit{map matching, map merging, rubber sheeting} lze do češtiny přeložit jako slučování map. Překlad tohoto pojmu však není vždy jasný, jelikož 
se jím označuje více různých úloh, procesů či činností. Proto zde uvádím několik definic z různýc zdrojů.

\begin{itemize}
 \item 'Conflation' je proces, jehož cílem je geometricky upravit (posunem, transformací) digitální data zobrazující stejné území (obvykle pořízená v~jiném čase) tak, aby byla 
    vzájemně geograficky korektní a vzájemně se překrývaly. %% slovem korektní si nejsem jistá 
    
    (huic.com)

  \item 'Conflation' je proces sjednocení dvou rozdílných datasetů.
    
    (Blasby, Davis, Kim, Ramsey: GIS Conflation Using Open Source Tools)

  \item 'Conflation' se zabývá kombinováním dat z~různých zdrojů. Jedná se o~jeden z~aktuálních problémů v~GIS.
   
    (Freitas, Afonso: Distributed Vector Based Spatial Data Conflation Services)

  \item 'Conflation' je soubor funkcí a procedur, které zarovnávají prvky jedné GIS vrstvy k~prvkům jiné a následně převádí atributy mezi~těmito vrstvami.
   
    (http://www.princegeorgescountymd.gov/Government/AgencyIndex/\newline OITC/GIS/glossary.asp)

  \item 'Conflation' je proces sladění poloh odpovídajících si prvků v~různých datových vrstvách. 'Conflation' funkce provádějí toto sladění tak, aby se odpovídající si prvky
    přesně překrývaly. 
    
    (http://maps.unomaha.edu/Peterson/GIS/notes/GISAnal1.html,\newline http://www.gov.ns.ca/snsmr/land/standards/post/manual/appedxa1.asp)

  \item Sada činností, které vzájemně přizpůsobují prvky dvou geografických datových vrstev a následně převádí atributy jedné vrstvy do~druhé. %% možná spíše zarovnávají než přizpůsobují
    
    (http://wiki.gis.com/wiki/index.php/GIS\_Glossary/C)
  
  \item 'Conflation' prvků je proces kombinování geografických informací z~překrývajících se zdrojových dat tak, aby byla zachována přesnost dat, minimalizována nadbytečná data
    a odstraněny konflikty mezi~daty. %% nebo rozdíly?
    'Conflation' geoprostorových dat je spojení či sladění dvou různých prostorových datasetů zahrnujících stejné území.
    
    (Shekhar, Xiong: Encyclopedia of GIS)
\end{itemize}


\section{Související pojmy}

adjustment, alignment, attribute transfer (přenos, převod atributů), geometry (geometrie - jako geometrický prvek), feature (prvek), feature collection (kolekce prvků),
internal conflation (v rámci jednoho datasetu), matching (odpovídající si), reference dataset (referenční dataset), subject dataset (upravovaný dataset), topologie, ... 

\section{Klasifikace 'conflation'}

\subsection{Dle typu vstupních vrstev}

\begin{enumerate}
  \item Imagery-to-Imagery
  \item Vector-to-Imagery
  \item Vector-to-Vector
\end{enumerate}


\subsection{Dle území zobrazovaného vstupními vrstvami}

\begin{enumerate}
  \item Vertikální - překrývající se datasety
  \item Horizontální - sousedící datasety
\end{enumerate}


\section{Obecný postup}

Obecný postup při~slučování vektorových map z~více zdrojů se skládá z~několika kroků, které jsou uvedeny níže. Jako u~většiny podobných operací je třeba nejdříve provést 
přípravu dat a následně data zpracovat a upravit.

\begin{enumerate}
  \item \textbf{Předzpracování dat}
    \subitem Tento krok slouží k~zajištění kompatibility vstupních dat tak, aby bylo možné je porovnávat. Obvykle spočívá v~převedení vstupních datasetů do stejného 
	      souřadnicového systému, zajištění stejných základních jednotek a dalších měnších úpravách. 
  \item \textbf{Kontrola kvality dat a topologické správnosti vrstev}
    \subitem Zde se kontroluje vnitřní konzistence dat každé vstupní vrstvy. V~tomto případě záleží především na~požadavcích zvoleného algoritmu pro~sloučení map. Může jít
	      například o~odstranění topologických chyb v~dané vrstvě jako jsou nežádoucí drobné překryty či mezery mezi~polygony.
  \item \textbf{Vyhledání odpovídajících si prvků}
      \subitem Následuje vyhledání prvků, které si v~upravovaných datasetech odpovídají, tedy zobrazují stejný předmět ve~skutečnosti. Tento krok je nezbytný proto, aby bylo
		možné rozhodnout, jak na~sebe vstupní vrstvy navazují.
  \item \textbf{Sloučení geometrických prvků a/nebo atributů}
      \subitem Po~rozpoznání odpovídajících si prvků je už možné upravit geometrii či atributy prvku z~jedné vrstvy s~přihlédnutím k~vlastnostem prvku z~jiné vrstvy. 
		Při~procesu slučování různých datasetů lze převádět pouze atributové hodnoty mezi~odpovídajícími si prvky nebo pouze změnit geometrii prvku tak, aby odpovídala
		geometrii prvku z~jiné vrstvy, která bývá označena za~referenční. Ve~složitějších úlohách už je možné převádět zároveň atributy i geometrii, a to nejen 
		jednosměrně (tedy z~vrstvy referenční do~vrstvy upravované), ale výsledkem může být vrstva, jejíž geometrie je kombinací geometrických vlastností obou 
		vstupních vrstev.
  \item \textbf{Závěrečné úpravy}
      \subitem Po~provedení automatického a někdy i manuálního sloučení datových vrstev je vhodné provést kontrolu výsledku. Často jsou potřeba ještě drobné úpravy, aby výsledná
		data odpovídala počátečním požadavkům. Ne vždy jsou totiž při automatickém procesu určeny správně všechny odpovídající si prvky a některé algoritmy mohou
		při úpravě geometrických vlastností narušit topologickou správnost dat.
\end{enumerate}