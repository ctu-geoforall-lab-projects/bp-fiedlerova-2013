\chapter{Conflation}
\label{2-conflation}

\section{Definice pojmu \textit{conflation}}
\label{definice}

\textit{Conflation} neboli \textit{map matching, map merging, rubber sheeting} 
lze do~češtiny přeložit jako slučování či spojování map. Překlad tohoto pojmu 
však není vždy jasný, jelikož se jím označuje více různých úloh, procesů či 
činností. Proto zde uvádím několik definic z~různýc zdrojů. 

\begin{itemize}[leftmargin=*]
 \item Proces, jehož cílem je geometricky upravit (posunem, transformací) 
    digitální data zobrazující stejné území (obvykle pořízená v~jiném čase) 
    tak, aby byla vzájemně geograficky korektní a~vzájemně se překrývaly. %% slovem korektní si nejsem jistá 
    
    (huic.com)

  \item Proces sjednocení dvou rozdílných datasetů.
    
    (Blasby, Davis, Kim, Ramsey: GIS Conflation Using Open Source Tools)

  \item Zabývá se kombinováním dat z~různých zdrojů. Jedná se o~jeden 
    z~aktuálních problémů v~GIS.
   
    (Freitas, Afonso: Distributed Vector Based Spatial Data Conflation 
    Services)

  \item Soubor funkcí a~procedur, které zarovnávají prvky jedné GIS vrstvy
    k~prvkům jiné a~následně převádí atributy mezi~těmito vrstvami.
   
    (http://www.princegeorgescountymd.gov/Government/AgencyIndex/
    \newline OITC/GIS/glossary.asp)

  \item Proces sladění poloh odpovídajících si prvků v~různých datových 
    vrstvách. Funkce pro spojení map (\textit{conflation}) provádějí toto 
    sladění tak, aby se odpovídající si prvky přesně překrývaly. 
    
    (http://maps.unomaha.edu/Peterson/GIS/notes/GISAnal1.html,
    \newline http://www.gov.ns.ca/snsmr/land/standards/post/manual/appedxa1.asp)

  \item Sada činností, které vzájemně přizpůsobují prvky dvou geografických
    datových vrstev a~následně převádí atributy jedné vrstvy do~druhé. %% možná spíše zarovnávají než přizpůsobují
    
    (http://wiki.gis.com/wiki/index.php/GIS\_Glossary/C)
  
  \item \textit{Feature conflation} je proces kombinování geografických 
    informací z~překrývajících se zdrojových dat, který zachovává přesnost 
    dat, minimalizuje nadbytečná data a~předchází konfliktům v~datech. %% nebo rozdíly?
    \textit{Conflation of geospatial data} (geoprostorových dat) je spojení
    či sladění dvou různých prostorových datasetů zahrnujících stejné území.
    
    \citation{gisencyclopedia} (Shekhar, Xiong: Encyclopedia of GIS)

\end{itemize}

\section{Historie} % asi ještě rozšířit, viz enc. of gis - geospatial conflation
\label{historie}

Až do 80.~let 20.~století bylo pořízení digitálních dat velmi drahé a~proto 
se často nestávalo, že by nějaká firma vlastnila více digitálních map jediného 
území. S~vývojem počítačů a~digitálních technologií se však tato data stávala 
stále dostupnějšími a~náhle vyvstala potřeba kombinovat data o~jednom území 
z~více zdrojů a~provádění aktualizace těchto dat. 

Jako první se k~takovému množství dat dostaly pochopitelně různé vládní 
instituce. Ačkoli první článek o~problematice spojování geografických dat 
z~více zdrojů vyšel už v~roce 1981 (\textit{M. White: The~Theory of~Geographical
Data Conflation}), k~opravdovému rozvoji došlo až po~roce 1985. V~té době 
totiž vznikl projekt, jehož cílem bylo propojení mapových souborů organizací
US~Census Bureau a~US Geological Society. Vzhledem k~množství dat bylo nutné
celý proces co nejvíce automatizovat. Použitý algoritmus, jehož hlavním autorem
je Alan Saalfeld, byl založen na~nalezení odpovídajících si prvků a~následné 
transformaci dat. 

S~narůstající dostupností dat a~zveřejněním této myšlenky přibývalo i~menších 
firem zabývajících se problémem kombinace mapových souborů z~různých zdrojů. 
Původní algoritmy brali v~potaz pouze geometrickou podobnost prvků, později 
byly brány v~úvahu i~jejich topologické vztahy a~s~rozvojem GIS technologií 
pak i~podobnost atributů jednotlivých prvků. Dnes jde o~jeden z~aktuálních 
problémů řešených v~oblasti GIS.


\section{Související pojmy} % slovníček dále používaných pojmů atd.
\label{pojmy}

adjustment, alignment, attribute transfer (přenos, převod atributů), geometry
(geometrie - jako geometrický prvek), feature (prvek), feature collection 
(kolekce prvků),ninternal conflation (v rámci jednoho datasetu), matching 
(odpovídající si), reference dataset (referenční dataset), subject dataset 
(upravovaný dataset), topologie, ... 


\section{Klasifikace (\textit{conflation})}
\label{klasifikace}

\subsection{Dle typu vstupních vrstev}
\label{dle-vstupu}

% možná obrázky k jednotlivým typům
\begin{enumerate}[leftmargin=*]
  \item \textbf{Imagery-to-Imagery, Raster-to-Raster}
    \subitem Jedná se o~případ, kdy je úkolem na~sebe napasovat dva rastrové 
      mapové soubory. Nejčastěji jde o~ortofoto a~naskenovanou analogovou mapu
      daného území. Tato úloha se využije, například pokud chceme porovnat 
      starou mapu se~současným stavem reprezentovaným právě leteckým snímkem. 
      Řešení tohoto problému vyžaduje poměrně složité techniky pro~nalezení 
      odpovídajících si objektů. Velmi důležitým prvkem je kvalita a~rozlišení
      vstupních dat.
  \item \textbf{Vector-to-Imagery, Vector-to-Raster}
    \subitem Kombinace vektorových a~rastrových dat stejného území se využívá 
      často ke~zpřesnění vektorových dat jejich napasováním na~ortofoto. Hlavní
      náplní této oblasti je vývoj algoritmů umožňující následující: % dopsat referenci na lit. - gis enc. 
	      \begin{enumerate}[leftmargin=*]
	       \item Detekce charakteristických hran rastrového obrazu a~jejich 
		  porovnání s~vektorovými daty.
	       \item Využití vektorových dat k~identifikaci hran v~rastru - tzv. 
		  \textit{Snakes algorithm}.
	       \item Užití stereo obrazu, výškových dat a~znalostí silničních dat
		  pro~porovnání vektorových a~rastrových dat. % ??????????
	       \item Využití prostorových informací stejně jako dalších vlastností
		  mapy (jako např. barev) k~rozpoznání odpovídajících si prvků.
	      \end{enumerate}
  \item \textbf{Vector-to-Vector}
    \subitem Případem sloučení dvou vektorových datasetů se zabývá tato práce. 
	Jednou ze~specifických a~velmi častých aplikací je navázání silničních 
	sítí, dále aktualizace digitálních dat aj. Existuje mnoho různých algoritmů
	pro~slučování vektorových dat, přičemž základní přístupy k~řešení problému 
	jsou následující: % dopsat referenci na lit. - gis enc. 
	      \begin{enumerate}[leftmargin=*]
	       \item Sloučení dat za~pomoci algoritmů, které pracují na~základě 
		  porovnávání geometrických vlastností prvků.
	       \item Algoritmy, které berou v~potaz podobnost tvarů prvků a~zároveň
		  podobnost jejich atributů.
	       \item Spojení vektorových dat s~neznámým souřadnicovým systémem 
		  na~základě rozložení významných bodových prvků (např. křižovatky 
		  cest).
	      \end{enumerate}

\end{enumerate}


\subsection{Dle území zobrazovaného vstupními vrstvami}
\label{dle-uzemi}

\begin{enumerate}[leftmargin=*]
  \item \textbf{Horizontální} % - sousedící datasety
    \subitem Za~\textit{horizontal conflation} se označují procesy, které 
	zpracovávají data ze~sousedících území. Cílem je získat mapové 
	soubory, jejichž hranice na~sebe dokonale navazují, a~to pokud 
	možno bez~ztráty přesnosti.
  \item \textbf{Vertikální} % - překrývající se datasety
    \subitem Při~\textit{vertical conflation} obsahují vstupní data překrývající
	se území. Jde tedy o~dva či více souborů zobrazujících jediné území, 
	třeba i~jen částečně. Může se jednat o~dvě verze té samé mapy nebo 
	o~dva datasety s~nějakými společnými prvky a~vlastnostmi. Výsledkem 
	celého procesu je jediný dataset, jehož přesnost není horší než přesnost
	původních dat a~obsahuje informace z~obou zdrojů, jde tedy o~vylepšenou, 
	obsahově bohatší mapu s~odpovídající přesností. 
\end{enumerate}


\section{Obecný postup}
\label{postup}

Obecný postup při~slučování vektorových map z~více zdrojů se skládá z~několika
kroků, které jsou uvedeny níže. Jako u~většiny podobných operací je třeba 
nejdříve provést přípravu dat a~následně data zpracovat a~upravit.

\begin{enumerate}[leftmargin=*]
  \item \textbf{Předzpracování dat}
    \subitem Tento krok slouží k~zajištění kompatibility vstupních dat tak, 
      aby bylo možné je porovnávat. Obvykle spočívá v~převedení vstupních 
      datasetů do~stejného souřadnicového systému, zajištění stejných 
      základních jednotek a~dalších měnších úpravách. 
  \item \textbf{Kontrola kvality dat a~topologické správnosti vrstev}
    \subitem Zde se kontroluje vnitřní konzistence dat každé vstupní vrstvy.
      V~tomto případě záleží především na~požadavcích zvoleného algoritmu 
      pro~sloučení map. Může jít například o~odstranění topologických chyb 
      v~dané vrstvě jako jsou nežádoucí drobné překryty či mezery 
      mezi~polygony.
  \item \textbf{Vyhledání odpovídajících si prvků}
      \subitem Následuje vyhledání prvků, které si v~upravovaných datasetech
      odpovídají, tedy zobrazují stejný předmět ve~skutečnosti. Tento krok je 
      nezbytný proto, aby bylo možné rozhodnout, jak na~sebe vstupní vrstvy 
      navazují.
  \item \textbf{Sloučení geometrických prvků a/nebo atributů}
      \subitem Po~rozpoznání odpovídajících si prvků je už možné upravit 
      geometrii či atributy prvku z~jedné vrstvy s~přihlédnutím k~vlastnostem 
      prvku z~jiné vrstvy. Při~procesu slučování různých datasetů lze převádět 
      pouze atributové hodnoty mezi~odpovídajícími si prvky nebo pouze změnit 
      geometrii prvku tak, aby odpovídala geometrii prvku z~jiné vrstvy, která 
      bývá označena za~referenční. Ve~složitějších úlohách už je možné převádět
      zároveň atributy i~geometrii, a~to nejen jednosměrně (tedy z~vrstvy
      referenční do~vrstvy upravované), ale výsledkem může být vrstva, jejíž 
      geometrie je kombinací geometrických vlastností obou vstupních vrstev.
  \item \textbf{Závěrečné úpravy}
      \subitem Po~provedení automatického a~někdy i~manuálního sloučení datových
      vrstev je vhodné provést kontrolu výsledku. Často jsou potřeba ještě drobné
      úpravy, aby výsledná data odpovídala počátečním požadavkům. Ne vždy jsou 
      totiž při~automatickém procesu určeny správně všechny odpovídající si prvky
      a~některé algoritmy mohou při~úpravě geometrických vlastností narušit 
      topologickou správnost dat.
\end{enumerate}


\section{Využití - hlavní aplikace}
\label{využití}

\subsection{oblasti, obory}
\label{obory}
kartografie, GIS, výpočetní geometrie, letecká fotogrammetrie, vojenský výcvik 
a~výzkum, krizový management, dopravní mapy - aktualizace dat, trh s nemovitostmi, ...

\subsection{aplikace}
\label{aplikace}
zpřesnění map, aktualizace prostorových dat, georeferencování - záznam dat, detekce 
chyb, rozdílů, ...

