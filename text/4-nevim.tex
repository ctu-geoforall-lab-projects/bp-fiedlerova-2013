\chapter{Zatím nevím}
\label{4-nevim}

\section{GEOS - Geometry Engine, Open Source}
\label{geos}

\textit{GEOS (Geometry Engine - Open Source)} je knihovna implementující 2D 
prostorové predikáty a funkce dle~OGC specifikace \textit{Simple Features} 
pro~SQL. \textit{GEOS} je přepisem knihovny \textit{Java Topology Suite} 
(viz kap. \ref{kap:jts}), do~jazyka C++. Knihovna je projektem OSGeo 
(The Open Source Geospatial Foundation) a je dostupná pod licencí LGPL.

\section{OGR Simple Feature Library}
\label{ogr}

\textit{OGR Simple Feature Library} je \textit{open source} knihovna umožňující
čtení a pořípadě i~zápis vektorových dat různých GIS formátů jako je ESRI 
Shapefile, S-57, SDTS, PostGIS, Oracle Spatial atd. \textit{OGR} je součástí 
knihovny \textit{GDAL}\footnote{Geospatial Data Abstraction Library - knihovna
umožňující čtení a zápis rastrových dat.}.

\section{Quantum GIS}
\label{qgis}

\textit{Quantum GIS} je \textit{open source} multiplatformní geografický 
informační nástroj, který patří mezi~oficiální projekty \textit{Open Source 
Geospatial Foundation} (OSGeo)\footnote{http://www.osgeo.org/}. QGIS je 
napsán v~jazyce C++, je publikován pod~licencí \textit{GNU General Public 
License}\footnote{http://www.gnu.org/licenses/gpl.html}. Program slouží 
ke~zpracování a~analýze vektorových i~rastrových dat. Kromě základní 
funkcionality samotné aplikace, existuje ještě velké množství volně 
dostupných zásuvných modulů, které umožňují poměrně široké využití 
tohoto nástroje. Tyto moduly jsou psané v~jazycích C++ a~Python.
 %% ML: kostrbate...

Vývoj projektu QGIS začal v~roce 2002. Původní myšlenka vzešla z~potřeby 
prohlížeče GIS dat pro operační systém GNU/Linux, který by podporoval většinu 
existujících formátů. To vedlo k~vytvoření \textit{Quantum GIS} projektu. Úplně 
první verze podporovala pouze \textit{PostGIS}\footnote{Prostorová nadstavba 
databázového systému \textit{PostgreSQL}.} vrstvy, poměrně rychle však byla 
doplněna podpora i~dalších datových formátů a~QGIS se z~pouhého prohlížecího 
nástroje stal plnohodnotným geografickým informačním systémem.


\subsection{Zásuvné moduly}
\label{qgis plugins}
...
