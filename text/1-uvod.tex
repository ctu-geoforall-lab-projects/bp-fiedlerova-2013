\chapter{Úvod}
\label{1-uvod}

Tato bakalářská práce se zabývá tématem slučování vektorových map
(\textit{vector - to - vector conflation}).

Spojování geografických dat z~různých zdrojů je jedním z~aktuálních
problémů v~oblasti geografických informačních systémů \zk{GIS}. 
S~rostoucí dostupností digitálních dat začala vzrůstat i~potřeba 
tato data kombinovat a~porovnávat. Na~internetu je volně k~dispozici 
spousta různých prostorových datasetů. Jejich přesnost, úplnost, 
atributy a~další vlastnosti jsou však velmi odlišné. Některé jsou 
geometricky přesné, ale neobsahují potřebné atributy nebo
některé žádoucí detaily. Méně přesné mapy s~potřebnými informacemi
lze upravit pomocí těch přesnějších, abychom získali požadovaný
výsledek. To a~veškeré podobné úlohy, které nás napadnou, jsou 
předmětem slučování map.

Pod anglickým pojmem \textit{conflation} se skrývá mnoho různých
procesů, úloh a~činností. Jejich cílem může být obohacení jedné 
mapy o~atributy či prvky z~mapy druhé, aktualizace datasetu, kombinace 
dvou map o~stejné přesnosti, detekce změn a jiné. Záměrem je vždy 
vytvoření nové mapy, z~níž lze vyčíst informace, které by nebylo možné
získat pouze z~jednoho vstupního zdroje.

Vzhledem k~velkému množství datasetů a~jejich rozsahu je automatizace
výše uvedených úloh logickým a~nevyhnutelným krokem. Ne vždy je však
automatické zpracování zárukou úspěchu. Každý algoritmus má své limity
a~nikdy nelze identifikovat $100\%$ odpovídajících si prvků. To může
být ještě sníženo nejednoznačností a~kvalitou vstupních dat. Proto
je stále třeba dodatečná ruční úprava a~kontrola dat. Ta je však
oproti úspoře času díky automatickému zpracování často zanedbatelná.

Výstupem práce má být zásuvný modul, který umožní spojení
vektorových map. Hlavním cílem je však vyhledání způsobů
řešení tohoto problému a~implementace některých z~nich.
Předpokládá se, že vznikne knihovna s~algoritmy pro slučování
vektorových dat, kterou bude možné dále rozšiřovat a~doplňovat
další funkcionalitu. Zásuvný modul bude využívat tuto knihovnu
a~s~jejím rozšiřováním bude možné i~sem přidávat další funkce.   

V~následující kapitole bude podrobněji popsána problematika 
slučování map (\textit{conflation}). V~kapitole \ref{3-nastroje}
pak budou představeny některé existující nástroje v~této oblasti.
Po~krátkém výčtu použitých technologií (kapitola \ref{4-technologie})
už se bude práce zabývat samotnou tvorbou knihovny a~zásuvného
modulu. Kapitola \ref{5-geoc} bude popisovat jednotlivé algoritmy
nově vznikající knihovny a~jejich implementaci. Následovat
bude popis zásuvného modulu a~jeho funkcionality včetně
ukázek jeho použití (kapitola \ref{6-plugin}). Poslední dvě kapitoly
budou věnovány řešení problémů a~nápady na~další vývoj 
a~vylepšení vzniklé aplikace.  