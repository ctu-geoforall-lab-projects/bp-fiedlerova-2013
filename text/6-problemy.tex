\chapter{Problémy a jejich řešení}
\label{6-problemy}

\section{Použití knihovny GEOS}
\label{geos problem}

Jako poměrně zásadní problém se ukázalo použití knihovny \textit{GEOS}. Přestože tato knihovna je napsána v~jazyku C++, obsahuje \textit{wrapper} v~podobě C~api\footnote{api
(application program interface) - rozhraní pro programování aplikací}. Hlavním důvodem toho je zajištění vyšší stability. Při~použití C~api totiž není nutné vlastní aplikaci,
která ho používá, při~každé změně v~knihovně \textit{GEOS} překompilovat. Proto většina programů využívajících \textit{GEOS} a~mezi nimi i~Quantum GIS používají právě C~api. 
Nicméně to má i~své nevýhody. V~C~api bohužel zatím není implementovánanveškerá funkcionalita C++~api. Některé chybějící funkce lze obejít aplikací sice složitějšího postupu,
nicméně výsledek je obdobný jako při~použití C++~api. Avšak jiná funkcionalita je pro~uživatele C~api zcela nedostupná. Po~dlouhých úvahách bylo proto v~knihovně \textit{GEOC}
přistoupeno k~využití C++~api i~s~rizikem, že bude při~každé změně nutné vše překompilovat.  

S~tím souvisí i~další problém převodu geometrie. Jelikož v~programu QGIS je používáno právě C~api, nelze GEOS geometrii jednoduše převést na~typ QGIS geometrie. Tento problém
byl vyřešen konverzí mezi těmito dvěma typy přes WKT\footnote{well known text - textový značkovací jazyk pro popis vektorové geometrie geografických objektů apod.}. Nevýhoda
tohoto postupu je však zpomalení běhu programu obzvláště u~rozsáhlejších datasetů.   

\section{Rychlost zpracování}
\label{rychlost}

Základem použitých algoritmů je vždy nalezení blízkých či podobných prvků ze~dvou vrstev. Pro~nalezení takových dvojic prvků je obecně třeba porovnat každý prvek z~jedné
vrstvy s~každým prvkem z~vrstvy druhé. Je tedy třeba porovnat $n\cdot m$ dvojic, kde $n$ je počet prvků vrstvy první a~$m$ vrstvy druhé. U~rozsáhlejších datasetů je tedy
nutné provést poměrně velké množství operací a~čas zpracování pak výrazně roste. Tento problém byl vyřešen využitím prostorových indexů. Díky tomu se porovnávají pouze
dvojice prvků, u~nichž to má smysl, tedy takové, jejichž minimální ohraničující obdélníky (v~některých algoritmech rozšířené o~toleranční vzdálenost) se protínají.