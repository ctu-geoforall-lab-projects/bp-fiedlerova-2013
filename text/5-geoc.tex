\chapter{GEOC}
\label{5-geoc}

\section{Tvorba knihovny GEOC}
\label{knihovna}


\section{Použité algoritmy}
\label{algoritmy}

\subsection{ Přichycení blízkých vrcholů \textit{Vertex Snapper} }
\label{vertexsnapper}

Nejjednodušším způsobem kombinace dvou vektorových vrstev je pouhé přichycení blízkých vrcholů cílové vrstvy k~vrstvě referenční. Postup je následující:

\begin{enumerate}
 \item Na~počátku je třeba určit vzdálenostní toleranci, tedy maximální vzdálenost mezi~dvěma body, kdy ještě bude provedeno jejich přichycení.
 \item Ke~každému prvku ze~zpracovávané vrstvy nalezneme nejbližší prvky z~vrstvy referenční. To jsou prvky, jejichž nejkratší vzdálenost od~zpracovávaného prvku není větší
	než vzdálenostní tolerance.
 \item Pro~každý bod ze~zpracovávaného prvku vypočteme vzdálenosti ke~všem bodům z~blízkých referenčních prvků.
 \item Pokud nejmenší z~těchto délek je menší než vzdálenostní tolerance, pak posunume zpracovávaný bod do~odpovídajícího referenčního bodu s touto nejmenší vzdáleností.
 \item Takto projdeme postupně všechny vrcholy všech prvků cílové vrstvy a~snažíme se k nim nalézt blízké body z prvků vrstvy referenční. 
\end{enumerate}

% Samozřejmě je vhodné kontrolovat, zda se dva body nepřichytí na jeden (otázkou je, zda je to správně či ne)
 

% nějaký obrázek 

\subsection{ Komplexnější algoritmus }
\label{conflation algorithm}

Nejčastější používaný postup při~spojování vektorových map je následující:

\begin{enumerate}
 \item Nejprve je třeba nalézt odpovídající si prvky v~obou překrývajících se vrstvách. Kritéria pro~určení odpovídajících si prvků mohou být velmi odlišná. Existuje mnoho
	algoritmů řešících tuto problematiku, přičemž postupy se mohou různit podle toho, zda je úkolem vyhledání odpovídajících si bodů, polygonů či linií. Kritéria a~postup
	použitý v~knihovně \textit{GEOC} je popsán níže.
 \item Poté, co se určí odpovídající si prvky, musí se určit totožné vrcholy těchto dvojic prvků. Ty z~vrcholů, které jsou určeny s~dostatečnou přesností (ta může být určena
	například danou vzdálenostní tolerancí), jsou označeny jako body budoucí triagulační sítě.
 \item Jak už bylo naznačeno, z~nalezených bodů se vytvoří pomocí Delaunayho triangulace trojúhelníková síť.
 \item Následně se provede lokální nejčastěji afinní transformace v~každém trojúhelníku sítě. Tak se přetransformují body cílové vrstvy do~systému vrstvy referenční.
 \item Celý postup je možné iterativně opakovat, dokud nedosáhneme požadovaného výsledku (ten může být dán např. podmínkou minimálního množství nalezených 
	odpovídajících si vrcholů).
\end{enumerate}

V~knihovně \textit{GEOC} je pro nalezení odpovídajících si prvků využito obdobného postupu jako ve~výše zmiňované knihovně \textit{JCS}. Využívá se vrcholová Hausdorffova 
vzdálenost, přičemž tato vzdálenost není počítána přímo. Splnění podmínky, že dané prvky nejsou od~sebe dále, než je daná Hausdorffova vzdálenost, se testuje pomocí obalových
zón jednotlivých prvků následujícím způsobem.

\begin{enumerate}
 \item Máme dva prvky A a~B ze~dvou různých překrývajících se vrstev.
 \item Pokud prvek B leží v~obalové zóně prvku A o~velikosti vzdálenostní tolerance a~A leží v~obalové zóně prvku B o~stejné velikosti, je možné, že si prvky odpovídají 
	a~pokračuje se dalším krokem. V~opačném případě si prvky neodpovídají.
 \item Dále se testuje, zda hranice prvku B leží v~obalové zóně hranice prvku A a~naopak. Je-li splněna i~tato podmínka, pak jsou prvky označeny za~odpovídající si.
\end{enumerate}


% obrázek

