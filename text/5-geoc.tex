\chapter{Knihovna GEOC}
\label{5-geoc}

Algoritmy týkající se slučování vektorových map (\textit{conflation}) byly
implementovány v~externí knihovně \zk{GEOC} bez závislosti na~QGIS API. 
Zásuvný modul \textbf{Conflate} zprostředkovává funkcionalitu knihovny 
v~prostředí Quantum GIS. Díky tomuto oddělení vznikla nezávislá knihovna, 
kterou bude možné případně použít i~v~jiných programech a~projektech.

Tato kapitola popisuje jednotlivé algoritmy a~jejich implementaci v~knihovně
\zk{GEOC}, zároveň také stručně pojednává o~možnostech  jejich využití.
Podrobný popis jednotlivých tříd a~jejich metod lze najít
v~dokumentaci na~přiloženém CD~(viz příloha~\ref{priloha-obsahCD}). 


\section{Reprezentace geometrie}
\label{reprezentace-geom}

Pro reprezentaci geometrie a~prostorové analýzy \zk{GEOC} využívá
knihovnu \zk{GEOS}. Geometrie prvků je představována třídou 
\texttt{geos::geom::Geometry}. Kromě samotné geometrie je v~algoritmech
často potřeba pracovat se~seznamem bodů jednoho či více prvků. V~těchto
případech je využita třída \texttt{geos::geom::CoordinateSequence}
a~\texttt{geos::geom::Coordinate}, která pak reprezentuje samostatný bod.


\section{Vertex Snapper} 
\label{vertexsnapper}

Při zpracování vrstev z~více zdrojů někdy stačí pouze upřesnit polohu či tvar 
prvků z~cílové vrstvy tak, aby se přiblížil prvkům z~vrstvy referenční. 
\mbox{Není-li} podrobnost obou datasetů příliš rozdílná, lze využít jednoduchého 
postupu \textbf{přichycení blízkých vrcholů}. Tento algoritmus byl inspirován
stejnojmenným zásuvným mo\-dulem knihovny \zk{JCS} (viz kapitola~\ref{jcs-plugin}).


\subsection{Popis algoritmu}
\label{vs-algoritmus}

Nejjednodušším způsobem kombinace dvou vektorových vrstev je pouhé přichycení 
blíz\-kých vrcholů cílové vrstvy k~vrstvě referenční. Algoritmus lze obecně 
popsat následují\-cími několika body.

\begin{enumerate}
 \item Na~počátku je třeba určit vzdálenostní toleranci, tedy maximální 
    vzdálenost mezi~dvě\-ma body, kdy ještě bude provedeno jejich přichycení.
 \item Ke~každému prvku ze~zpracovávané vrstvy se najdou nejbližší prvky 
    z~vrstvy refe\-ren\-ční. To jsou prvky, jejichž nejkratší vzdálenost 
    od~zpracovávaného prvku není větší než vzdálenostní tolerance.
 \item Pro~každý bod ze~zpracovávaného prvku jsou vypočteny vzdálenosti 
    ke~všem bodům z~blízkých referenčních prvků.
 \item Pokud nejmenší z~těchto délek je menší než vzdálenostní tolerance, 
    pak je zpracová\-va\-ný bod posunut do~odpovídajícího referenčního bodu 
    s~touto nejmenší vzdáleností.
 \item Takto se prochází postupně všechny vrcholy všech prvků cílové vrstvy 
    a~vyhledávají se k~nim blízké body z~prvků vrstvy referenční. 
\end{enumerate}

  \begin{figure}[H]
    \centering
      \input{./pictures/vs-princip.pdf_tex}
      \caption{Postup přichycení vrcholů}
      \label{fig:vs-princip}
  \end{figure}


\subsection{Implementace}
\label{vs-implementace}
Algoritmus pro~přichycení vrcholů upravované vrstvy k~referenční 
je implementován ve~tří\-dách \texttt{Vertex\-Snapper} 
a~\texttt{Vertex\-Geometry\-Editor\-Operation}. Lepší představu 
o~jednotlivých tří\-dách lze získat z~\zk{UML} diagramu  tříd
v~příloze~\ref{priloha-uml}. Při~po\-užití v~externí aplikaci 
stačí po~pře\-dání nezbytných vstupních parametrů (k~tomu 
slouží metody \texttt{setRefGeometry()}, \texttt{setSubGeometry()} 
a~\texttt{setTolDistance()}) třídě \texttt{Vertex\-Snapper} zavolat 
funkci \texttt{snap()}. Ta vyhledá blízké prvky s~využitím prostorových 
indexů sesta\-vených metodou \texttt{build\-Index()} třídy 
\texttt{Spatial\-Index\-Builder}.

Výsledky vyhledávání poté předá funkci \texttt{snap\-Vertices()}. 
Uvnitř této metody se vytvoří instance  třídy 
\texttt{Vertex\-Geometry\-Editor\-Operation}, která edituje příslušnou 
geo\-metrii přichycením blízkých vrcholů v~metodě \texttt{edit()}. 
\texttt{Vertex\-Geometry\-Editor\-Operation} je potomkem třídy 
\texttt{geos::\-operation::\-Coordinate\-Operation}, která je 
\textit{inter\-face}\footnote{Rozhraní třídy, kde jsou deklarovány 
pouze abstraktní metody bez implementace, není možné vytvořit 
instanci této třídy.} třídou pro editaci geometrie.

Pro kontrolu validity geometrií je volána metoda \texttt{isValid()} 
třídy \texttt{geos::geom::\linebreak Geometry}.

Výsledné geometrie lze získat zavoláním metody \texttt{getNewGeometry()},
která vrací vektor s~interní reprezentací geometrií.



\subsection{Využití}
\label{vs-vyuziti}

Přichycení bodů jedné vrstvy k~vrstvě druhé má své výhody i~nevýhody, které 
je před volbou tohoto způsobu zpracování třeba zvážit. Použití této metody 
je vhodné v~takových případech, kdy jsou k~dispozici dvě vrstvy o~rozdílné 
přesnosti (tento rozdíl však nesmí být příliš veliký) a~prvky vzájemně se 
překrývající. Cílem je upřesnit polohu a~tvar prvků jednoho datasetu. 
Dopředu je třeba si uvědomit, že kromě polohy prvků je měněn i~jejich tvar.

Využít by tento postup šel i~pro~přichycení dvou sousedních vrstev o~stejné 
přesnosti, avšak to znamená, že by se změnil pouze tvar krajních prvků 
(nedošlo by k~posunu celé vrstvy), a~to nemusí být vždy žádoucí. Jako jediný
z~algoritmů \zkratka{GEOC} má pak smysl pro~bodové vrstvy.

Pro~rozumné výsledky je důležité zvolit vhodnou toleranční vzdálenost. Tato 
hodnota by měla odpovídat maximální vzdálenosti, o~kterou se vrchol prvku 
může posunout. Při~volbě příliš krátké vzdálenosti se výsledná vrstva nemusí 
vůbec odlišovat od té vstupní. Naopak \mbox{je-li} zvolená vzdálenost delší 
než nejkratší úsek geometrie (linie, polygonu), může dojít k~přichycení dvou 
bodů k~jednomu bodu z~referenční vrstvy. Zda je toto přípustné či nikoli, je 
už na~rozhodnutí uživatele.


\section{Coverage Alignment} 
\label{coverage alignment}

\textit{Coverage alignment} lze vysvětlit jako \textbf{zarovnání jedné vrstvy 
k~vrstvě druhé}. Tento způsob je složitější než výše uvedené přichytávání vrcholů.
V~knihovně \zk{GEOC} je využit opět pro~úpravu jedné vrstvy na~základě vrstvy 
referenční. Do~upravované vrstvy nejsou žádné prvky přidávány ani z~ní mazány,
jde pouze o~jejich modifikaci. Velmi podobný algoritmus se dá použít 
i~ke~kombinaci dvou vrstev.

\subsection{Popis algoritmu}
\label{ca-algoritmus}

Nejčastější používaný postup při~spojování vektorových map je následující.

\begin{enumerate}
 \item Nejprve je třeba nalézt odpovídající si prvky v~obou překrývajících se 
    vrstvách. Kritéria pro~určení odpovídajících si prvků mohou být velmi 
    odlišná. Existuje mnoho algoritmů řešících tuto problematiku, přičemž 
    postupy se mohou různit podle toho, zda je úkolem vyhledání 
    odpovídajících si bodů, polygonů či linií. Kritéria a~postup použitý 
    v~knihovně \zk{GEOC} je popsán níže.
 \item Poté, co se určí odpovídající si prvky, musí se nalézt totožné vrcholy 
    těchto dvojic prvků. Ty z~vrcholů, které jsou určeny s~dostatečnou 
    přesností (ta může být určena například danou vzdálenostní tolerancí), 
    jsou označeny jako body budoucí triangulační sítě.
 \item Jak už bylo naznačeno, z~nalezených bodů se vytvoří pomocí 
    \zkratkatext{DT}\footnote{Delaunayho triangulace z~množiny bodů v~rovině 
    vytvoří takovou trojúhelníkovou síť, pro kterou platí, že v~kružnici 
    opsané každému trojúhelníku, neleží žádný jiný bod. \zk{DT} maximalizuje
    minimální úhly trojúhelníků.} trojúhelníková síť. 
 \item Následně se provede lokální, nejčastěji afinní transformace v~každém 
    trojúhelníku sítě. Tak se přetransformují body cílové vrstvy do~systému 
    vrstvy referenční.
 \item Celý postup je možné iterativně opakovat, dokud není dosaženo 
    požadovaného výsledku (ten může být dán např. podmínkou minimálního 
    množství nově nale\-zených odpovídajících si vrcholů či prvků).
\end{enumerate}

  \begin{figure}[ht]
    \centering
      \input{./pictures/ca-princip.pdf_tex}
      \caption{Postup zarovnání vrstev}
      \label{fig:ca-princip}
  \end{figure}

V~knihovně \zk{GEOC} je pro nalezení odpovídajících si prvků využito 
obdobného postupu jako ve~výše zmiňované knihovně \zk{JCS}. Využívá 
se vrcholová Hausdorffova vzdálenost, přičemž tato vzdálenost není počítána 
přímo. Splnění podmínky, že dané prvky nejsou od~sebe dále, než je daná 
Hausdorffova vzdálenost, se testuje pomocí obalových zón jednotlivých prvků 
následujícím způsobem.

\begin{enumerate}
 \item Mějme dva prvky A a~B ze~dvou různých překrývajících se vrstev.
 \item Pokud prvek B leží v~obalové zóně prvku A o~velikosti vzdálenostní 
    tolerance a~A leží v~obalové zóně prvku B o~stejné velikosti, je možné, 
    že si prvky odpovídají, a~pokračuje se dalším krokem. Situace je naznačena
    na~obrázku~\ref{fig:buffer}. V~opačném případě si prvky neodpovídají.
 \item Dále se testuje, zda hranice prvku B leží v~obalové zóně hranice prvku
    A a~naopak (viz obrázek~\ref{fig:buffer-boundary}). Je-li splněna i~tato 
    podmínka, pak jsou prvky označeny za~odpovídající.
\end{enumerate}

  \begin{figure}[H]
    \centering
      \small
      \def\svgwidth{380pt}
      \input{./pictures/buffer-test.pdf_tex}
      \caption{Porovnání prvků na základě jejich obalových zón}
      \label{fig:buffer}
  \end{figure}

  \begin{figure}[H]
    \centering
      \small
      \def\svgwidth{390pt}
      \input{./pictures/buffer-boundary.pdf_tex}
      \caption{Porovnání prvků na základě obalových zón jejich hranic}
      \label{fig:buffer-boundary}
  \end{figure}

\subsection{Implementace}
\label{ca-implementace}
Algoritmus je opět implementován pod jedinou metodou \texttt{align()} 
třídy \texttt{Coverage\-Align\-ment}, kterou je možné zavolat po~nastavení
základních parametrů. Ta po\-stupně volá metody provádějící jednotlivé
kroky algoritmu.

Nejdříve je třeba nalézt odpovídající si prvky. K~tomu slouží třída 
\texttt{Matching\-Geometry}, která k~dané geometrii najde odpovídající
(metoda \texttt{set\-Match()}). Obdobně jako u~\texttt{Vertex\-Snapper} je i~zde 
při hledání blízkých prvků využíváno prostoro\-vých indexů, které jsou 
vytvořeny metodou \texttt{SpatialndexBuilder::build\-Index()}. Testy
obalových zón jsou pro\-vá\-děny za~pomoci funkcí 
\texttt{buffer()}, která vytvoří obalovou zónu, \texttt{getBoundary()},
která vrací hranici geometrie, a~konečně \texttt{contains()}, která
testuje, zda jedna geometrie obsahuje druhou. Všechny jsou přitom
metodami třídy \texttt{geos::geom::Geometry}.

Určení blízkých bodů je provedeno prostřednictvím metody 
\texttt{choose\-Matching\-Points()}, která dále využívá
\texttt{find\-Closest\-Points()} a~\texttt{clean\-Matching\-Points()} 
třídy \texttt{Coverage\-Alignment}.

Třetím krokem je vytvoření trojúhelníkové sítě metodou
\texttt{create\-TIN()}, která k~tomu využívá třídu 
\texttt{Tri\-an\-gu\-la\-tion}.  Ta pouze aplikuje třídu 
\texttt{geos::triangu\-late::Delaunay\-Triangulation\-Builder}, kde je 
vytvářen \zk{TIN} ze~zadaných bodů.

Konečně je provedena postupně transformace všech prvků. Funkce pro
transformaci poskytuje třída \texttt{Affine\-Trans\-for\-mation},
která transformuje prvky na~základě předané geometrie a~triangulační
sítě. Ta je volána prostřednictvím třídy pro editaci 
\texttt{Align\-Geo\-metry\-Edi\-tor\-Ope\-ra\-tion}.

Propojení jednotlivých tříd a~nejdůležitější metody zobrazuje \zk{UML}
diagram tříd  pro tento algoritmus v~příloze~\ref{priloha-uml}.


\subsection{Využití}
\label{ca-vyuziti}

Na~rozdíl od~předchozího algoritmu je tento trochu šířeji využitelný. Je opět 
vhodný pro~zpřesnění vrstvy dle vrstvy referenční, avšak tentokrát nejsou 
pouze přichytá\-vány blízké vrcholy, ale jsou upravovány téměř všechny vrcholy. 
To zajišťuje reálnější výsledky i~v situacích, kdy hustota vrcholů v~obou 
datasetech je velmi rozdílná.


\section{LineMatcher}
\label{line matcher}

Častým předmětem spojování vektorových dat jsou liniové mapy silnic, cest apod.
Proto byl do~knihovny doplněn i~algoritmus \textit{Line Matcher}, který je
určen speciálně pro liniové vrstvy. Výrazný rozdíl oproti předchozím algoritmům
je také fakt, že tento nezarovnává jednu vrstvu ke~druhé, ale výsledkem je
průměr z~odpovídajících si prvků obou vrstev. Nezáleží tedy na~tom, která 
z~vrstev je referenční. Prvky se zde nemyslí celé linie, ale pouze jejich
segmenty.

\subsection{Popis algoritmu}
\label{lm-algoritmus}

Jak je uvedeno výše, tento algoritmus pracuje pouze s~úseky jednotlivých
linií, a~to z~toho důvodu, že jedna dlouhá linie může být v~jiném datasetu
rozdělena na~několik menších například přerušením na~křižovatkách. Pro každou
linii jednoho datasetu je pak provedeno následující. 

\begin{enumerate}
 \item Ke~každému segmentu linie jsou nalezeny blízké linie z~druhého datasetu.
       Stejně jako u~předchozích, i~zde je blízkost definována zvolenou toleranční
       vzdáleností.
 \item Testovaný segment je dále porovnáván s~každým úsekem z~těchto blízkých
       linií na~zá\-kladě kritérií \ref{crit} uvedených níže.
 \item Pokud je splněna předem zvolená hodnota podobnosti segmentů pro všechna 
       kritéria $S_1,S_2,S_3$, je výsledná podobnost určena hodnotou $S$.
 \item Z~odpovídajících segmentů je nalezen takový, jehož podobnost s~testovaným
       je největší (hodnota $S$ je co nejblíže jedné).
 \item Výsledný segment je průměrem ze~dvou nejlépe si odpovídajících segmentů.
\end{enumerate}

\subsubsection*{Kritéria pro určení odpovídajících si liniových segmentů}
První dvě kritéria (viz \cite{moosavi}) nejlépe určují podobnost dvou 
linií. Třetí kritérium bylo přidáno pro praktické použití tak, aby se neurčovaly
pouze podobné prvky, ale i~prvky blízké. Jednotlivá kritéria určující podobnost 
mohou nabývat hodnot od~$0$ do~$1$, kde $1$ znamená naprostou shodu.

\begin{equation}
 \begin{aligned}
 S_1 = \frac{ \Delta l_{max} - \Delta l }{ \Delta l_{max} - \Delta l_{min}},\ 
 S_2 &= \frac{ \alpha_{max} - \alpha }{ \alpha_{max} - \alpha_{min}},\ 
 S_3 = \frac{ d_{max} - d }{ d_{max} - d_{min}}\\
 S &= \frac{S_1+S_2+S_3}{3}
 \label{crit}
 \end{aligned}
\end{equation}

Hodnoty $\Delta l_{max}$, $\Delta l_{min}$ udávají maximální a~minimální rozdíl délek
segmentů v~celém datasetu, $\Delta l$ je pak rozdíl délek porovnávaných segmentů.
$\alpha$ udává odchylku dvou segmentů. Hodnota $\alpha_{max}$ ve většině případů může
být zvolena $90^{\circ}$, $\alpha_{min}$ pak $0^\circ$. Proměnná $d$ ve~třetím 
kritériu reprezentuje vzdálenost segmentů, která je spočtena jako průměr vzdáleností
jejich počátečních a~koncových bodů. Jelikož se porovnávají pouze blízké segmenty,
$d_{max}$ je rovno toleranční vzdálenosti, $d_{min}$ by pak mělo odpovídat nejmenší
vzdálenosti krajních bodů dvojice segmentů z~obou datasetů.


\subsection{Implementace}
\label{lm-implementace}

Veškeré funkce pro tento algoritmus jsou obsaženy ve~třídě \texttt{Line\-Matcher}
(blíže viz pří\-loha~\ref{priloha-uml}, obrázek~\ref{fig:uml-lm}).

Funkce \texttt{match()} hledá postupně ke~každé linii blízké linie z~druhého datasetu.
Následně předá výsledky tohoto hledání metodě \texttt{match\-Line()}, která ke~každému
úseku linie nalezne prostřednictvím metody \texttt{matching\-Segment()} odpovídající
segment mezi předanými blíz\-kými prvky. Pokud k~nějakému úseku linie není určen
žádný dostatečně podobný, je zde linie přerušena. Z~původního jednoduchého liniového
prvku tak může vzniknout multiprvek. Pro vytváření nových geometrií jsou využívány
metody \texttt{createLineString()} a~\texttt{createMultiLineString()} třídy 
\texttt{geos::geom::GeometryFactory} knihovny \zk{GEOS}.
 
Výpočet kritérií podobnosti segmentů se nachází v~metodě \texttt{similarity()}. 

\subsection{Využití}
\label{lm-vyuziti}

\textit{Line Matcher} lze využít pro spojování liniových vrstev. Map různých cest, 
silnic, tras atd. existuje velké množství i~díky rozvoji technologií 
\zk{GNSS}\footnote{Data tras z~\zk{GNSS} je nutné před použitím upravit, nejlépe 
zjednodušením linií, aby délka úseků přibližně odpovídala druhé vrstvě.}. Právě
k~jejich slučování slou\-ží \textit{Line Matcher}. Hodí se i~pro vyhledání 
společných prvků obsažených ve~dvou datasetech. Příkladem takové aplikace je 
určení cyklotras vedoucích po~silnici z~vrstvy silnic a~cyklotras.

Vzhledem k~požadavku co nejobecnějšího využití je výsledkem tohoto algoritmu
průměr z~odpovídajících si prvků.  Někdy by však bylo vhodnější nechat prvkům
původní polohu, pouze vymazat ty nepasující. Jindy může být požadavkem také
ponechání i neodpovídajících si prvků apod. Různýmí drobnými modifikacemi
by se tedy dalo dosáhnout lepšího využití pro nějaký konkrétní případ.
