\chapter{Použité technologie}
\label{4-technologie}

Tato kapitola si klade za~cíl seznámit čtenáře s~technologiemi, které byly 
při tvorbě zásuvného modulu a~externí knihovny využity. Uvedeny jsou pouze
základní informace o~jednotlivých projektech, více podrobností lze
nalézt na~uvedených oficiálních internetových stránkách projektů.


\section{Quantum GIS}
\label{qgis}

  \begin{figure}[H]
    \centering
      \includegraphics[width=120pt]{./pictures/qgis.png}
      \caption[QGIS logo]{QGIS logo 
      (zdroj: \url{http://www.qgis.org/wiki/File:QGis_Logo.png})}
      \label{fig:qgis}
  \end{figure}

\zkratka{QGIS} je \textit{open source} multiplatformní geografický 
informační ná\-stroj, který patří mezi~oficiální projekty \zk{OSGeo}.
\zkratka{QGIS} je napsán v~jazyce C++, je pu\-blikován pod~licencí 
\zkratka{GPL}\footnote{\url{http://www.gnu.org/licenses/gpl.html}}. 
Program slouží ke~zpracování a~analýze vektorových i~rastrových dat. 
Základní funkcionalitu sa\-motné aplikace rozšiřuje velké množství 
volně dostupných zásuvných modulů. Tyto moduly jsou psané v~jazycích 
C++ nebo Python.

Vývoj projektu \zk{QGIS} začal v~roce 2002. Původní myšlenka vzešla 
z~potřeby prohlížeče \zk{GIS} dat pro operační systém GNU/Linux, který 
by podporoval většinu existujících formátů. To vedlo k~vytvoření 
\textit{Quantum GIS} projektu. Úplně první verze podporovala pouze 
\textit{PostGIS}\footnote{Prostorová nadstavba databázového systému 
\textit{PostgreSQL}.} vrstvy, poměrně rychle však byla doplněna podpora 
dalších datových formátů a~\zk{QGIS} se z~pouhého prohlížecího nástroje 
stal plnohodnotným geografickým informačním systémem.


\section{GEOS - Geometry Engine, Open Source}
\label{geos}

\zkratka{GEOS}\footnote{\url{http://trac.osgeo.org/geos/}}  je přepisem 
knihovny \zkratkatext{JTS} (viz kap. \ref{jts}) do~jazyka C++. Záměrem 
projektu je poskytovat kompletní funkcionalitu \zk{JTS} pro~C++. Knihovna 
byla původně vyvíjena firmou Refractions Research of Victoria, Canada. Nyní 
je projektem \zkratka{OSGeo} a~je dostupná pod licencí \zkratka{LGPL}.

Knihovna \zk{GEOS} implementuje třídy pro reprezentaci geometrických
objektů (bod, linie, polygon, vícenásobný bod, geometrická kolekce
- soubor prvků aj.). Umožňuje určovat vztahy mezi prostorovými objekty 
(test, zda se dva objekty protínají, dotýkají, překrývají, jeden obsahuje 
druhý atd.), provádět prostorové operace (výpočet vzdálenosti, plochy 
polygonu, konvexní obálky, obalové zóny, sjednocení atd.). Obsahuje také 
třídy, které dokáží číst a~převádět \zkratka{WKT}\footnote{Textový 
značkovací jazyk pro popis vektorové geometrie geografických objektů 
apod.} a~\zkratka{WKB}\footnote{Binární forma pro popis vektorové 
geometrie geografických objektů apod.} formáty. Stejně jako knihovna 
\zk{JTS} se řídí specifikací \textit{Simple Features} pro \zk{SQL}.
Obsahuje C~i~C++ API.

\section{Qt projekt}
\label{qt}

  \begin{figure}[hbt]
    \centering
      \includegraphics[width=100pt]{./pictures/qt.png}
      \caption[Qt logo]{Qt logo 
	(zdroj: \url{http://qt.digia.com/About-us/Logos-for-Download/})}
      \label{fig:qt}
  \end{figure}

Qt je multiplatformní knihovna pro vývoj aplikací včetně grafického 
uživatelského rozhraní určená především pro vývojáře pracující 
v~jazyce C++, existuje však i~pro některé další jazyky.

Projekt Qt\footnote{\url{http://qt.digia.com/}} byl vyvinut v~roce 1999 
norskou společností Trolltech, v~roce 2008 ho tato firma prodala firmě 
Nokia, až nakonec v~roce 2012 skončil v~rukou společnosti Digia. 
Přes všechny tyto změny však zůstal jedním z~oblíbených nástrojů pro 
vytváření desktopových a~mobilních aplikací s~grafickým rozhraním. 
\textit{Qt} je licencován pod \textit{open source} licencemi \zk{LGPL} 
verze 2.1, \zk{GPL} verze 3.0, ale existuje i~komerční licence pro~vývoj
 proprietárních projektů.

V~rámci projektu nevznikla pouze knihovna, ale i~řada nástrojů usnadňujících
vývoj aplikací. Jedním z nich je i~program \textbf{Qt Creator}. Jedná se 
o~vývojové prostředí~(\zk{IDE}) obsahující mimo jiné i~nástroj pro návrh 
grafického rozhraní\linebreak[10] (\textit{\zk{UI}~designer}) a~nástroje pro 
\textit{debugging} (testování, ladění počítačového programu).
Tento nástroj byl použit pro tvorbu zásuv\-ného modulu i~knihovny, jež jsou
výsledkem této práce.