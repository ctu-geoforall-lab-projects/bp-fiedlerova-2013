\chapter{Použité technologie}
\label{4-technologie}

Tato kapitola si klade za~cíl seznámit čtenáře s~technologiemi, které byly 
při tvorbě zásuvného modulu a~externí knihovny využity. Uvedeny jsou pouze
základní informace o~jednotlivých projektech, více podrobností pak lze
nalézt na~uvedených internetových stránkách.


\section{GEOS - Geometry Engine, Open Source}
\label{geos}

\textit{GEOS (Geometry Engine - Open Source)}
\footnote{\url{http://trac.osgeo.org/geos/}} 
je knihovna implementující 2D prostorové predikáty a funkce dle~OGC 
specifikace \textit{Simple Features} pro~SQL. \textit{GEOS} je přepisem 
knihovny \textit{Java Topology Suite} (viz kap. \ref{kap:jts}), do~jazyka 
C++. Knihovna je projektem OSGeo (The Open Source Geospatial Foundation) 
a~je dostupná pod licencí LGPL.

% \section{OGR Simple Feature Library}
% \label{ogr}
% 
% \textit{OGR Simple Feature Library} je \textit{open source} knihovna umožňující
% čtení a pořípadě i~zápis vektorových dat různých GIS formátů jako je ESRI 
% Shapefile, S-57, SDTS, PostGIS, Oracle Spatial atd. \textit{OGR} je součástí 
% knihovny \textit{GDAL}\footnote{Geospatial Data Abstraction Library - knihovna
% umožňující čtení a zápis rastrových dat.}.

\section{Quantum GIS}
\label{qgis}

\textit{Quantum GIS} je \textit{open source} multiplatformní geografický 
informační nástroj, který patří mezi~oficiální projekty \textit{Open Source 
Geospatial Foundation}\footnote{\url{http://www.osgeo.org/}}. QGIS je 
napsán v~jazyce C++, je publikován pod~licencí \textit{GNU General Public 
License}\footnote{\url{http://www.gnu.org/licenses/gpl.html}}. Program slouží 
ke~zpracování a~analýze vektorových i~rastrových dat. Kromě základní 
funkcionality samotné aplikace, existuje ještě velké množství volně 
dostupných zásuvných modulů, které umožňují poměrně široké využití 
tohoto nástroje. Tyto moduly jsou psané v~jazycích C++ a~Python.
 %% ML: kostrbate... opraveno

Vývoj projektu QGIS začal v~roce 2002. Původní myšlenka vzešla z~potřeby 
prohlížeče GIS dat pro operační systém GNU/Linux, který by podporoval většinu 
existujících formátů. To vedlo k~vytvoření \textit{Quantum GIS} projektu. Úplně 
první verze podporovala pouze \textit{PostGIS}\footnote{Prostorová nadstavba 
databázového systému \textit{PostgreSQL}.} vrstvy, poměrně rychle však byla 
doplněna podpora i~dalších datových formátů a~QGIS se z~pouhého prohlížecího 
nástroje stal plnohodnotným geografickým informačním systémem.


%\subsection{Zásuvné moduly}
%\label{qgis plugins}


\section{QT}
\label{qt}

\textit{Qt}\footnote{\url{http://qt.digia.com/}} je multiplatformní knihovna pro 
vývoj aplikací včetně grafického uživatelského rozhraní určená především 
pro vývojáře pracující v~jazyce C++, existuje však i~pro některé další jazyky.

Projekt \textit{Qt} byl vyvinut v~roce 1999 norskou společností Trolltech, 
v~roce 2008 ho tato firma prodala firmě Nokia, až nakonec v~roce 2012 skončil 
v~rukou společnosti Digia. Přes všechny tyto změny však zůstal jedním z~oblíbených
nástrojů pro vytváření desktopových a~mobilních aplikací s~grafickým rozhraním.
\textit{Qt} je licencován pod \textit{open source} licencí GNU Lesser General 
Public License~(LGPL)\footnote{\url{http://qt-project.org/doc/qt-4.8/lgpl.html}}).

V~rámci projektu nevznikla pouze knihovna, ale i~řada nástrojů usnadňujících
vývoj aplikací. Jedním z nich je i~program \textbf{Qt Creator}. Jedná se o~vývojové 
prostředí~(\textit{IDE}) obsahující mimo jiné i~nástroj pro návrh grafického 
rozhraní (\textit{UI~designer}) a~nástroje pro \textit{debugging}\footnote{debugging
 - testování, ladění počítačového programu}.
