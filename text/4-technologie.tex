\chapter{Použité technologie}
\label{4-technologie}

Tato kapitola si klade za~cíl seznámit čtenáře s~technologiemi, které byly 
při tvorbě zásuvného modulu a~externí knihovny využity. Uvedeny jsou pouze
základní informace o~jednotlivých projektech, více podrobností pak lze
nalézt na~uvedených oficiálních internetových stránkách projektů.


\section{GEOS - Geometry Engine, Open Source}
\label{geos}

\zkratka{GEOS}\footnote{\url{http://trac.osgeo.org/geos/}}  je přepisem 
knihovny \zkratkatext{JTS} (viz kap. \ref{kap:jts}) do~jazyka C++. 
Záměrem je poskytovat kompletní funkcionalitu \zk{JTS} pro~C++. Knihovna byla 
původně vyvíjena firmou Refractions Research of Victoria, Canada. Nyní je
projektem \zkratka{OSGeo}\footnote{\url{http://www.osgeo.org/}}  a~je 
dostupná pod licencí \zkratka{LGPL}.

Knihovna \zk{GEOS} implementuje třídy pro reprezentaci geometrických
objektů (\texttt{Point, LineString, Polygon, MultiPoint, MultiLineString,
MultiPolygon} a \texttt{GeometryCollection}). Umožňuje zjišťovat vztahy 
mezi prostorovými objekty (test, zda se dva objekty protínají, dotýkají, 
překrývají, jeden obsahuje druhý atd.), provádět prostorové operace 
(výpočet vzdálenosti, plochy polygonu, konvexní obálky, obalové zóny, 
sjednocení atd.). Obsahuje také třídy, které dokáží číst a~převádět
\zkratka{WKT}\footnote{textový značkovací jazyk pro popis vektorové 
geometrie geografických objektů apod.} a~\zkratka{WKB}\footnote{binární
forma pro popis vektorové geometrie geografických objektů apod.} formáty. 
Stejně jako knihovna \zk{JTS} se řídí specifikací \textit{Simple Features}
pro \zk{SQL}. Obsahuje C~i~C++ API.

% \texttt{Intersects, Touches, Disjoint, Crosses, Within, Contains,
% Overlaps, Equals, Covers}
% \texttt{Union, Distance,
% Intersection, Symmetric Difference, Convex Hull, Envelope, Buffer, 
% Simplify, Polygon Assembly, Valid, Area, Length}

% \section{OGR Simple Feature Library}
% \label{ogr}
% 
% \textit{OGR Simple Feature Library} je \textit{open source} knihovna umožňující
% čtení a pořípadě i~zápis vektorových dat různých GIS formátů jako je ESRI 
% Shapefile, S-57, SDTS, PostGIS, Oracle Spatial atd. \textit{OGR} je součástí 
% knihovny \textit{GDAL}\footnote{Geospatial Data Abstraction Library - knihovna
% umožňující čtení a zápis rastrových dat.}.

\section{Quantum GIS}
\label{qgis}

\zkratka{QGIS} je \textit{open source} multiplatformní geografický 
informační nástroj, který patří mezi~oficiální projekty \zk{OSGeo}.
\zkratka{QGIS} je napsán v~jazyce C++, je publikován pod~licencí 
\zkratka{GPL}\footnote{\url{http://www.gnu.org/licenses/gpl.html}}. 
Program slouží ke~zpracování a~analýze vektorových i~rastrových dat. 
Kromě základní funkcionality samotné aplikace, existuje ještě velké 
množství volně dostupných zásuvných modulů, které umožňují poměrně 
široké využití tohoto nástroje. Tyto moduly jsou psané v~jazycích 
C++ a~Python.
 %% ML: kostrbate... opraveno

Vývoj projektu \zk{QGIS} začal v~roce 2002. Původní myšlenka vzešla z~potřeby 
prohlížeče \zk{GIS} dat pro operační systém GNU/Linux, který by podporoval většinu 
existujících formátů. To vedlo k~vytvoření \textit{Quantum GIS} projektu. Úplně 
první verze podporovala pouze \textit{PostGIS}\footnote{Prostorová nadstavba 
databázového systému \textit{PostgreSQL}.} vrstvy, poměrně rychle však byla 
doplněna podpora i~dalších datových formátů a~\zk{QGIS} se z~pouhého prohlížecího 
nástroje stal plnohodnotným geografickým informačním systémem.


\section{Qt projekt}
\label{qt}

Qt\footnote{\url{http://qt.digia.com/}} je multiplatformní knihovna pro 
vývoj aplikací včetně grafického uživatelského rozhraní určená především 
pro vývojáře pracující v~jazyce C++, existuje však i~pro některé další jazyky.

Projekt Qt byl vyvinut v~roce 1999 norskou společností Trolltech, v~roce 
2008 ho tato firma prodala firmě Nokia, až nakonec v~roce 2012 skončil v~rukou
společnosti Digia. Přes všechny tyto změny však zůstal jedním z~oblíbených
nástrojů pro vytváření desktopových a~mobilních aplikací s~grafickým rozhraním.
\textit{Qt} je licencován pod \textit{open source} licencemi \zk{LGPL} verze 2.1,
\zk{GPL} verze 3.0, ale existuje i~komerční licence pro~vývoj proprietárních 
projektů\footnote{\url{http://qt-project.org/doc/qt-5.0/qtdoc/licensing.html}}.

V~rámci projektu nevznikla pouze knihovna, ale i~řada nástrojů usnadňujících
vývoj aplikací. Jedním z nich je i~program \textbf{Qt Creator}. Jedná se o~vývojové 
prostředí~(\zkratka{IDE}) obsahující mimo jiné i~nástroj pro návrh grafického 
rozhraní (\textit{\zk{UI}~designer}) a~nástroje pro \textit{debugging}
\footnote{debugging - testování, ladění počítačového programu}.
