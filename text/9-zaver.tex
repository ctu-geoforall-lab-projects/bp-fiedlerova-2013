\chapter{Závěr}
\label{9-zaver}

Cílem této bakalářské práce bylo nastudování algoritmů pro
slučování vektorových map (\textit{vector-to-vector conflation}
a~následné vytvoření zásuvného modulu pro \zkratkatext{QGIS}, 
který pomocí těchto algoritmů umožní zpracování dat. Spojováním
vektorových map se myslí vytvoření nové mapy na~základě
dat ze~dvou různých zdrojů. Samotné algoritmy týkající se
slučování datasetů jsou implementovány v~externí C++ knihovně
\zkratka{GEOC} využívající knihovnu \zk{GEOS}. Kromě tohoto 
je výstupem práce i~přehled některých existujících nástrojů 
zabývajících se slučováním map. 

Algoritmy pro spojování map jsou v~knihovně nezávislé 
na~\zk{QGIS} \zk{API} tak, aby bylo možné knihovnu \zk{GEOC}
případně později využít i~jinými programy. K~problému slučování
vektorových map existuje mnoho přístupů. Aplikace všech by
však byla příliš náročná, proto byl zvolen pouze přístup 
zabývající se pouze geometrickými vlastnostmi dat.

Jelikož se jedná o~poměrně novou oblast v~\zk{GIS}, bylo
obtížné najít literaturu, která by popisovala nějaké obecné 
algoritmy. Většina dosavadních prací či článků se totiž zabývá 
spojováním map pro nějaký specifický případ. Na druhou stranu
jsem se však dověděla mnoho nového o~pro mě donedávna neznámém
tématu.

Výsledkem této práce je C++ knihovna \zk{GEOC}, která 
implementuje tři různé algoritmy pro slučování vektorových map 
na~základě jejich geometrických vlastností, a zásuvný modul 
\textit{Conflate} využívající tuto knihovnu. Přestože knihovnu
i~zásuvný modul je možné ještě v~různých směrech vylepšit,
lze říci, že stanovené cíle práce byly splněny.