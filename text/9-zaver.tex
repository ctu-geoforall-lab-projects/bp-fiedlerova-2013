\chapter{Závěr}
\label{9-zaver}

Cílem této bakalářské práce bylo nastudování algoritmů pro
slučování vektorových map (\textit{vector to vector conflation})
a~následné vytvoření zásuvného modulu pro \zkratkatext{QGIS}, 
který pomocí těchto algoritmů umožní zpracování dat. Kromě 
toho je výstupem práce i~přehled některých existujících 
nástrojů zabývajících se slučováním map.

Výsledný zásuvný modul \textbf{Conflate} z~daných vstupních 
vrstev na~základě para\-metrů zvolených uživatelem vytvoří novou 
vrstvu, která je kombinací původních. Samotné algoritmy týkající 
se slučování datasetů jsou implementovány v~externí C++ knihovně
\zkratka{GEOC} využívající knihovnu \zk{GEOS}. 

Knihovna \zk{GEOC} je navržena nezávisle na~\zk{QGIS} \zk{API} tak, 
aby ji bylo možné případně později využít i~jinými programy. Zatím
obsahuje tři základní algoritmy pro sloučení vektorových dat. 
K~problému sjednocení vektorových map existuje mnoho přístupů. 
Aplikace všech by však byla příliš náročná, proto byl zvolen jediný 
přístup zabývající se pouze geometrickými vlastnostmi dat.


Jelikož se jedná o~poměrně novou oblast v~\zk{GIS}, bylo
obtížné najít literaturu, která by popisovala nějaké obecné 
algoritmy. Většina dosavadních prací či článků se totiž zabývá 
spojováním map pro nějaký specifický případ. Na druhou stranu
jsem se však dověděla mnoho nového o~pro mě donedávna neznámém
tématu.

Přestože knihovnu i~zásuvný modul je možné ještě v~různých směrech 
vylepšit, lze říci, že stanovené cíle práce byly splněny. Nejbližším
krokem by nyní mělo být uvedení projektu do~takového stavu, aby 
mohl být poskytnut uživatelům programu \zkratkatext{QGIS}.